% Chapter Template
% cSpell:words parencite onfwhitepaper includegraphics resizebox sdncomponents
% linewidth comparqui redireccionar enrutamiento subfigure toposdn Nicira toposinsdn
\chapter{Marco teórico} % Main chapter title 

\label{Chapter2} % for referencing this chapter elsewhere, use \ref{ChapterX}
En el presente capítulo, en primer lugar, se explicará un concepto fundamental
sobre el que se basó el proyecto: las redes definidas por software. Estas
representan una transformación, tanto técnica como económica, de las redes tal y
como las conocemos hoy en día. Las ventajas son tales que empresas como Google,
Facebook, China Unicom, AT\&T, Deutsche Telekom, entre otras, intentan migrar
sus redes hacia este nuevo paradigma.

En segundo lugar, los ataques de denegación de servicio serán el objetivo a
describir, ya que representan una amenaza a solucionar tanto en las redes
actuales como en las futuras. Aprovechando las ventajas de la administración de
las redes definidas por software, este proyecto intenta abordar esta
problemática desde un nuevo enfoque.

Luego, se expone una sección con el propósito de destacar las características de
los sistemas de detección de intrusiones, elementos fundamentales para la
detección y mitigación de los ataques de denegación de servicio. Estos forman
parte de la solución brindada a lo largo de este trabajo.

Por último, la principal contribución del presente proyecto es la implementación de
un mecanismo de detección DDoS combinando el uso de los mencionados sistemas de
detección de intrusiones con las redes definidas por software. La
infraestructura apropiada para poder validar correctamente el producto y darle
una mayor relevancia debe ser aquella que maneje una enorme cantidad de
tráfico, presente múltiples subredes y puntos de acceso, posea posibilidades de
escalar en un futuro y mantenga una jerarquía entre los dispositivos de red que
la componen. Un buen escenario, considerando también las características de
algunas de las empresas mencionadas anteriormente, son los proveedores de
servicios de Internet, por lo que se destina una sección al final de este
capítulo en donde se explican su arquitectura y sus componentes.


%---------bitext comileArquitectura del protocolo----------------------------------------
%	SECTION 1
%----------------------------------------------------------------------------------------
\section{Redes definidas por software} \label{sec:sdn}

El mundo SDN entra en el concepto de la \textbf{desagregación}, que consiste
simplemente en el desacoplamiento de dos componentes normalmente integrados, con
el objetivo de crear opciones y flexibilidad. Si los usuarios pueden mezclar y
combinar diferentes tipos y versiones de dichos componentes, entonces pueden
construir un sistema que sea un ajuste exclusivo para su propósito específico.

\subsection{Definición} 

La \textit{Open Networking Foundation} (ONF) \parencite{onf} es una sociedad sin
fines de lucro que impulsa la transformación de la infraestructura de la red y
de los modelos de negocios, dedicándose al desarrollo, estandarización y
comercialización de \textit{Software Defined Networking} (SDN). Ellos nos
brindan la primera definición, en donde SDN es una arquitectura dinámica y
adaptable, muy útil para satisfacer las necesidades variables en ancho de banda
de las aplicaciones actuales por medio de la desagregación de las funciones de
control de las de reenvío de la red. Por un lado, el control de dicha red puede
volverse programable y centralizado y, por el otro, las funciones de reenvío se
efectúan a través de la infraestructura subyacente \parencite{def_citrix}.

El hecho es que las redes de cualquier tamaño necesitan ser controladas. En una
organización pequeña, con tener una sola persona talentosa que pueda diseñar,
construir, administrar y mantener los servidores, \textit{firewalls} y
enrutadores es suficiente. Sin embargo, esto no es sostenible ni aceptable en
redes más grandes. Y es aquí donde SDN cumple un papel muy importante.

\subsection{Arquitectura SDN} \label{sec:arch_SDN}
Antes de entrar en detalles sobre la arquitectura SDN, es necesario especificar
el papel del plano de control y del plano de datos, tanto para las redes
tradicionales como para las que se encuentran dentro del campo SDN.

Si nos situamos en un dispositivo de red, a través de los enlaces (cable, fibra
o medios inalámbricos) éste recibe datagramas que debe procesar y reenviar. El
proceso de reenvío de paquetes mueve los datagramas desde una interfaz de
entrada hacia una de salida apropiada según la información contenida en un
estructura de datos conocida como \textit{Forwarding Information Base} (FIB) que
posee el dispositivo de red en cuestión. Dicha función de reenvío forma parte
del \textbf {plano de datos} \parencite{arch_tradicional}.

Como se mencionó anteriormente, en los dispositivos de red se necesita
información para realizar el proceso de reenvío en forma adecuada. Para obtener
dicha información (por ejemplo, entradas de la tabla FIB), se construye una
vista de la topología de la red y, en base a eso, se arman las tablas de
enrutamiento conocidas como \textit {Routing Information Base} (RIB), a partir
de las cuales se calculan las mejores rutas de acuerdo a determinados criterios
de selección. Lo mencionado es tarea del \textbf {plano de control}. Además
comprende funciones de configuración (configuración de interfaces, configuración
de protocolos, etc.) y el monitoreo del sistema \parencite{arch_tradicional}.

Entrando más en detalle, una tabla RIB presenta entradas que corresponden a
todos los posibles caminos aprendidos por los protocolos administradores de la
topología de la red desde un origen hacia un destino particular. De acuerdo a
estas entradas, una tabla FIB del plano de datos se confecciona a partir de la
elección de uno de todos los caminos posibles, en base a los mencionados
criterios.

Si ahora tenemos en cuenta a las redes tradicionales, sus dispositivos de red
incluyen a ambos planos, el de control y el de datos, e implementan funciones de
los mismos por medio de aplicaciones que son propietarias del fabricante del
equipo. De esta forma, se está en presencia de una de las desventajas
principales de dichas redes: la configuración distribuida y dependiente del mencionado
fabricante del equipo. Lo primero se debe a que como el plano de control se
encuentra en todos los dispositivos de red, los cambios en las configuraciones
deben realizarse equipo por equipo.

En este contexto surgen las redes definidas por software, caracterizadas
principalmente por:

\begin{itemize}
\item \textbf{Separación física del plano de datos del plano de control.} Esto
  garantiza que la infraestructura física de conmutación y de reenvío de paquetes
  se pueda construir a partir de equipos estándar, de menor costo y sin la
  configuración dependiente del fabricante. Estos dispositivos son conocidos como
  conmutadores de caja blanca \parencite{sdn_approach}.
\item \textbf{Gestión central.} El plano de control pasa a localizarse en un
  único ente centralizado programable, conocido como controlador SDN. A través de
  esta gestión central, todos los dispositivos de la red se controlan desde un
  único punto y, de esta forma, se logra acabar con la desventaja de las redes
  tradicionales acerca de la configuración distribuida. En una arquitectura SDN,
  desde el controlador se mantiene una visión global de la topología y se
  simplifica la comunicación entre las aplicaciones, servicios y dispositivos de
  red \parencite{sdn_oreilly}. Esto permite a los administradores monitorear,
  configurar, asegurar y optimizar los recursos de dicha red de acuerdo a
  necesidades específicas. Cabe destacar además que esta gestión central
  programable favorece la consistencia en las políticas de la red y el
  mantenimiento de las aplicaciones.
\end{itemize}

A su vez, entre los dispositivos SDN y el mencionado controlador, existe una interfaz de
programación que separa ambos planos y recibe el nombre de
\textit{Control-Data-Plane Interface (CDPI)}.

Entrando más en detalle, la ONF define una arquitectura de alto nivel para SDN,
con tres capas principales como se ve en la Figura \ref{fig:layers}:

\begin{itemize}
\item Capa de datos.
\item Capa de control.
\item Capa de aplicación.
\end{itemize}

\begin{figure}[H]
	\centering 
	% \resizebox{.85\textwidth}{!}{\includegraphics{Figures/sdn-arch.png}}%
  \includegraphics[scale=0.6]{sdn_layers}
	\caption[Arquitectura SDN.]{Arquitectura SDN \parencite{onf}.}
	\label{fig:layers}
\end{figure}


\subsubsection*{Capa de datos}
La capa de datos forma parte del plano de datos e incluye a dispositivos de red
conocidos como dispositivos SDN. Algunos de estos dispositivos consisten en
conmutadores SDN, virtuales o físicos. A diferencia de aquellos utilizados en
las redes tradicionales, presentan una mayor programabilidad en su
funcionamiento, lo que deriva en un hardware más genérico.

Los conmutadores SDN poseen en su estructura interna una interfaz de
comunicación con el controlador, el cual administra las tablas de flujo de
dicho dispositivo, explicadas posteriormente en la Sección \ref{sec:opflow},
y a partir de las cuales se manejan los flujos de datos y las funciones de
procesamiento de los paquetes.

Los conmutadores SDN virtuales son programas de software que simulan el
comportamiento y la estructura interna de un \textit{switch} SDN. Existen distintos
productos en el mercado, como por ejemplo, Open vSwitch (OVS), Indigo, entre
otros. Su principal diferencia con los dispositivos físicos es que el
procesamiento de paquetes se realiza vía software \parencite{sdn_approach}.

Por otra parte, se encuentran los conmutadores SDN físicos, que permiten
trabajar a mayor velocidad pero su costo es mayor. Huawei, Cisco, HP, IBM,
entre otros, son empresas que proveen estos dispositivos, los cuales poseen
la implementación correspondiente en hardware para el procesamiento de paquetes.


\subsubsection*{Capa de control} \label{sec:control_layer}

La capa de control forma parte del plano de control y se sitúa físicamente en el
controlador SDN, proporcionando servicios de red y actuando como organizadora y
mediadora. Como se observa en la Figura \ref{fig:layers}, la capa de control
está conectada a dos interfaces.

La interfaz \textit{SOUTHBOUND} brinda comunicación con los dispositivos de red
de la capa de datos. El controlador proporciona primitivas a los dispositivos,
generalmente para que éstos informen el estado de la red e importen reglas de
reenvío sobre cómo tratar los paquetes entrantes \parencite{sdn_approach}.

La interfaz \textit{NORTHBOUND} facilita la interacción con las aplicaciones que
se encuentran en la capa superior. El controlador expone funciones y operaciones
de la red a través de interfaces de programación de aplicaciones (APIs)
bidireccionales \parencite{sdn_oreilly}.

En caso de que exista un gran dominio de red administrativa, existirán varios
controladores y se conectarán a través de una interfaz este-oeste. Esto es
necesario ya que los mismos tienen que compartir información de la red y
coordinar sus procesos de toma de decisiones.

\subsubsection*{Openflow}\label{sec:opflow}

En primer lugar, el controlador no puede hacer nada sin un medio que permita dar
instrucciones a los dispositivos de red. Justamente, para desagregar las
operaciones de reenvío de paquetes de las de control de manera adecuada, se debe
mantener la comunicación entre ellas. Así surge una capa de abstracción que
permite a la capa de datos mantener el procesamiento de paquetes y la
comunicación con la capa de control centralizada. Es aquí en donde OpenFlow
puede desempeñarse, permitiendo la separación deseada \parencite{sdn_approach}.

OpenFlow es un conjunto de protocolos y una API. Contiene el protocolo de
comunicaciones entre el plano de datos y el plano de control SDN, posee parte
del comportamiento del plano de datos y forma parte de las interfaces
\textit{SOUTHBOUND}. Además, OpenFlow es no propietario, permite programar el
plano de datos de los conmutadores SDN y su funcionamiento está basado en tres
componentes fundamentales como se observa en la Figura \ref{fig:openflow_1}:

\begin{itemize}
\item Controlador SDN.
\item Dispositivos de red SDN con capacidad OpenFlow.
\item Tablas de flujo de los conmutadores SDN con capacidad OpenFlow.
\end{itemize}


\begin{figure}[H]
	\centering 
	% \resizebox{.85\textwidth}{!}{\includegraphics{Figures/sdn-arch.png}}%
	\includegraphics[scale=1.0]{openflow}
	\caption[Componentes fundamentales de una arquitectura SDN con OpenFlow]{Componentes fundamentales de una arquitectura SDN con OpenFlow
    \parencite{sdn_approach}.}
	\label{fig:openflow_1}
\end{figure}


Teniendo en cuenta esos tres componentes, OpenFlow se caracteriza por permitir,
entre otras cosas, lo siguiente
\parencite{sdn_oreilly}\parencite{sdn_simplified}:

\begin{itemize}
  \item Establecer una sesión de control entre el controlador SDN y el
    conmutador SDN.
  \item Definir una estructura de mensaje para intercambiar modificaciones
    de flujo y recopilar estadísticas.
  \item Definir una estructura de mensajes \textit{PACKET\_IN} y
    \textit{PACKET\_OUT}. El primero se utiliza para que los conmutadores SDN
    le envíen información de los paquetes al controlador para que éste actúe
    como manejador de excepciones. Por otra parte, el segundo se utiliza para
    que el mencionado controlador le envíe a los \textit{switches} SDN paquetes
    que se necesitan enviar por el plano de datos \parencite{sdn_oreilly}.
  \item Definir la estructura fundamental de un conmutador SDN (puertos y
    tablas). 
  \item La configuración y gestión en la asignación de puertos de un
    conmutador físico a un controlador particular.
\end{itemize}

Para lograr esos comportamientos, el dispositivo de red SDN con capacidad
OpenFlow cuenta con un enlace por medio del cual se conecta al controlador. Este
último es quien le configura el comportamiento a partir del intercambio de
mensajes que se envían utilizando TLS sobre TCP \parencite{sdn_oreilly} 
\parencite{sdn_approach}. Obviamente, el que inicia la comunicación es el
dispositivo de red SDN, que conoce la dirección IP del
controlador.

Como se dijo anteriormente, por medio de OpenFlow el controlador puede modificar
el comportamiento de un conmutador SDN. Esto se logra alterando las tablas de
flujo de dicho conmutador, las cuales poseen los siguientes campos principales
en sus entradas \parencite{opf151}, tal y como se observa en la Figura
\ref{fig:openflow_2}:

\begin{itemize}
  \item Cabecera del paquete que define el flujo de paquetes. (\textit{Match}).
  \item Acción o instrucción de cómo se deben procesar los paquetes de un
    determinado flujo. (Descartar flujo de paquetes, modificar campos del flujo
    de paquetes, encapsular y enviar al controlador el flujo de paquetes o
    enviar por un puerto o puertos específicos los paquetes del correspondiente
    flujo).
  \item Estadísticas sobre la cantidad de paquetes y de bytes de un flujo.
    Estadísticas sobre el tiempo de inactividad de dicho flujo.
\end{itemize}

\begin{figure}[H]
	\centering 
	% \resizebox{.85\textwidth}{!}{\includegraphics{Figures/sdn-arch.png}}%
	\includegraphics[scale=0.7]{newnorm}
	\caption[Tablas de flujo de un dispositivo OpenFlow]{Tablas de flujo de un dispositivo OpenFlow
    \parencite{new_norm_for_networks}.}
	\label{fig:openflow_2}
\end{figure}

Finalmente, el conmutador hará la acción correspondiente con los paquetes que le
ingresen de la red, de acuerdo a las coincidencias de éstos con las entradas de
la tabla de flujo, teniendo en cuenta también las prioridades de dichas
entradas.

\subsubsection*{Capa de aplicación}
La capa de aplicación forma parte del plano de control y no se encuentra en el
controlador SDN. Está constituida por aplicaciones que son responsables de
gestionar, configurar, decidir y monitorear las características de una red SDN
como, por ejemplo, las entradas de las tablas de flujo que están programadas en
los dispositivos de red. Estas aplicaciones, utilizando la interfaz
\textit{NORTHBOUND}, no solamente influyen en la capa de datos, sino también en
la de control \parencite{sdn_simplified}. Por medio de APIs expuestas por el
controlador, dichas aplicaciones pueden, por ejemplo, acceder a los datos del
estado de la red para reaccionar a cambios de la topología, para lograr un
equilibrio de carga \parencite{load_bal}, para seleccionar las mejores rutas de
los flujos o para aplicaciones de comunicación como la priorización de VoIP
\parencite{voip}. También pueden redireccionar el tráfico con el fin de
inspeccionarlo en el caso de aplicaciones de seguridad \parencite{autoscaling}.

\subsubsection*{Seguridad de la Red} \label{sec:segnetsdn}
Los requerimientos de las aplicaciones de redes tradicionales requieren el
despliegue de una política de seguridad por toda la red y de una configuración
complicada e individual en los \textit{firewalls} \parencite{distr_firewall} y/u
otros dispositivos \parencite{detect_chi}. SDN ofrece una plataforma adecuada
para centralizar, combinar y verificar políticas y configuraciones, a fin de
garantizar una buena implementación de la protección y de la prevención de
problemas de seguridad de manera proactiva en la red. Esto significa que una
empresa puede ampliar sus capacidades de defensa bloqueando ataques específicos
y haciendo cambios para adaptarse a las nuevas amenazas.

El controlador SDN puede impulsar las actualizaciones de la política de
seguridad global de forma centralizada a través de la red, y un dispositivo
puede filtrar paquetes y redirigir el tráfico sospechoso a otros dispositivos de
seguridad para su posterior análisis.

\subsubsection*{Ventajas y desventajas de SDN con respecto a las redes tradicionales}
De acuerdo a \parencite{RFC7149}, \parencite{sdn_universidad} y
\parencite{sdn_approach}, una de las ventajas de SDN con respecto a las redes
tradicionales comprende la configuración centralizada e independiente de los
fabricantes de los equipos de red, lo que permite una mayor consistencia en las
políticas de red y un favorecimiento a la escalabilidad, agilidad, flexibilidad
y posibilidad de innovación, ya que en menor tiempo las empresas pueden
desplegar o modificar servicios, aplicaciones e infraestructura de sus redes. 


Otra de las ventajas consiste en la disminución de ciertos costos. Por un lado,
una reducción del costo de capital (CAPEX) ya que el objetivo de SDN es utilizar
estándares abiertos. Las empresas pueden usar fácilmente opciones de múltiples
proveedores y no estar limitadas a un solo vendedor. Por el otro, una disminución
del costo de operación (OPEX), ya que el esfuerzo y el tiempo necesarios para
realizar tareas de modificación y/o mantenimiento se reduce. Además si las
configuraciones de red se pueden hacer de manera automatizada mejora la
trazabilidad de las operaciones.

Teniendo en cuenta las mismas fuentes de información, entre las
desventajas se encuentran la escalabilidad de la red limitada a la capacidad del
controlador y las inconsistencias que sufren las soluciones actuales en algunas de
sus características y funciones, por tratarse de un paradigma nuevo. Con respecto 
a la seguridad existe la posibilidad de dejar fuera de servicio a la red si se 
efectúa un ataque al controlador (lógica centralizada), si se intenta suplantar 
la identidad del mismo o si se trata de interrumpir la comunicación entre éste y 
los dispositivos SDN.


\section {Denegación de servicio}	

En pocas palabras, en \parencite{ddos_amir} y \parencite{ddos_kumar} se hace
referencia a la situación en la que el usuario normal no puede acceder a un
servicio debido a intentos malintencionados que impiden el acceso al mismo. Para
el caso de una red, el impedimento podría venir dado por la disminución del
ancho de banda o por el consumo de los recursos que sustentan el servicio.

\subsection {DoS (Denegación de servicio)}

Todo el tráfico creado maliciosamente proviene de una sola fuente, la cual
inunda la red con un tráfico abrumador para evitar o disminuir transacciones
legítimas \parencite{ddos_mirkovic}. El atacante puede (aleatoriamente) cambiar
la dirección IP de origen en un intento de ocultarse pero, en este tipo de
ataques, solo hay un remitente que elabora el tráfico que luego se dirige
directamente al objetivo sin que se utilice ningún \textit{host} intermedio. Por
ende, dichos ataques ya no son muy comunes debido a que son fáciles de detectar,
atribuir y mitigar \parencite{ddos_hh}.

\subsection {DDoS (Denegación de servicio distribuido)}

Consiste en la versión amplificada del anterior. En este tipo de ataque, varios
\textit{hosts} (a veces cientos de miles) se centran en un solo objetivo. Cada
computadora con acceso a Internet puede comportarse como un atacante, momento en
el que pasan a ser conocidas como \textit{bots} (\textit{zombies}) y el conjunto
de ellos forman la denominada \textit{botnet} \parencite{ddos_amir}.

En un ataque DDoS, el atacante infecta una red de computadoras (\textit{botnet})
con algún tipo de \textit{malware} que le permite manejarlas de manera remota
\parencite{ddos_kumar}. Cuando llegue la ocasión, el atacante envía la orden al
\textit{botnet}, donde cada uno de los \textit{bots} integrantes lanzan un
ataque hacia el \textit{host} víctima (ver Figura \ref{fig:ddos_flood}). De esta
forma, el verdadero atacante mantiene su identidad oculta.

Los ataques de denegación de servicio distribuido pueden intentar limitar el
ancho de banda de los enlaces de la red o consumir los recursos del
\textit{host} víctima. A continuación, se explicarán en las siguientes
subsecciones los ataques que afectan al ancho de banda, los cuales se
clasifican, de acuerdo a \parencite{ddos_amir} y \parencite{ddos_hh}, en ataques
por inundación (\textit{flood}) y en ataques por reflexión y amplificación.
\\

\begin{figure}[H]
	\centering 
	% \resizebox{.85\textwidth}{!}{\includegraphics{Figures/sdn-arch.png}}%
	\includegraphics[scale=0.7]{ddos_flood}
	\caption[Ataque DDoS]{Ataque DDoS \parencite{ddos_amir}.}
	\label{fig:ddos_flood}
\end{figure}


\subsubsection*{Inundación}

El ataque de inundación (\textit{flooding}) está diseñado para hacer que una red
o servicio no se encuentre disponible para usuarios legítimos debido a su
colapso por las grandes cantidades de tráfico que recibe proveniente de
distintos \textit{hosts} tales como computadores de oficina, portátiles,
servidores o dispositivos IoT. Un solo \textit{host} puede generar decenas de
megabits por segundo de tráfico malicioso y, mientras más \textit{hosts} generen
este tipo de tráfico, mayor será la efectividad del ataque \parencite{ddos_hh}.

Una vez que la \textit{botnet} está establecida, el atacante puede dirigir el
atentado enviando instrucciones a cada \textit{bot} vía control remoto. Tal como
se observa en la Figura \ref{fig:ddos_flood}, cuando la mencionada
\textit{botnet} fija la dirección IP de una víctima, cada \textit{bot}
responderá enviando solicitudes al objetivo, lo que posiblemente causará que el
servidor o la red desborden su capacidad, resultando en una denegación de
servicio al tráfico normal. En vista de que cada \textit{bot} es un dispositivo
de Internet genuino, puede ser difícil determinar una fuente concreta del
tráfico malicioso \parencite{ddos_flloogind}.

De acuerdo a \parencite{ddos_hh}, el primero de estos ataques por inundación fue
el denominado \verb|ping flood|, en donde se enviaron tantos paquetes de eco
ICMP (\verb|ping|) a un \textit{host} como fue posible. Luego fueron
reemplazados por la inundación de paquetes TCP explicados en la Sección
\ref{sec:flood} que, debido a la necesidad de tres paquetes para iniciar una
conexión, pueden generar más tráfico y, a su vez, permiten lograr un mayor
consumo de los recursos de las víctimas. Posteriormente, estos últimos ataques
fueron reemplazadas por inundaciones de paquetes UDP ya que, al ser un protocolo
sin estado de la conexión, permite que se falsifique la dirección IP de origen,
dificultando el rastreo y el bloqueo de dichas inundaciones.


\subsubsection*{Reflexión y amplificación}

Esta categoría de ataques de amplificación o reflexión, también llamados DRDoS
(\textit{Distributed Reflection Denial of Service}), utilizan fallas de
protocolo y otras vulnerabilidades para amplificar las cantidades de datos
transmitidos contra un sistema objetivo. Dichos ataques utilizan dispositivos
conocidos como amplificadores o reflectores, los cuales pueden ser máquinas
legítimas o dispositivos intermedios y, en comparación con las redes de
\textit{bots}, generalmente no están infectados y no están controlados
\parencite{drdos_kumar}.

El primer principio sobre el que se basa este tipo de ataques es la reflexión,
la cual se produce cuando un atacante falsifica la dirección IP de origen de los
paquetes, no con una dirección aleatoria sino simulando ser la víctima
\parencite{ddos_amir}, y enviando los mencionados paquetes a \textit{hosts} que
no pueden distinguir las solicitudes legítimas de las falsificadas (en especial
si se utiliza UDP o ICMP, al ser protocolos que no mantienen el estado de la
conexión). Por lo tanto, responden directamente a la víctima como se observa en
la Figura \ref{fig:ddos_reflexion}.

El segundo principio es la amplificación. Su concepto es hacer que el
\textit{host} intermedio, el cual no es consciente de ser un amplificador,
desencadene una respuesta con un mayor tamaño que la petición inicial. Esto
multiplica de manera efectiva el flujo de tráfico del atacante
\parencite{ddos_andre}. El impacto en estos \textit{hosts} de amplificación es
mínimo, no obstante el tráfico dirigido al objetivo podría ser mucho más
perjudicial.

Ejemplificando, uno de los ataques de reflexión y amplificación es Smurf,
descrito en la Sección \ref{sec:smurf}, que utiliza una dirección falsificada
y el direccionamiento de difusión para amplificar un flujo de paquetes. El
sistema de amplificación objetivo es una red que permite la comunicación con la
dirección de difusión (\textit{broadcast}) y posee un número relativamente alto
de \textit{hosts} activos \parencite{art_exploit}.

Otro de los ataques más frecuentemente utilizados es el ataque de amplificación
DNS, en el cual los atacantes usan servidores DNS de acceso público enviando
solicitudes de búsqueda, con el objetivo de inundar un sistema de destino con
tráfico de respuesta DNS. La solicitud demanda tanta información como sea
posible para maximizar el efecto de la amplificación \parencite{CISA}.


\begin{figure}[H]
	\centering 
	% \resizebox{.85\textwidth}{!}{\includegraphics{Figures/sdn-arch.png}}%
	\includegraphics[scale=0.7]{ddos_refle}
	\caption[Ataque DDoS por reflexión y amplificación]{Ataque DDoS por reflexión y amplificación \parencite{ddos_amir}.}
	\label{fig:ddos_reflexion}
\end{figure}


\section{Detección de intrusos}

La detección de intrusos es el arte de detectar actividad no autorizada
\parencite{ddos_hh}, como son los intentos exitosos o no de conexiones, inicios
de sesión y accesos a recursos no autorizados.

\subsection{Detección de intrusos basada en firmas}
\label{subsec:deteccion_firmas}
 
La detección de intrusos basada en firmas implica buscar un patrón de
comportamiento malicioso en el tráfico de la red como, por ejemplo, una
secuencia de paquetes, una cantidad específica de paquetes de cierto tipo o una
serie de bytes determinados al analizar el encabezado de un paquete o la carga
útil del mismo \parencite{ids_snort}. Si la actividad coincide con alguna de las
firmas de la base de datos de un sistema de detección de intrusiones, éste emite
una alarma indicando una actividad sospechosa.

Las mencionadas firmas de estos sistemas se construyen en base a las reglas
definidas en el lenguaje que cada uno proporciona. Estas reglas se encuentran en
bases de datos y se actualizan de manera progresiva. A su vez los sistemas de
detección de intrusiones, al examinar la información del encabezado del paquete,
buscan coincidencias en las direcciones, protocolos, opciones, parámetros del
fragmento, etc. Por otro lado, miran dentro de la carga útil del paquete, en
búsqueda de URLs específicas o mal formadas \parencite{ids_signatures}.

La gran desventaja es que estos sistemas solo detectan ataques conocidos. Un
ataque desconocido, al no coincidir con ninguna entrada en la base de datos, es
totalmente ignorado. Para agregar una entrada es necesario comenzar con la
comprensión del comportamiento del ataque de red para desarrollar la firma
correspondiente. Además, estas firmas también son propensas a falsos positivos,
ya que generalmente se basan en expresiones regulares y concordancia de cadenas
\parencite{ddos_foster}.


\subsection {Detección de intrusos basada en anomalías}

La técnica de detección de anomalías se centra en el concepto de un
comportamiento esperado de la red en lugar de buscar patrones en una base de
datos que se tiene que actualizar de manera progresiva. Este comportamiento de red
aceptado, es aprendido y/o determinado por los administradores de dicha red.
Cualquier comportamiento que se encuentre fuera del modelo predefinido o
aceptado (por ejemplo, una conexión inesperada a un \textit{host}, puerto)
genera eventos por detección de anomalías \parencite{ddos_foster}. Este sistema,
mientras más tiempo se encuentre operativo, más aprenderá de la red, además de
que posee el potencial de descubrir amenazas desconocidas.

Sin embargo, las desventajas de estos motores es el resultado de la detección.
Muchas veces las alarmas manifiestan más falsos positivos que la detección
basada en firmas. Además, si solamente se destina el sistema para la detección
de amenazas conocidas, la cantidad de recursos necesarios es elevada en
comparación con el sistema de la sección \ref{subsec:deteccion_firmas}
\parencite{ids_vs}.


\subsection{IDS (Sistema de detección de intrusiones)}

Los sistemas de detección de intrusiones (IDS, por sus siglas en inglés) son
programas de detección de actividades maliciosas y accesos no autorizados en un
computador o una red, generalmente utilizando una combinación de métodos basados
en el comportamiento y/o las firmas para detectar amenazas desconocidas como
conocidas, respectivamente \parencite{ids_snort}. Estos sistemas suelen tener
sensores virtuales, es decir, elementos capaces de capturar, procesar, descifrar
y descomponer paquetes de un segmento de red en tiempo real
\parencite{ids_snort}. Dichos elementos son conocidos como \textit{sniffers} de
red y, en base a ellos, el motor del IDS puede obtener datos externos sobre el
tráfico de la red y detectar la presencia o no de ataques. No obstante, estos
IDSs por sí solos no pueden cortar los mencionados ataques. Es por eso que
generalmente se los integra con un \textit{firewall}. Este último consiste en un
elemento de red que controla el cruce de paquetes a través de los límites de una
red en función de una política de seguridad específica
\parencite{fwll_springer}.

El funcionamiento de los sistemas IDS se basa en la generación de alertas cuando
el tráfico de red inspeccionado coincide con firmas de ataques conocidos o
comportamientos sospechosos (paquetes malformados, escaneos de puertos, etc.)
que se encuentran en la base de datos del correspondiente IDS. También, este
sistema no sólo analiza qué tipo de tráfico es, sino que también revisa el
contenido y su comportamiento.

Por último, la clasificación de los sistemas de detección de intrusiones es la
siguiente:
\begin{itemize}
\item \textbf{HIDS} (\textit{Host} IDS): el HIDS solo protege el sistema
  \textit{host} en el que reside, detectando modificaciones o rastros producto de
  las actividades de los intrusos en el equipo atacado cuando éstos intentan
  adueñarse del mismo \parencite{ids_toolkit}.
\item \textbf {NIDS} (\textit{Network} IDS): se trata de un tipo de IDS que
  detecta ataques que se efectúan a todo el segmento de la red. Su
  funcionamiento se detalla en la subsección siguiente y es el que se utilizará
  en el presente proyecto integrador.
\end{itemize}

\subsubsection* {Sistema de detección de intrusiones en una red} \label{sec:NIDS}

Si nuestro sistema en riesgo se trata de una red, la detección consiste
simplemente en tratar de detectar los signos de un intruso o de alguien con
malas intenciones en la red antes de que se produzcan daños, ya sea por la
pérdida de datos o por la denegación de un servicio.

Un NIDS trabaja con los datos que circulan a través de un segmento de red y el
tipo de sensor utilizado en este sistema puede ser un \textit{sniffer} de red
\parencite{ids_toolkit}\parencite{ids_snort}. A partir de dicho sensor se
analizan todos los paquetes, buscando en ellos patrones sospechosos. Los NIDS no
sólo vigilan el tráfico entrante, sino también el saliente y el tráfico local,
ya que algunos ataques podrían iniciarse desde el propio sistema protegido
\parencite{ids_snort}.


\paragraph {NIDS/Firewall.}

Como se mencionó anteriormente, en los sistemas tradicionales un NIDS no puede
cortar un ataque. Por ende suele estar acompañado de un \textit{firewall}.
Entonces, la combinación de componentes de software o de hardware de NIDS y de
\textit{firewall} se utilizan para evitar los accesos no autorizados, en donde
el NIDS actúa como sensor y el firewall como actuador. (Ver Figura
\ref{fig:firewall_NIDS}).

\begin{figure}[!]
	\centering 
	% \resizebox{.85\textwidth}{!}{\includegraphics{Figures/sdn-arch.png}}%
	\includegraphics[scale=0.7]{nids}
	\caption[Combinación de NIDS con \textit{firewall}]{Combinación de NIDS con \textit{firewall} \parencite{ids_snort}.}
	\label{fig:firewall_NIDS}
\end{figure}

\paragraph{Ubicación de NIDS en una red.}

Acorde a \parencite{ubicacion_nids}, los modos de ubicación de los dispositivos
NIDS en una red son:

\begin{itemize}
\item \textbf{Delante del firewall:} genera una gran cantidad de \textit{logs},
  muchos de ellos innecesarios debido a que algunos ataques no sobrepasan a
  dicho \textit{firewall}, por lo que no resultan efectivos.
\item \textbf{Detrás del firewall:} es la ubicación más común. Genera menor
  cantidad de \textit{logs} ya que monitorea y detecta ataques que atraviesan al
  correspondiente \textit{firewall}. A su vez, el emplazamiento del NIDS puede
  conectarse a un \textit{hub}, actuando como \textit{bridge}, o al
  puerto replicador de un \textit{switch}.

\item \textbf {Combinación de casos anteriores:} utiliza dos máquinas ya que se
  requiere la implementación de los dos ítems explicados anteriormente. Permite
  ejercer un mayor control pero usa el doble de la cantidad de hardware. (Ver Figura
\ref{fig:firewall_NIDS}).
\item \textbf {Máquina única firewall/NIDS:} utiliza una sola máquina que cumple
  con las funciones del firewall y del NIDS.
\item \textbf {Otras ubicaciones:} usada para ofrecer distintos grados de
  seguridad, como es el caso de IDSs monitoreando el tráfico de distintos
  segmentos de red.
\end{itemize}

\section {Proveedor de Servicios de Internet} \label{sec:isp}

En \parencite{isp_sun} se define a un ISP como aquel sistema que proporciona
servicios de Internet a suscriptores comerciales y residenciales (usuarios).
Además, es el encargado de conectar a los proveedores de servicios básicos como
correo electrónico, alojamiento web y noticias con los usuarios finales. Por
otra parte, los ISP ofrecen sus propios servicios de valor agregado como, por
ejemplo, la seguridad.

\subsection{Infraestructura básica}

En \parencite{isp_sun} y \parencite{isp_cisco} una infraestructura ISP consta de
las siguientes partes:

\begin{itemize}
\item \textbf{Capa central o capa núcleo.} Es la red troncal de conmutación de
  alta velocidad de la red. Un enrutador central debe poder admitir múltiples
  interfaces de alta velocidad.
\item \textbf{Capa de distribución de la red.} Es el punto de delimitación entre
  la capa central y la de acceso.
\item \textbf {Capa de acceso.} Consiste de dos partes: 
\begin{itemize} 
\item \textbf{Subcapa de servicio.} Es aquella que proporciona acceso a los
  servicios. Es comúnmente donde se encuentran los servidores, como el servidor
  web, el servidor de correo electrónico, el servidor proxy y los
  \textit{firewalls}.
\item \textbf{Subcapa de acceso.} Las funciones principales de la capa de acceso
  son conectar varias redes de área local (LAN) de usuarios finales a la capa de
  distribución. Ésta capa se caracteriza por un entorno de ancho de banda
  compartido.
\end{itemize}
\item \textbf{Capa de frontera.} Es aquella que permite la comunicación entre
  distintos proveedores de servicio.
\end{itemize}

\subsection{Tipos de enrutadores}
Los tipos de enrutadores en la infraestructura de un ISP, tal y como muestra la
Figura \ref{fig:ISP}, son los siguientes \parencite{isp_cisco}:

\begin{itemize}
\item \textbf{Enrutadores principales (Nucléo):} se ubican en la capa central y presentan
  múltiples interfaces de alta velocidad.
\item \textbf {Enrutadores de distribución:} se encuentran en la capa de
  distribución y presentan una mayor densidad de puertos y una menor velocidad
  que los enrutadores anteriores.
\item \textbf {Enrutadores de acceso:} se ubican en la subcapa de acceso. Presentan 
una mayor densidad de puertos, una menor velocidad y brindan acceso a Internet a los
usuarios finales.
\item \textbf {Enrutadores de frontera:} se encuentran en la capa de frontera y
  presentan conexiones hacia otros proveedores de servicio.
\item \textbf {Enrutadores de servicios:} se ubican en la subcapa de servicio y le 
brindan acceso a Internet a los servidores y a los distintos servicios.
\end{itemize}


\begin{figure}[H]
	\centering 
	\includegraphics[scale=0.7]{ips}
	\caption[Topología de un ISP tradicional sin capa de frontera]{Topología de un ISP tradicional sin capa de frontera \parencite{isp_cisco}.}
	\label{fig:ISP}
\end{figure}