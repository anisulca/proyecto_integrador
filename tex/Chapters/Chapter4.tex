% Chapter Template
% cSpell:words parencite onfwhitepaper includegraphics resizebox sdncomponents
% linewidth comparqui redireccionar enrutamiento subfigure toposdn Nicira toposinsdn
\chapter{Aplicación de detección de anomalías} % Main chapter title 

\label{sec:Chapter4} % Change X to a consecutive number; for referencing this
                     % chapter elsewhere, use \ref{ChapterX}

En el presente capítulo se explicará el diseño y la implementación de la
aplicación que se situará en el núcleo de ONOS (ver Figura
\ref{fig:arquitectura_onos}) y que estará encargada de detectar comportamientos
sospechosos en la red y de disminuir la sobrecarga de tráfico en los sistemas de
detección de intrusos.

Para ello, en primer lugar, se describen los motivos que desencadenan la
necesidad del desarrollo de esta aplicación. En segundo lugar, se detallan los
requerimientos de la misma. Y en tercer lugar, el principio de funcionamiento de
dicha aplicación se desglosa y se explica etapa por etapa, realizando el
correcto mapeo de las explicaciones con los requerimientos correspondientes. Por
último, se detalla la implementación llevada a cabo, en donde se hace foco en
las funciones más importantes.

\section{Motivación} \label{sec:motivacion_app_4}

Por lo discutido en la Sección \ref{sec:state_art}, una parte de los objetivos
principales del proyecto consiste en reducir el procesamiento de los
dispositivos de detección.

Uno de los factores más importantes es la cantidad de tráfico que analizan estos
dispositivos, debido a que existen reglas que no solo inspeccionan las cabeceras
de los paquetes sino también la carga útil de los mismos. De este modo, cuanto
mayor sea el tráfico mayor es el procesamiento que éstos realizan. A su vez,
dicha cantidad de tráfico es dependiente, entre otras cosas, de la ubicación de
los IDSs dentro de la topología.


\paragraph{Enfoques de emplazamiento del IDS.}
De acuerdo a \parencite{ddos_kumar}, cuanto más cerca esté el detector a la
posición del servicio objetivo, mayor será la cantidad de tráfico que analiza.
Tal enfoque es capaz de proporcionar una mejor precisión de detección, debido a
que posee un gran dominio para analizar el tráfico. Sin embargo, este enfoque
solamente detecta el ataque después de que llega a la víctima. Además, si bien
se puede cortar el tráfico malicioso en el enrutador más cercano a ella, el
legítimo aún no puede viajar a través de la red y llegar a su destino, debido a
que el ancho de banda está afectado. Por otro lado, si el emplazamiento es más
cercano a los dispositivos de borde directamente conectados a los \textit{hosts}
maliciosos, se puede evitar la congestión no solo del lado de la víctima, sino
también de la red, sin afectar a los demás usuarios legítimos. Sin embargo,
durante un ataque, generalmente el tráfico no proviene de una sola fuente, sino
de manera distribuida por múltiples puntos de acceso. Esto dificulta la
detección debido a la ubicación de los detectores de intrusos. Para resolver los
problemas de precisión a la hora de percibir dichos ataques y evitar el consumo
de ancho de banda por parte de éstos, se encuentra la defensa de red que ubica a
los IDSs en una zona intermedia \parencite{ddos_kumar}. Este último es el
enfoque que se utilizará en el presente proyecto.

\paragraph{Problemas de procesamiento frente a la escalabilidad de la red.} A
medida que la red escala se imposibilita que los sistemas de detección de
intrusiones analicen todo el tráfico que atraviesa a ciertos dispositivos de
red, independientemente de su ubicación. Para evitar la sobrecarga, se puede
plantear que los análisis en los IDSs sean temporales y solamente en aquellos
casos en donde se detecten comportamientos sospechosos, en base a uno esperado.
Para esto último se requiere un análisis estadístico llevado a cabo por el
controlador, el cual posee información global de la red y decide si existe o no
una anomalía.

\paragraph{} En base a lo descrito anteriormente y a lo mencionado en la Sección
\ref{sec:segnetsdn} surgen los requerimientos explicados en la siguiente
sección.

%---------------------bitext comileArquitectura del protocolo--------------------
%	SECTION 1
%---------------------------------------------------------------------------------
\section{Requerimientos} 

Los requerimientos funcionales de la aplicación son los siguientes:

\paragraph{RF01.} La aplicación debe recopilar métricas cada 10 segundos del
tráfico entrante a la zona intermedia (dispositivos de distribución) de la red.
\paragraph{RF02.} La aplicación debe almacenar las métricas totales por día y
debe mantener una ventana con los últimos 7 días.
\paragraph{RF03.} La aplicación, usando dichas métricas, debe detectar
comportamientos sospechosos en base a uno esperado.
\paragraph{RF04.} La aplicación debe identificar a los dispositivos de la
subcapa de acceso que presenten un comportamiento sospechoso.
\paragraph{RF05.} La aplicación, ante la presencia de tráfico potencialmente
malicioso, debe replicarlo temporalmente hacia los IDSs más cercanos a los
dispositivos detectados en RF04, al menos el tiempo mínimo necesario para que
Snort pueda reconocer si se está en presencia de un ataque o no.

\section{Principios de funcionamiento}

A continuación, se desarrolla una introducción a la lógica detrás de la
aplicación de detección de anomalías, para facilitar la futura comprensión de la
estructura y el funcionamiento de la misma. De manera resumida, el
comportamiento de esta aplicación se puede observar en el diagrama de la Figura
\ref{fig:diagram1}.

\begin{figure}[H]
	\centering 
	\includegraphics[scale=0.7]{activity_1}
	\caption{Diagrama de actividades de la aplicación.}
	\label{fig:diagram1}
\end{figure}

\subsection{Recolección de métricas}

De acuerdo a lo descrito en la Sección \ref{sec:control_layer}, la interfaz
SOUTHBOUND permite a los dispositivos informar el estado de los mismos usando
OpenFlow. Esto le posibilita al controlador obtener métricas de los puertos,
tales como la cantidad de paquetes y la cantidad de bytes que los atraviesan,
mediante \textbf{contadores} en el plano de datos \parencite{opf151}.

Por otro lado, basándonos en el sistema propuesto por \parencite{estado_arte_2}
para la detección de ataques distribuidos, se plantea uno propio aplicando a una
topología de red ISP la gestión SDN. Para ello, es necesario identificar los
dispositivos (de la red ISP) que proporcionan las métricas y, en conjunto con el
enfoque de emplazamiento de los sistemas de detección explicado al inicio de
esta sección, se propone la recolección de dichas métricas desde los conmutadores
identificados como \textit{distribution}, tal como se muestra en la Figura
\ref{fig:ISP}. Estos dispositivos se encuentran en una ubicación intermedia al
mantener conexiones tanto con las capas de acceso como con las capas del núcleo.

Como se mencionó anteriormente, gracias a OpenFlow se pueden obtener desde el
plano de datos las métricas de todos los dispositivos y de todos sus puertos
(\textbf{RF01}). No obstante, todas ellas no son necesarias para el análisis
posterior. Sólo son indispensables aquellas pertenecientes a los puertos de los
conmutadores SDN de distribución cuyos enlaces se conecten a los dispositivos de
borde de la subcapa de acceso. Así sólo se deben obtener métricas del tráfico
entrante a la red, tal como se observa en la Figura \ref{fig:trafico_entrante}.


\begin{figure}[H]
	\centering 
	\includegraphics[scale=1.5]{intraffic_1}
	\caption{Tráfico entrante desde los dispositivos de acceso a los de
    distribución.}
	\label{fig:trafico_entrante}
\end{figure}


\subsection{Detección de  anomalías}

La detección de anomalías se resuelve realizando un análisis estadístico basado
en la prueba de bondad de ajuste chi-cuadrado. Ésta se utiliza para comprobar si
una muestra de datos sigue una distribución determinada. La manera de definirlo
es realizando una prueba de hipótesis para verificar si dicha muestra difiere de
una población de datos esperada. La forma de cuantificar las diferencias es
usando (\ref{eq:chi_cuadrado}).

\begin{equation}\label{eq:chi_cuadrado}
\chi^2={\sum_{i=0}^{k} \left [ \frac{(O_{i} - E_{i})^2}{E_{i}} \right ]}
\end{equation}\\

En donde \(O_{i}\) representa los datos observados y \(E_{i}\) es el valor
esperado para cada observación.

Para nuestro proyecto estas variables están definidas por la cantidad de
paquetes y la cantidad de bytes pertenecientes al tráfico que ingresa a la red.
El primero se utiliza para saber si un ataque afecta a los recursos de los
dispositivos de la red, mientras que el segundo se usa para determinar si existe
un impacto en el ancho de banda \parencite{estado_arte_2}. Los datos anteriores
se comparan con valores esperados. Si éstos difieren estadísticamente, se
considera un tráfico con un comportamiento sospechoso (\textbf{RF03}).

Por otro lado, los valores esperados no pueden ser estáticos. De ser así, no se
considerarían las variaciones del tráfico legítimo a largo plazo y se podría
estar etiquetando a dicho tráfico legítimo como sospechoso, dando lugar a las
detecciones conocidas como \textbf{falsos positivos}. Esto explica el por qué
del requerimiento \textbf{RF02}, ya que es necesario conocer la cantidad de
tráfico que circulaba en la red anteriormente.

Finalmente, una vez considerado el tráfico sospechoso, el diseño del
\textbf{RF04} se basa en identificar los dispositivos de distribución que
superan una cierta diferencia estadística con respecto a sus valores esperados.
Luego, se deben determinar cuáles de los conmutadores de la subcapa de acceso
conectados a dichos \textit{distribution} presentan un porcentaje de tráfico que
compromete a la red.

\subsection{Duplicación del tráfico}

Una vez que se clasifica el tráfico entrante como sospechoso, es necesario
enviarlo a un IDS para su inspección.

Una primera solución sería desviar el tráfico completamente hacia el detector de
ataques, pero como el marcado es sobre los puertos del dispositivo
\textit{distribution}, el tráfico proveniente de éstos también incluye tráfico
legítimo. En este caso, si se realiza el completo redireccionamiento, se estaría
cortando el servicio a clientes legítimos.

Otra solución es la duplicación temporal del tráfico, usando las directivas de
la interfaz SOUTHBOUND para aplicar reglas con el fin de enviar este tráfico
tanto a su destino original como al IDS, tal como se observa en la Figura
\ref{fig:dupl_trafico}. Finalmente, ésta es la solución elegida para el
\textbf{RF05}.

\begin{figure}[H]
	\centering 
	\includegraphics[scale=1.5]{intraffic_2}
	\caption{Duplicación de tráfico sospechoso.}
	\label{fig:dupl_trafico}
\end{figure}

Un aspecto importante a tener en cuenta es que si se duplica todo el tráfico
entrante a un dispositivo \textit{distribution}, se estaría enviando información
innecesaria al sistema de detección de intrusos. Es por ello que solo se debe
reproducir aquel tráfico proveniente de los puertos con comportamientos que
difieren en una cierta proporción a lo esperado y cuyos enlaces se conecten a
conmutadores de la subcapa de acceso, tal y como se mencionó anteriormente.

\section {Detalles de la implementación}

En esta sección se hace hincapié en los detalles de la implementación haciendo
foco en las funciones más importantes con el fin de cumplir con los
requerimientos funcionales. Para ello se tiene en cuenta y se adapta lo
propuesto en \parencite{estado_arte_2}. A su vez, una visión estática y global
de lo efectuado se encuentra en las Figuras \ref{fig:diagrama_clases_1} y
\ref{fig:diagrama_clases_1_2}.

\begin{figure}[H]
	\centering 
	\includegraphics[width=0.9\textwidth]{Clases1}
	\caption{Diagrama de clases para la detección de anomalías.}
	\label{fig:diagrama_clases_1}
\end{figure}

Para la recolección de las métricas (\textbf{RF01}), el controlador ONOS
mediante la capa Proveedores (ver Figura \ref{fig:arquitectura_onos}) ofrece
estadísticas de los puertos de los conmutadores SDN mediante el servicio
\verb|deviceService|. Éste brinda dos métodos importantes:

\begin{itemize}
\item \verb|getPortDeltaStatistics (DeviceId deviceId)|. Obtiene una lista con
  todas las estadísticas de un dispositivo en un intervalo de tiempo, cuyo valor
  por defecto es de 10 segundos.
\item \verb|getPortStatistics (DeviceId deviceId)|. Obtiene una lista con el
  total de las estadísticas de cada puerto de un dispositivo desde que empieza a
  enviar y recibir tráfico.
\end{itemize}

El primer método permite obtener métricas de los dispositivos cada 10 segundos.
El segundo facilita el diseño e implementación del \textbf{RF02}, ya que basta
con almacenar el valor dado por este método al inicio y al final del día para
luego realizar una diferencia. Luego se conforma una cola con estas diferencias
en donde también se almacenan los valores obtenidos de los 7 días anteriores.

Para la implementación del \textbf{RF03}, primero se obtienen los valores
estadísticos esperados bajo circunstancias normales para el tráfico entrante a
un dispositivo \textit{distribution} \(i\), desde la subcapa de acceso, para un
intervalo de tiempo de 10 segundos (\textbf{\(G_i\)}), a partir de las colas
mencionadas anteriormente. Esto se realiza tanto para la cantidad de paquetes
como para la cantidad de bytes, tal y como se observa en (\ref{eq:gicount}) y
(\ref{eq:gisize}).

\begin{equation}
	cant\_paquetes\_semana_i={\sum_{j=1}^7 cant\_paquetes\_dia_{j}}
\end{equation}

\begin{equation}\label{eq:gicount}
    G_{i\_paquetes}={\frac{cant\_paquetes\_semana_i}{\frac{7*24*60*60}{10}}}
\end{equation}

\begin{equation}
	cant\_bytes\_semana_i={\sum_{j=1}^7 cant\_bytes\_dia_{j}}
\end{equation}

\begin{equation}\label{eq:gisize}
	G_{i\_bytes}={\frac{cant\_bytes\_semana_i}{cant\_intervalos}}
\end{equation}

\begin{equation}\label{eq:intervalo}
  cant\_intervalos=\frac{7\ dias * 24\frac{horas}{dia} * 60\frac{minutos}{hora} * 60\frac{segundos}{minuto}}{10 \ segundos}
\end{equation} \\

Luego se ejecuta la prueba de bondad de ajuste chi-cuadrado, la cual está dada
por (\ref{eq:Xcount}) y (\ref{eq:Scount}) y se calcula cada vez que se obtienen
métricas nuevas cada 10 segundos.

\begin{equation}\label{eq:Xcount}
	\chi^2 = {\sum_{i=1}^{n}   \frac{N_{i\_paquetes}-G_{i\_paquetes}}{G_{i\_paquetes}}}
\end{equation}

\begin{equation}\label{eq:Scount}
	\chi^2 = {\sum_{i=1}^{n}   \frac{N_{i\_bytes}-G_{i\_bytes}}{G_{i\_bytes}}}
\end{equation}\\


En donde \(N_{i}\) son los datos recolectados de los últimos 10 segundos por el
dispositivo \textit{distribution}. Luego, los resultados se comparan con los
valores de la tabla de distribución chi cuadrado usando \textbf{\(n\)} grados de
libertad, que representan la cantidad de estos dispositivos de distribución, y
un nivel de significancia (denotado como \(\alpha\)) de 0.05. (Un nivel de
significancia de 0.05 indica un riesgo del 5\% de concluir que existe tráfico
sospechoso cuando no lo hay). Si alguna de las comparaciones arroja que el valor
de la tabla es menor, se considera que en la red existe un potencial ataque de
denegación de servicio. El comportamiento del algoritmo se puede observar en el
diagrama de la Figura \ref{fig:diagram2}.

\begin{figure}[H]
	\centering 
	\includegraphics[width=\textwidth]{sequence_1}
	\caption{Diagrama de secuencia para la detección de anomalías.}
	\label{fig:diagram2}
\end{figure}

Una vez detectada la anomalía en la red se deben determinar los dispositivos de
borde comprometidos (\textbf{RF04}). Para ello primero es necesario encontrar
los dispositivos de distribución que superen un umbral en cuanto a la diferencia
entre el comportamiento observado y el esperado correspondiente. Esto se hace
mediante (\ref{eq:ineq_1}) y (\ref{eq:ineq_2}).

\begin{equation}\label{eq:ineq_1}
	{\frac{(N_{i\_paquetes}-G_{i\_paquetes})^2}{G_{i\_paquetes}}} > umbral_{paquetes}
\end{equation}

\begin{equation}\label{eq:ineq_2}
	{\frac{(N_{i\_bytes}-G_{i\_bytes})^2}{G_{i\_bytes}}} > umbral_{bytes}
\end{equation}\\

En donde el valor del umbral se toma igual a 3.5, de acuerdo a
\parencite{ddos_amir}, con el objetivo de tener un balance entre la cantidad de
falsos positivos y falsos negativos en la detección.

Determinados los dispositivos de distribución se deben buscar cuáles de sus
\(k\) enlaces entrantes desde la subcapa de acceso presentan un nivel de tráfico
mayor o igual al 80\% \parencite{estado_arte_2} en relación a todo el que
ingresa a dicho dispositivo, tal y como se muestra en (\ref{eq:ineq_3}) y
(\ref{eq:ineq_4}). En caso de que ningún enlace cumpla con alguna de estas
condiciones planteadas en las inecuaciones, se los considera a todos
sospechosos.

\begin{equation}\label{eq:ineq_3}
	{\frac{cant\_paquetes\_ultimos\_10\_segundos\_enlace_i}{\sum_{i=1}^{k}
      cant\_paquetes\_ultimos\_10\_segundos\_enlace_i}} \geq 0.8
\end{equation}

\begin{equation}\label{eq:ineq_4}
  {\frac{cant\_bytes\_ultimos\_10\_segundos\_enlace_i}{\sum_{i=1}^{k}
      cant\_bytes\_ultimos\_10\_segundos\_enlace_i}} \geq 0.8
\end{equation}\\

Luego, a partir de los mencionados enlaces se obtienen los dispositivos de borde
solicitados por \textbf{RF04}. Un diagrama para este requerimiento se puede
observar en la Figura \ref{fig:diagram2.1}.

\begin{figure}[H]
	\centering 
	\includegraphics[width=\textwidth]{sequence_2_1}
	\caption{Diagrama de secuencia para RF04.}
	\label{fig:diagram2.1}
\end{figure}

Un aspecto importante a aclarar es que los términos negativos no son tenidos en
cuenta en (\ref{eq:Xcount}), (\ref{eq:Scount}), (\ref{eq:ineq_1}) y
(\ref{eq:ineq_2}), ya que no se busca detectar anomalías ante niveles de
tráfico por debajo de lo esperado.

\begin{figure}[H]
	\centering 
	\includegraphics[width=\textwidth]{Clases2}
	\caption{Diagrama de clases para la detección de anomalías.}
	\label{fig:diagrama_clases_1_2}
\end{figure}


Por otra parte, para implementar \textbf{RF05} se usa el servicio de ONOS llamado
\verb|IntentService|, el cual permite conectar dos puntos en la red a alto
nivel. Cuando llega un paquete nuevo a la red, si las tablas OpenFlow (ver
Figura \ref{fig:openflow_2}) en los dispositivos no tienen reglas establecidas
para tomar acciones sobre el mismo, se envían al plano de control. El
controlador, mediante una aplicación de enrutamiento que usa el servicio antes
mencionado, se encarga de instalar las reglas OpenFlow en los dispositivos
intermedios estableciendo la comunicación entre los puntos.

La clase \verb|SinglePointToMultiPointIntent| permite conectar un punto
de conexión de entrada a dos o más puntos de salida. Llegado el momento de
realizar la duplicación de tráfico se obtienen todas las conexiones hechas por
el servicio \verb|IntentService| y se agrega como nuevo destino el IDS, a
aquellas que provengan del dispositivo de borde detectado como sospechoso,
tal y como se mencionó en párrafos anteriores. Así se logra el comportamiento
deseado de la Figura \ref{fig:dupl_trafico}.

En cuanto al aspecto del \textbf{RF05} que solicita el envío del tráfico al IDS
más cercano, previamente se debe poseer una lista con todos los detectores de la
red. Luego, ONOS provee el servicio \verb|topologyService| para obtener todos
los caminos entre dos puntos de conexión. Se usa este servicio para adquirir las
rutas más cortas entre los dispositivos de la subcapa de acceso devueltos por el
\textbf{RF04} y los IDSs más cercanos.

