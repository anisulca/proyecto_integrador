%%%%%%%%%%%%%%%%%%%%%%%%%%%%%%%%%%%%%%%%%
% Masters/Doctoral Thesis 
% LaTeX Template
% Version 2.5 (27/8/17)
%
% This template was downloaded from:
% http://www.LaTeXTemplates.com
%
% Version 2.x major modifications by:
% Vel (vel@latextemplates.com)
%
% This template is based on a template by:
% Steve Gunn (http://users.ecs.soton.ac.uk/srg/softwaretools/document/templates/)
% Sunil Patel (http://www.sunilpatel.co.uk/thesis-template/)
%
% Template license:
% CC BY-NC-SA 3.0 (http://creativecommons.org/licenses/by-nc-sa/3.0/)
%
%%%%%%%%%%%%%%%%%%%%%%%%%%%%%%%%%%%%%%%%%

%----------------------------------------------------------------------------------------
%	PACKAGES AND OTHER DOCUMENT CONFIGURATIONS
%----------------------------------------------------------------------------------------

\documentclass[
11pt, % The default document font size, options: 10pt, 11pt, 12pt
% oneside, Two side (alternating margins) for binding by default, uncomment to
% switch to one side
spanish, % ngerman for German
singlespacing, % Single line spacing, alternatives: onehalfspacing or doublespacing
% draft, % Uncomment to enable draft mode (no pictures, no links, overfull hboxes indicated)
% nolistspacing, If the document is onehalfspacing or doublespacing, uncomment
% this to set spacing in lists to single
% liststotoc, % Uncomment to add the list of figures/tables/etc to the table of contents
% toctotoc, % Uncomment to add the main table of contents to the table of contents
% parskip, % Uncomment to add space between paragraphs
% nohyperref, % Uncomment to not load the hyperref package
headsepline, % Uncomment to get a line under the header
% chapterinoneline,Uncomment to placeSDN the chapter title next to the number on
% one line consistentlayout, Uncomment to change the layout of the declaration,
% abstract and acknowledgements pages to \textit{match} the default layout
]{MastersDoctoralThesis} % The class file specifying the document structure

\usepackage[utf8]{inputenc} % Required for inputting international characters
\usepackage[T1]{fontenc} % Output font encoding for international characters
\usepackage{mathpazo} % Use the Palatino font by default

\usepackage[backend=bibtex,bibencoding=ascii,style=numeric,natbib=true,sorting=none]{biblatex}
% Use the bibtex backend with the authoryear citation style (which resembles APA)

\addbibresource{example.bib} % The filename of the bibliography

\usepackage[autostyle=true]{csquotes} % Required to generate language-dependent
                                      % quotes in the bibliography
\usepackage{float}
% \usepackage{tabu}
\usepackage{booktabs}
% \usepackage[table,xcdraw]{xcolor}
\usepackage{amssymb}
\usepackage{amsmath}
\usepackage{graphicx}
\usepackage{listings}
\usepackage{multirow}
\usepackage[table,xcdraw]{xcolor}

\usepackage{longtable}
  % declare the path(s) where your graphic files are
\graphicspath{{./Figures/}}
  % and their extensions so you won't have to specify these with
  % every instance of \includegraphics
  %\DeclareGraphicsExtensions{.pdf,.jpeg,.png,.eps, .svg}
  %\usepackage{svg}

%----------------------------------------------------------------------------------------
%	MARGIN SETTINGS
%----------------------------------------------------------------------------------------

\geometry{
	paper  = a4paper,     % Change to letterpaper for US letter
	outer  = 2.5cm,       % Inner margin
	inner  = 3.8cm,       % Outer margin
	top    = 1.5cm,       % Top margin
	bottom = 1.5cm,       % Bottom margin
	bindingoffset = .5cm, % Binding offset
	%showframe, % Uncomment to show how the type block is set on the page
}

%----------------------------------------------------------------------------------------
%	THESIS INFORMATION
%----------------------------------------------------------------------------------------

\thesistitle{Prototipo de sistema de diagnóstico de nistagmo mediante seguimiento de mirada} % Your thesis title, this is used in the title and abstract, print it
       % elsewhere with \ttitle
\supervisor{Ing. Nancy \textsc{Brambilla}} % Your supervisor's name, this is used in
                                       % the title page, print it elsewhere with
                                       % \supname
\examiner{Ing. Josefina \textsc{Meirovich},\\Lic. Nadia \textsc{Leizica},\\ Prof. Diego \textsc{Obregón} } % Your examiner's name, this is not

% currently used anywhere in the template, print it elsewhere with \examname
% \degree{Doctor of Philosophy} Your degree name, this is used in the title page
% and abstract, print it elsewhere with \degreename
\authorA{Daniela \textsc{Mamaní}}
\matriculaA{39930365}
\emailA{\texttt{daniela.mamani@mi.unc.edu.ar}}
\authorB{Anahí \textsc{Sulca}}
\matriculaB{38471174}
\emailB{\texttt{anahi.sulca@unc.edu.ar }}
% Your name, this is used in the title page and abstract, print it elsewhere
% with \authorname \addresses{} Your address, this is not currently used
% anywhere in the template, print it elsewhere with \addressname

% \subject{Biological Sciences} Your subject area, this is not currently used
% anywhere in the template, print it elsewhere with \subjectname
\keywords{Nystagmus} %palabras claves p bsuqueda
% Keywords for your thesis, this is not currently used anywhere in the template,
% print it elsewhere with \keywordnames
\university{Universidad Nacional de
  Córdoba} % Your university's name and URL, this is used in the title page and
% abstract, print it elsewhere with \univname \department{Facultad de Ciencias
% Exactas Físicas y Naturales} Your department's name and URL, this is used in
% the title page and abstract, print it elsewhere with \deptname
% \group{\href{http://researchgroup.university.com}{Research Group Name}} Your
% research group's name and URL, this is used in the title page, print it
% elsewhere with \groupname
\faculty{Facultad de Ciencias Exactas Físicas y Naturales}
% Your faculty's name and URL, this is used in the title page and abstract,
% print it elsewhere with \facname

\AtBeginDocument{
\hypersetup{pdftitle=\ttitle} % Set the PDF's title to your title
\hypersetup{pdfauthor=\authornameA} % Set the PDF's author to your name
\hypersetup{pdfkeywords=\keywordnames} % Set the PDF's keywords to your keywords
}

\begin{document}

\renewcommand\listtablename{\'Indice de tablas}

\frontmatter % Use roman page numbering style (i, ii, iii, iv...) for the pre-content pages

\pagestyle{plain} % Default to the plain heading style until the thesis style is
                  % called for the body content

%----------------------------------------------------------------------------------------
%	TITLE PAGE
%----------------------------------------------------------------------------------------

\begin{titlepage}
	\begin{center}
				
		\resizebox{1\textwidth}{!}{\includegraphics{Figures/logo_unc_fcefyn.png}}%
				
		\vspace*{.06\textheight}
		{\scshape\LARGE \univname\par}\vspace{0.5cm} % University name
		{\scshape\LARGE \facname\par}\vspace{1.5cm}
		\textsc{\Large Proyecto Integrador}\\[0.5cm] % Thesis type
		\textsc{\Large Ingeniería en Computación}\\[0.5cm]
				
		\HRule \\[0.4cm] % Horizontal line
		{\huge \bfseries \ttitle\par}\vspace{1.0cm} % Thesis title
		\HRule \\[1.5cm] % Horizontal line
				 
		\begin{minipage}[t]{0.6\textwidth}
			\begin{flushleft} \large
				\emph{Autores:}\\
				\authornameA \\ % Author name - remove the \href bracket to remove the link
				\matriculanameA \\
				\emailnameA \\
				\vspace{0.4cm} % Thesis title
        %\\~\\
				\authornameB \\ % Author name - remove the \href bracket to remove the link
				\matriculanameB \\
				\emailnameB
			\end{flushleft}
		\end{minipage}
		\begin{minipage}[t]{0.2\textwidth}
			\begin{flushright} \large
				\emph{Director:} \\
				\supname \\
        \vspace{0.2cm} %\\~\\
				\emph{Co-director:} \\
				\examname
			\end{flushright}
		\end{minipage}\\[2cm]
				 
		\vfill
				
		% \large \textit{A thesis submitted in fulfillment of the requirements\\ for
		% the degree of \degreename}\\[0.3cm] % University requirement text
		% \textit{in the}\\[0.4cm]
		% \groupname\\\deptname\\[2cm] % Research group name and department name
				 
		\vfill
				
		{\large {Octubre 2021}}\\[4cm] % Date
		%\includegraphics{Logo} % University/department logo - uncomment to place it
				 
		\vfill
	\end{center}
\end{titlepage}

%----------------------------------------------------------------------------------------
%	DEDICATION
%----------------------------------------------------------------------------------------

\dedicatory{Para nuestras familias y por todos los años de estudio\ldots} 

% %----------------------------------------------------------------------------------------
% %	DECLARATION PAGE
% %----------------------------------------------------------------------------------------

% \begin{declaration}
% \addchaptertocentry{\authorshipname} % Add the declaration to the table of contents
% \noindent I, \authorname, declare that this thesis titled, \enquote{\ttitle}
% and the work presented in it are my own. I confirm that:

% \begin{itemize} 
% \item This work was done wholly or mainly while in candidature for a research
%   degree at this University.
% \item Where any part of this thesis has previously been submitted for a degree
%   or any other qualification at this University or any other institution, this
%   has been clearly stated.
% \item Where I have consulted the published work of others, this is always clearly attributed.
% \item Where I have quoted from the work of others, the source is always given.
%   With the exception of such quotations, this thesis is entirely my own work.
% \item I have acknowledged all main sources of help.
% \item Where the thesis is based on work done by myself jointly with others,
%   I have made clear exactly what was done by others and what I have contributed myself.\\
% \end{itemize}
 
% \noindent Signed:\\
% \rule[0.5em]{25em}{0.5pt} % This prints a line for the signature
 
% \noindent Date:\\
% \rule[0.5em]{25em}{0.5pt} % This prints a line to write the date
% \end{declaration}

% \cleardoublepage

%----------------------------------------------------------------------------------------
%	QUOTATION PAGE
%----------------------------------------------------------------------------------------

% \vspace*{0.2\textheight}

% \noindent\enquote{\itshape Thanks to my solid academic training, today I can
% write hundreds of words on virtually any topic without possessing a shred of
% information, which is how I got a good job in journalism.}\bigbreak

% \hfill Dave Barry

%----------------------------------------------------------------------------------------
%	ABSTRACT PAGE
%----------------------------------------------------------------------------------------

\begin{abstract}
	%\addchaptertocentry{\abstractname} % Add the abstract to the table of contents
		
	Las \textbf{redes definidas por software (SDN)} consisten en un reciente
  paradigma y candidato de la nueva generación de la arquitectura de Internet.
  Propuesto para solucionar la complicada administración de los dispositivos que
  conforman las distintas redes, tales como conmutadores y enrutadores, busca ir
  de la mano con la computación en la nube en cuanto al ofrecimiento de
  servicios de tecnología de la información para empresas.

	SDN brinda ventajas respecto a las redes tradicionales que pueden aprovecharse
  para solucionar problemas de las mismas, como por ejemplo, los \textbf{ataques
    de denegación de servicio (DoS)} y sus versiones distribuidas
  (\textbf{DDoS}), las cuales afectan a la disponibilidad de la mencionada
  computación en la nube o, por ejemplo, a la de los clientes de un
  \textbf{proveedor de servicios de Internet (ISP)}. Este último representa un
  caso de uso muy aplicable en Argentina.

	Actualmente, las soluciones brindadas en materia de seguridad utilizan
  componentes de hardware y de software que no se ajustan, o por lo menos en la
  forma en la cual se utilizan, a la naturaleza dinámica de las redes SDN. En el
  presente proyecto se logra generar una solución escalable, práctica y con una
  visión global de la topología de red que permite detectar y mitigar ataques
  DoS y/o DDoS efectuados a cualquier \textit{host} cliente de un ISP.

	Para lograr lo anterior, en primer lugar, se buscó adquirir conocimientos en
  materia de redes definidas por software, ataques de denegación de servicio,
  proveedores de servicios de Internet, sistemas de detección de intrusos, entre
  otros conceptos. Luego, se seleccionaron justificadamente las herramientas
  adecuadas para lograr poner a prueba los conocimientos adquiridos mencionados
  anteriormente. En tercer lugar, se diseñaron e implementaron 3 aplicaciones
  con el fin de aprovechar la acción conjunta de todos los componentes de un
  sistema SDN para mitigar estos ataques. Luego, se debió validar y verificar el
  correcto funcionamiento de estas aplicaciones, por lo que se generó un entorno
  de emulación virtual con una topología característica de un ISP. Por último,
  se enunciaron las limitaciones del proyecto integrador y los posibles trabajos
  futuros que pueden desencadenarse a partir del mismo.

\end{abstract}

% \textbf{Palabras Clave:} \textit{\keywordnames}

%----------------------------------------------------------------------------------------
%	ACKNOWLEDGEMENTS
%----------------------------------------------------------------------------------------
 
\begin{acknowledgements}
	
	%\addchaptertocentry{\acknowledgementname} % Add the acknowledgements to the
                                            % table of contents
	Muchas gracias a nuestras familias por el apoyo y la ayuda incondicional que
  nos han brindado en todos estos años de estudio. \bigskip
			
	
	Muchas gracias a nuestros directores, Hugo Carrer y Natasha Tomattis por la
  confianza, la paciencia, la supervisión y la ayuda que nos han otorgado
  durante todo el desarrollo del proyecto. \bigskip
			
	
	Muchas gracias a nuestros amigos y a todas las personas que, durante estos
  cinco años de carrera, hemos tenido el agrado de conocer. \bigskip
		
	
	Muchas gracias a Fundación Fulgor y a Fundación Tarpuy, y a todo su personal,
  por los momentos, enseñanzas y oportunidades que nos han brindado. \bigskip
			
	
	Muchas gracias a la Facultad de Ciencias Exactas, Físicas y Naturales de la
  Universidad Nacional de Córdoba por darnos la oportunidad de realizar esta
  hermosa carrera. \vspace*{\fill}
		
\end{acknowledgements}

%----------------------------------------------------------------------------------------
%	LIST OF CONTENTS/FIGURES/TABLES PAGES
%----------------------------------------------------------------------------------------

\hypersetup{
	linkcolor=black,
	citecolor=black,
	urlcolor=black
	}

\tableofcontents % Prints the main table of contents

\listoffigures % Prints the list of figures

\listoftables % Prints the list of tables

\lstlistoflistings

%----------------------------------------------------------------------------------------
%	ABBREVIATIONS
%----------------------------------------------------------------------------------------

\begin{abbreviations}{ll} % Include a list of abbreviations (a table of two columns)
	\textbf{API} & \textbf{A}pplication \textbf{P}rogramming \textbf{I}nterface\\
	\textbf{CLI} & \textbf{C}ommand-\textbf{l}ine \textbf{i}nterface\\
	\textbf{CDPI} & \textbf{C}ontrol-\textbf{D}ata-\textbf{P}lane \textbf{I}nterface\\
	\textbf{DDoS} & \textbf{D}istributed \textbf{D}enial \textbf{o}f \textbf{S}ervice\\
	\textbf{DNS} & \textbf{D}omain \textbf{N}ame \textbf{S}ystem\\
	\textbf{DoS} & \textbf{D}enial \textbf{o}f \textbf{S}ervice\\
	\textbf{DRDoS} & \textbf{D}istributed \textbf{R}eflection \textbf{D}enial \textbf{o}f \textbf{S}ervice\\
	\textbf{FIB} & \textbf{F}orwarding \textbf{I}nformation \textbf{B}ase\\
	\textbf{GUI} & \textbf{G}raphical \textbf{U}ser \textbf{I}nterface\\
	\textbf{HTTP} & \textbf{H}yper\textbf{t}ext \textbf{T}ransfer \textbf{P}rotocol\\
	\textbf{ICMP} & \textbf{I}nternet \textbf{C}ontrol \textbf{M}essage \textbf{P}rotocol\\
	\textbf{IDS} & \textbf{I}ntrusion \textbf{D}etection \textbf{S}ystem\\
	\textbf{IoT} & \textbf{I}nternet \textbf{o}f \textbf{T}hings\\
	\textbf{IP} & \textbf{I}nternet \textbf{P}rotocol\\
	\textbf{ISP} & \textbf{I}nternet \textbf{S}ervice \textbf{P}rovider\\
	\textbf{IT} & \textbf{I}nformation \textbf{T}echnology \\
	\textbf{LAN} & \textbf{L}ocal \textbf{A}rea \textbf{N}etwork\\
	\textbf{NIDS} & \textbf{N}etwork \textbf{I}ntrusion \textbf{D}etection \textbf{S}ystem\\
	\textbf{ODL} & \textbf{O}pen\textbf{D}ay\textbf{l}ight\\
	\textbf{ONF} & \textbf{O}pen \textbf{N}etworking \textbf{F}oundation\\
	\textbf{ONOS} & \textbf{O}pen \textbf{N}etworking \textbf{O}perating \textbf{S}ystem\\
	\textbf{OVS} & \textbf{O}pen \textbf{v}\textbf{S}witch\\
	\textbf{POM} & \textbf{P}roject \textbf{O}bject \textbf{M}odel\\
	\textbf{RIB} & \textbf{R}outing \textbf{I}nformation \textbf{B}ase\\
	\textbf{SDN} & \textbf{S}oftware \textbf{D}efined \textbf{N}etworking\\
	\textbf{SVG} & \textbf{S}calable \textbf{V}ector \textbf{G}raphics\\
	\textbf{TCP} & \textbf{T}ransmission \textbf{C}ontrol \textbf{P}rotocol\\
	\textbf{TLS} & \textbf{T}ransport \textbf{L}ayer \textbf{S}ecurity \\
	\textbf{UDP} & \textbf{U}ser \textbf{D}atagram \textbf{P}rotocol\\
	\textbf{UML} & \textbf{U}nified \textbf{M}odeling \textbf{L}anguage\\
	\textbf{URL} & \textbf{U}niform \textbf{R}esource \textbf{L}ocator\\
\end{abbreviations}

%----------------------------------------------------------------------------------------
%	THESIS CONTENT - CHAPTERS
%----------------------------------------------------------------------------------------

\mainmatter % Begin numeric (1,2,3...) page numbering

\pagestyle{thesis} % Return the page headers back to the "thesis" style

% Include the chapters of the thesis as separate files from the Chapters folder
% Uncomment the lines as you write the chapters

% Chapter Template
% cSpell:words parencite onfwhitepaper includegraphics resizebox sdncomponents
% linewidth comparqui redireccionar enrutamiento subfigure toposdn Nicira toposinsdn
\chapter{Introducción} % Main chapter title 

\label{Chapter1} % Change X to a consecutive number; for referencing this chapter elsewhere, use \ref{ChapterX}

Existe una tendencia a que los servicios informáticos provistos sean cada vez más elásticos, bajo demanda, accesibles y económicos, con el fin de que se adapten de manera más rápida a los cambios en el mercado y a los tiempos y recursos de los clientes y usuarios. Por ello, surgen nuevos conceptos en cuanto al ofrecimiento de servicios IT empresariales para cumplir con los principios anteriores. Las redes definidas por software constituyen un claro ejemplo. 

\textbf{\textit{Software Defined Networking} (SDN)} consiste en un reciente paradigma y candidato de la nueva generación de la arquitectura de Internet. Es por ello que compañías como Google ya lo han adoptado en sus centros de datos internos. Propuesto para solucionar la complicada administración de los dispositivos que conforman las distintas redes, tales como conmutadores y enrutadores, busca ir de la mano con la computación en la nube en cuanto al ofrecimiento de los mencionados servicios de tecnología de la información para las distintas empresas \parencite{provicion_it}.

SDN brinda ventajas respecto a las redes tradicionales que pueden aprovecharse para solucionar problemas de las mismas, como por ejemplo, los \textbf{ataques de denegación de servicio (DoS)} y sus versiones distribuidas (\textbf{DDoS}), los cuales afectan a la disponibilidad de la mencionada computación en la nube o, por ejemplo, a la de los clientes de un \textbf{proveedor de servicios de Internet (ISP)}. Este último representa un caso de uso muy aplicable en Argentina. A su vez, los mencionados ataques producen que usuarios legítimos no puedan
utilizar correctamente los servicios, por la saturación del ancho de banda de
los enlaces y/o por la sobrecarga de los recursos del sistema y/o de la
red \parencite{autoscaling}.


El treceavo reporte anual \textit{Worldwide Infrastructure Security Report}, de
la proveedora de seguridad \textit{Arbor Networks}, menciona que el 87 por
ciento de los servicios provistos han experimentado amenazas y ataques DDoS
\parencite{sec_report}. Las soluciones para la defensa hacia estos ataques DDoS
existentes utilizan ciertos componentes de hardware y de software en la red para
detectarlos o mitigarlos. Sin embargo, en el nuevo paradigma de SDN, a
causa de la dinámica natural de las redes, dichas soluciones no se podrán
mantener a largo plazo. El desafío de este proyecto está centrado en explotar
las características de las redes definidas por software para la defensa en
cuanto a los ataques descritos anteriormente. 

A lo largo del presente trabajo, se podrá observar cómo desde un ISP con una red SDN se logra generar una solución escalable, práctica y con una visión global de la topología que permite detectar y mitigar ataques DoS y/o DDoS efectuados a cualquier \textit{host} cliente de dicho ISP. Una de las ventajas que se obtiene a partir de la solución mostrada consiste en el aprovechamiento de la acción conjunta de todos los componentes de un sistema SDN para detener los mencionados ataques.

Por último, en el transcurso de esta sección se introducirán aspectos generales y significativos del proyecto, como por ejemplo, la motivación, los objetivos, la organización del texto y el estado actual de las soluciones y tecnologías directamente relacionadas.


%-----------------------------bitext comileArquitectura del protocolo--------------------
%	SECTION 1
%----------------------------------------------------------------------------------------
\section{Motivación} 

Las motivaciones del proyecto fueron las siguientes:

\begin {itemize}
\item El estudio acabado de la problemática de los ataques DoS y DDoS, que
  representan una importante amenaza a la alta disponibilidad en las redes actuales.
\item La aplicación de las ventajas que proporcionan las redes definidas por software en
  el área de detección y mitigación de los ataques DoS y DDoS.
\item La aplicación de los conocimientos adquiridos durante la cursada de la carrera de
  Ingeniería en Computación.
\item La posibilidad de realizar futuras publicaciones científicas en base al
  proyecto desarrollado.
\item La posibilidad de aplicación de lo efectuado en proveedores de servicios
  de Internet en algún futuro cercano.
\end{itemize}


\section {Objetivos}
El objetivo general de este proyecto es aprovechar las ventajas de las redes
definidas por software (SDN) en el campo de la ciberseguridad, más concretamente
en la detección y mitigación de ataques de denegación de servicio. Para esto
será necesario introducirse en el campo de la seguridad en las redes, analizando
con detalle los diferentes tipos de ataques de denegación de servicio y cómo
afectan éstos a las aplicaciones de negocio y a las redes. Para extender estos
conceptos a arquitecturas SDN, será indispensable estudiar los principios de éstas y los protocolos que involucran, haciendo foco en el protocolo Openflow
\parencite{opf151}.

Entrando más en detalle, los objetivos particulares son los siguientes:
\begin{itemize}
\item Adquirir un amplio conocimiento en el ámbito de SDN y las tecnologías que
  integran dicho paradigma.
\item Adquirir un amplio conocimiento en el ámbito de los ataques DoS y DDoS,
  incluyendo comportamientos, herramientas de generación de los mismos y
  herramientas de recolección de las métricas.
\item Generar un ambiente de emulación que admita la creación de distintas redes SDN y la generación de ataques fácilmente reconfigurables y escalables.
\item Recolectar las métricas necesarias para medir la efectividad de los
  ataques, así como también la mitigación de los mismos.
\item Desarrollar aplicaciones en un controlador SDN que permitan la detección y
  mitigación de determinados ataques.
\item Diseñar un esquema de red SDN con varios detectores IDS (\textit{Intrusion
  Detection System}) como vehículo de prueba de las aplicaciones desarrolladas en
  el controlador.
\end{itemize}


\section {Estructura del texto}
El presente trabajo se encuentra constituido por los siguientes capítulos o secciones:
\begin{enumerate}
\item \textbf {Introducción:} en esta sección se presentan los aspectos más generales y significativos del proyecto. Se incluyen las motivaciones, los objetivos del mismo, la organización de la estructura del texto y, por último, un resumen del estado actual de las tecnologías directamente relacionadas con este trabajo.

\item \textbf {Marco teórico:} para poder entender el presente proyecto es
  necesario adquirir conocimientos sobre determinados conceptos que se explican en
  este capítulo.
\item \textbf {Análisis de las tecnologías y planificación:} los conceptos
  desarrollados en la sección anterior deben implementarse mediante determinadas
  tecnologías y herramientas, las cuales se explican y analizan en este
  capítulo. Además, se detallan aspectos importantes de la Ingeniería de
  Software aplicada a la gestión y administración del proyecto.
\item \textbf {Aplicación de detección de anomalías:} en esta sección se describe
  el diseño y la implementación de la aplicación encargada de detectar tráfico
  de red sospechoso en base a un comportamiento estadístico esperado de dicha
  red.
\item \textbf {Aplicación de filtrado del tráfico:} en este capítulo se detalla el diseño y la implementación de la aplicación cuya función es la de mitigar los ataques que sufre la mencionada red.
\item \textbf {Interfaz gráfica de usuario:} en esta sección se explica el
  diseño y la implementación tanto de la API como de la aplicación, que
  posibilitan al administrador del sistema una configuración y un uso sencillos.
\item \textbf {Diseño e implementación del entorno de emulación:} para que las aplicaciones diseñadas puedan desenvolverse y evaluarse, bajo este título se detalla el entorno de trabajo utilizado.
\item \textbf {Resultados:} en este capítulo se evalúan los resultados obtenidos
  y se verifica el cumplimiento de los requerimientos definidos a lo largo de este
  proyecto. Además, se describen los distintos casos de \textit{tests} llevados a cabo.
\item \textbf {Conclusiones:} las conclusiones del proyecto integrador, sus limitaciones
  y los posibles trabajos futuros que pueden desencadenarse a partir del mismo se
  describen en esta sección.
\item \textbf {Apéndices:} en estos capítulos se le presenta al lector un claro
  tutorial de cómo desplegar y utilizar el entorno de trabajo y las aplicaciones
  desarrolladas con el fin de cumplir con los objetivos del proyecto. Además se
  amplían algunos conceptos vistos durante el transcurso del presente trabajo.
  
\end{enumerate}


\section{Estado del arte} \label{sec:state_art}
Se presentan a continuación algunas de las soluciones e implementaciones
existentes más importantes, relacionadas con nuestro proyecto, con el fin de
efectuar un marco de comparación adecuado.

\subsection*{Evaluación de técnicas de detección de ataques DDoS en una
arquitectura SDN con controlador Floodlight}
En \parencite{estado_arte_1} se utilizan las características de la arquitectura
SDN y del controlador Floodlight para detectar y mitigar ataques DDoS por
inundación que utilizan el \textit{flag} SYN de TCP. Se estudian  algoritmos de
detección de anomalías, buscando ventajas y desventajas de cada uno. Estos
algoritmos son la matriz de correlación, la prueba de bondad de ajuste con
Chi-Cuadrado y la entropía de Shannon. 

Este proyecto \parencite{estado_arte_1} propone soluciones que presentan una
escalabilidad limitada debido a que la tarea de recolección de información del
tráfico de la red, el procesamiento de paquetes del plano de datos en busca de
ciertos patrones y el almacenamiento de los estados de las conexiones entre los
clientes y el servidor las efectúa el controlador SDN. El diseño de esta
arquitectura ejerce una gran presión sobre los recursos del controlador y
provoca un claro incremento en la latencia de la red. Por otro lado, la
topología con la que se prueba los algoritmos es reducida y las reglas de
eliminación de tráfico malicioso se aplican a todo el tráfico proveniente de un
\textit{host} atacante. Esto perjudica a los procesos de un \textit{bot} que no
participan del ataque y que podrían estar siendo usados por un usuario legítimo.


\subsection*{Detección de ataques DDoS en arquitecturas basadas en SDN}
En \parencite{estado_arte_2} se propone detectar ataques de denegación de servicio en las redes SDN con arquitecturas especiales, utilizando el algoritmo estadístico basado en la entropía de Shannon y la recolección de direcciones IP de destino de los paquetes. Si bien este método resulta eficiente, puede presentar una gran cantidad de falsos positivos, ya que la detección solamente se basa en la cantidad de paquetes dirigidos hacia una misma dirección destino.

\subsection*{Un sistema de detección de ataques DoS/DDoS usando el enfoque estadístico de Chi-Cuadrado}
En \parencite{estado_arte_3} se propone un sistema distribuido de detección de ataques de denegación de servicios para redes tradicionales que utiliza también un enfoque estadístico. Este enfoque es conocido como la distribución de probabilidad de Chi-Cuadrado, la cual será utilizada en este proyecto y se describe en la Sección \ref{sec:Chapter4}.

\subsection*{Mitigación de ataques DoS utilizando técnicas basadas en anomalías y reglas en redes definidas por software}
En \parencite{estado_arte_4}, se propone detectar y mitigar ataques de denegación de servicio en las redes definidas por software a partir de dos tipos de sistemas de detección de intrusiones (IDS), realizando comparaciones entre ellos como conclusión del trabajo. No se realiza ningún análisis estadístico y la topología de red es simple. A su vez, todo el tráfico en el plano de datos es analizado por estos IDSs, lo cual limita la escalabilidad de esta solución.


\subsection*{Sistema de detección de intrusiones basado en Machine Learning para redes definidas por software.}
En \parencite{estado_arte_5}, bajo una red definida por software se busca
detectar y mitigar ataques con la ayuda de un sistema de detección de
intrusiones basado en firmas y una red neuronal que utiliza el algoritmo
\textit{backpropagation} para detectar anomalías en el tráfico y ataques no
reconocidos por el IDS. Si bien esta solución es novedosa, posee los
inconvenientes de que todo el tráfico del plano de datos es observado por dicho
IDS y que la capacidad de cómputo depende de la cantidad de \textit{hosts}
presentes en la red SDN, por lo que la escalabilidad es difícil y/o limitada.

% Chapter Template
% cSpell:words parencite onfwhitepaper includegraphics resizebox sdncomponents
% linewidth comparqui redireccionar enrutamiento subfigure toposdn Nicira toposinsdn
\chapter{Marco teórico} % Main chapter title 

\label{Chapter2} % for referencing this chapter elsewhere, use \ref{ChapterX}
En el presente capítulo, en primer lugar, se explicará un concepto fundamental
sobre el que se basó el proyecto: las redes definidas por software. Estas
representan una transformación, tanto técnica como económica, de las redes tal y
como las conocemos hoy en día. Las ventajas son tales que empresas como Google,
Facebook, China Unicom, AT\&T, Deutsche Telekom, entre otras, intentan migrar
sus redes hacia este nuevo paradigma.

En segundo lugar, los ataques de denegación de servicio serán el objetivo a
describir, ya que representan una amenaza a solucionar tanto en las redes
actuales como en las futuras. Aprovechando las ventajas de la administración de
las redes definidas por software, este proyecto intenta abordar esta
problemática desde un nuevo enfoque.

Luego, se expone una sección con el propósito de destacar las características de
los sistemas de detección de intrusiones, elementos fundamentales para la
detección y mitigación de los ataques de denegación de servicio. Estos forman
parte de la solución brindada a lo largo de este trabajo.

Por último, la principal contribución del presente proyecto es la implementación de
un mecanismo de detección DDoS combinando el uso de los mencionados sistemas de
detección de intrusiones con las redes definidas por software. La
infraestructura apropiada para poder validar correctamente el producto y darle
una mayor relevancia debe ser aquella que maneje una enorme cantidad de
tráfico, presente múltiples subredes y puntos de acceso, posea posibilidades de
escalar en un futuro y mantenga una jerarquía entre los dispositivos de red que
la componen. Un buen escenario, considerando también las características de
algunas de las empresas mencionadas anteriormente, son los proveedores de
servicios de Internet, por lo que se destina una sección al final de este
capítulo en donde se explican su arquitectura y sus componentes.


%---------bitext comileArquitectura del protocolo----------------------------------------
%	SECTION 1
%----------------------------------------------------------------------------------------
\section{Redes definidas por software} \label{sec:sdn}

El mundo SDN entra en el concepto de la \textbf{desagregación}, que consiste
simplemente en el desacoplamiento de dos componentes normalmente integrados, con
el objetivo de crear opciones y flexibilidad. Si los usuarios pueden mezclar y
combinar diferentes tipos y versiones de dichos componentes, entonces pueden
construir un sistema que sea un ajuste exclusivo para su propósito específico.

\subsection{Definición} 

La \textit{Open Networking Foundation} (ONF) \parencite{onf} es una sociedad sin
fines de lucro que impulsa la transformación de la infraestructura de la red y
de los modelos de negocios, dedicándose al desarrollo, estandarización y
comercialización de \textit{Software Defined Networking} (SDN). Ellos nos
brindan la primera definición, en donde SDN es una arquitectura dinámica y
adaptable, muy útil para satisfacer las necesidades variables en ancho de banda
de las aplicaciones actuales por medio de la desagregación de las funciones de
control de las de reenvío de la red. Por un lado, el control de dicha red puede
volverse programable y centralizado y, por el otro, las funciones de reenvío se
efectúan a través de la infraestructura subyacente \parencite{def_citrix}.

El hecho es que las redes de cualquier tamaño necesitan ser controladas. En una
organización pequeña, con tener una sola persona talentosa que pueda diseñar,
construir, administrar y mantener los servidores, \textit{firewalls} y
enrutadores es suficiente. Sin embargo, esto no es sostenible ni aceptable en
redes más grandes. Y es aquí donde SDN cumple un papel muy importante.

\subsection{Arquitectura SDN} \label{sec:arch_SDN}
Antes de entrar en detalles sobre la arquitectura SDN, es necesario especificar
el papel del plano de control y del plano de datos, tanto para las redes
tradicionales como para las que se encuentran dentro del campo SDN.

Si nos situamos en un dispositivo de red, a través de los enlaces (cable, fibra
o medios inalámbricos) éste recibe datagramas que debe procesar y reenviar. El
proceso de reenvío de paquetes mueve los datagramas desde una interfaz de
entrada hacia una de salida apropiada según la información contenida en un
estructura de datos conocida como \textit{Forwarding Information Base} (FIB) que
posee el dispositivo de red en cuestión. Dicha función de reenvío forma parte
del \textbf {plano de datos} \parencite{arch_tradicional}.

Como se mencionó anteriormente, en los dispositivos de red se necesita
información para realizar el proceso de reenvío en forma adecuada. Para obtener
dicha información (por ejemplo, entradas de la tabla FIB), se construye una
vista de la topología de la red y, en base a eso, se arman las tablas de
enrutamiento conocidas como \textit {Routing Information Base} (RIB), a partir
de las cuales se calculan las mejores rutas de acuerdo a determinados criterios
de selección. Lo mencionado es tarea del \textbf {plano de control}. Además
comprende funciones de configuración (configuración de interfaces, configuración
de protocolos, etc.) y el monitoreo del sistema \parencite{arch_tradicional}.

Entrando más en detalle, una tabla RIB presenta entradas que corresponden a
todos los posibles caminos aprendidos por los protocolos administradores de la
topología de la red desde un origen hacia un destino particular. De acuerdo a
estas entradas, una tabla FIB del plano de datos se confecciona a partir de la
elección de uno de todos los caminos posibles, en base a los mencionados
criterios.

Si ahora tenemos en cuenta a las redes tradicionales, sus dispositivos de red
incluyen a ambos planos, el de control y el de datos, e implementan funciones de
los mismos por medio de aplicaciones que son propietarias del fabricante del
equipo. De esta forma, se está en presencia de una de las desventajas
principales de dichas redes: la configuración distribuida y dependiente del mencionado
fabricante del equipo. Lo primero se debe a que como el plano de control se
encuentra en todos los dispositivos de red, los cambios en las configuraciones
deben realizarse equipo por equipo.

En este contexto surgen las redes definidas por software, caracterizadas
principalmente por:

\begin{itemize}
\item \textbf{Separación física del plano de datos del plano de control.} Esto
  garantiza que la infraestructura física de conmutación y de reenvío de paquetes
  se pueda construir a partir de equipos estándar, de menor costo y sin la
  configuración dependiente del fabricante. Estos dispositivos son conocidos como
  conmutadores de caja blanca \parencite{sdn_approach}.
\item \textbf{Gestión central.} El plano de control pasa a localizarse en un
  único ente centralizado programable, conocido como controlador SDN. A través de
  esta gestión central, todos los dispositivos de la red se controlan desde un
  único punto y, de esta forma, se logra acabar con la desventaja de las redes
  tradicionales acerca de la configuración distribuida. En una arquitectura SDN,
  desde el controlador se mantiene una visión global de la topología y se
  simplifica la comunicación entre las aplicaciones, servicios y dispositivos de
  red \parencite{sdn_oreilly}. Esto permite a los administradores monitorear,
  configurar, asegurar y optimizar los recursos de dicha red de acuerdo a
  necesidades específicas. Cabe destacar además que esta gestión central
  programable favorece la consistencia en las políticas de la red y el
  mantenimiento de las aplicaciones.
\end{itemize}

A su vez, entre los dispositivos SDN y el mencionado controlador, existe una interfaz de
programación que separa ambos planos y recibe el nombre de
\textit{Control-Data-Plane Interface (CDPI)}.

Entrando más en detalle, la ONF define una arquitectura de alto nivel para SDN,
con tres capas principales como se ve en la Figura \ref{fig:layers}:

\begin{itemize}
\item Capa de datos.
\item Capa de control.
\item Capa de aplicación.
\end{itemize}

\begin{figure}[H]
	\centering 
	% \resizebox{.85\textwidth}{!}{\includegraphics{Figures/sdn-arch.png}}%
  \includegraphics[scale=0.6]{sdn_layers}
	\caption[Arquitectura SDN.]{Arquitectura SDN \parencite{onf}.}
	\label{fig:layers}
\end{figure}


\subsubsection*{Capa de datos}
La capa de datos forma parte del plano de datos e incluye a dispositivos de red
conocidos como dispositivos SDN. Algunos de estos dispositivos consisten en
conmutadores SDN, virtuales o físicos. A diferencia de aquellos utilizados en
las redes tradicionales, presentan una mayor programabilidad en su
funcionamiento, lo que deriva en un hardware más genérico.

Los conmutadores SDN poseen en su estructura interna una interfaz de
comunicación con el controlador, el cual administra las tablas de flujo de
dicho dispositivo, explicadas posteriormente en la Sección \ref{sec:opflow},
y a partir de las cuales se manejan los flujos de datos y las funciones de
procesamiento de los paquetes.

Los conmutadores SDN virtuales son programas de software que simulan el
comportamiento y la estructura interna de un \textit{switch} SDN. Existen distintos
productos en el mercado, como por ejemplo, Open vSwitch (OVS), Indigo, entre
otros. Su principal diferencia con los dispositivos físicos es que el
procesamiento de paquetes se realiza vía software \parencite{sdn_approach}.

Por otra parte, se encuentran los conmutadores SDN físicos, que permiten
trabajar a mayor velocidad pero su costo es mayor. Huawei, Cisco, HP, IBM,
entre otros, son empresas que proveen estos dispositivos, los cuales poseen
la implementación correspondiente en hardware para el procesamiento de paquetes.


\subsubsection*{Capa de control} \label{sec:control_layer}

La capa de control forma parte del plano de control y se sitúa físicamente en el
controlador SDN, proporcionando servicios de red y actuando como organizadora y
mediadora. Como se observa en la Figura \ref{fig:layers}, la capa de control
está conectada a dos interfaces.

La interfaz \textit{SOUTHBOUND} brinda comunicación con los dispositivos de red
de la capa de datos. El controlador proporciona primitivas a los dispositivos,
generalmente para que éstos informen el estado de la red e importen reglas de
reenvío sobre cómo tratar los paquetes entrantes \parencite{sdn_approach}.

La interfaz \textit{NORTHBOUND} facilita la interacción con las aplicaciones que
se encuentran en la capa superior. El controlador expone funciones y operaciones
de la red a través de interfaces de programación de aplicaciones (APIs)
bidireccionales \parencite{sdn_oreilly}.

En caso de que exista un gran dominio de red administrativa, existirán varios
controladores y se conectarán a través de una interfaz este-oeste. Esto es
necesario ya que los mismos tienen que compartir información de la red y
coordinar sus procesos de toma de decisiones.

\subsubsection*{Openflow}\label{sec:opflow}

En primer lugar, el controlador no puede hacer nada sin un medio que permita dar
instrucciones a los dispositivos de red. Justamente, para desagregar las
operaciones de reenvío de paquetes de las de control de manera adecuada, se debe
mantener la comunicación entre ellas. Así surge una capa de abstracción que
permite a la capa de datos mantener el procesamiento de paquetes y la
comunicación con la capa de control centralizada. Es aquí en donde OpenFlow
puede desempeñarse, permitiendo la separación deseada \parencite{sdn_approach}.

OpenFlow es un conjunto de protocolos y una API. Contiene el protocolo de
comunicaciones entre el plano de datos y el plano de control SDN, posee parte
del comportamiento del plano de datos y forma parte de las interfaces
\textit{SOUTHBOUND}. Además, OpenFlow es no propietario, permite programar el
plano de datos de los conmutadores SDN y su funcionamiento está basado en tres
componentes fundamentales como se observa en la Figura \ref{fig:openflow_1}:

\begin{itemize}
\item Controlador SDN.
\item Dispositivos de red SDN con capacidad OpenFlow.
\item Tablas de flujo de los conmutadores SDN con capacidad OpenFlow.
\end{itemize}


\begin{figure}[H]
	\centering 
	% \resizebox{.85\textwidth}{!}{\includegraphics{Figures/sdn-arch.png}}%
	\includegraphics[scale=1.0]{openflow}
	\caption[Componentes fundamentales de una arquitectura SDN con OpenFlow]{Componentes fundamentales de una arquitectura SDN con OpenFlow
    \parencite{sdn_approach}.}
	\label{fig:openflow_1}
\end{figure}


Teniendo en cuenta esos tres componentes, OpenFlow se caracteriza por permitir,
entre otras cosas, lo siguiente
\parencite{sdn_oreilly}\parencite{sdn_simplified}:

\begin{itemize}
  \item Establecer una sesión de control entre el controlador SDN y el
    conmutador SDN.
  \item Definir una estructura de mensaje para intercambiar modificaciones
    de flujo y recopilar estadísticas.
  \item Definir una estructura de mensajes \textit{PACKET\_IN} y
    \textit{PACKET\_OUT}. El primero se utiliza para que los conmutadores SDN
    le envíen información de los paquetes al controlador para que éste actúe
    como manejador de excepciones. Por otra parte, el segundo se utiliza para
    que el mencionado controlador le envíe a los \textit{switches} SDN paquetes
    que se necesitan enviar por el plano de datos \parencite{sdn_oreilly}.
  \item Definir la estructura fundamental de un conmutador SDN (puertos y
    tablas). 
  \item La configuración y gestión en la asignación de puertos de un
    conmutador físico a un controlador particular.
\end{itemize}

Para lograr esos comportamientos, el dispositivo de red SDN con capacidad
OpenFlow cuenta con un enlace por medio del cual se conecta al controlador. Este
último es quien le configura el comportamiento a partir del intercambio de
mensajes que se envían utilizando TLS sobre TCP \parencite{sdn_oreilly} 
\parencite{sdn_approach}. Obviamente, el que inicia la comunicación es el
dispositivo de red SDN, que conoce la dirección IP del
controlador.

Como se dijo anteriormente, por medio de OpenFlow el controlador puede modificar
el comportamiento de un conmutador SDN. Esto se logra alterando las tablas de
flujo de dicho conmutador, las cuales poseen los siguientes campos principales
en sus entradas \parencite{opf151}, tal y como se observa en la Figura
\ref{fig:openflow_2}:

\begin{itemize}
  \item Cabecera del paquete que define el flujo de paquetes. (\textit{Match}).
  \item Acción o instrucción de cómo se deben procesar los paquetes de un
    determinado flujo. (Descartar flujo de paquetes, modificar campos del flujo
    de paquetes, encapsular y enviar al controlador el flujo de paquetes o
    enviar por un puerto o puertos específicos los paquetes del correspondiente
    flujo).
  \item Estadísticas sobre la cantidad de paquetes y de bytes de un flujo.
    Estadísticas sobre el tiempo de inactividad de dicho flujo.
\end{itemize}

\begin{figure}[H]
	\centering 
	% \resizebox{.85\textwidth}{!}{\includegraphics{Figures/sdn-arch.png}}%
	\includegraphics[scale=0.7]{newnorm}
	\caption[Tablas de flujo de un dispositivo OpenFlow]{Tablas de flujo de un dispositivo OpenFlow
    \parencite{new_norm_for_networks}.}
	\label{fig:openflow_2}
\end{figure}

Finalmente, el conmutador hará la acción correspondiente con los paquetes que le
ingresen de la red, de acuerdo a las coincidencias de éstos con las entradas de
la tabla de flujo, teniendo en cuenta también las prioridades de dichas
entradas.

\subsubsection*{Capa de aplicación}
La capa de aplicación forma parte del plano de control y no se encuentra en el
controlador SDN. Está constituida por aplicaciones que son responsables de
gestionar, configurar, decidir y monitorear las características de una red SDN
como, por ejemplo, las entradas de las tablas de flujo que están programadas en
los dispositivos de red. Estas aplicaciones, utilizando la interfaz
\textit{NORTHBOUND}, no solamente influyen en la capa de datos, sino también en
la de control \parencite{sdn_simplified}. Por medio de APIs expuestas por el
controlador, dichas aplicaciones pueden, por ejemplo, acceder a los datos del
estado de la red para reaccionar a cambios de la topología, para lograr un
equilibrio de carga \parencite{load_bal}, para seleccionar las mejores rutas de
los flujos o para aplicaciones de comunicación como la priorización de VoIP
\parencite{voip}. También pueden redireccionar el tráfico con el fin de
inspeccionarlo en el caso de aplicaciones de seguridad \parencite{autoscaling}.

\subsubsection*{Seguridad de la Red} \label{sec:segnetsdn}
Los requerimientos de las aplicaciones de redes tradicionales requieren el
despliegue de una política de seguridad por toda la red y de una configuración
complicada e individual en los \textit{firewalls} \parencite{distr_firewall} y/u
otros dispositivos \parencite{detect_chi}. SDN ofrece una plataforma adecuada
para centralizar, combinar y verificar políticas y configuraciones, a fin de
garantizar una buena implementación de la protección y de la prevención de
problemas de seguridad de manera proactiva en la red. Esto significa que una
empresa puede ampliar sus capacidades de defensa bloqueando ataques específicos
y haciendo cambios para adaptarse a las nuevas amenazas.

El controlador SDN puede impulsar las actualizaciones de la política de
seguridad global de forma centralizada a través de la red, y un dispositivo
puede filtrar paquetes y redirigir el tráfico sospechoso a otros dispositivos de
seguridad para su posterior análisis.

\subsubsection*{Ventajas y desventajas de SDN con respecto a las redes tradicionales}
De acuerdo a \parencite{RFC7149}, \parencite{sdn_universidad} y
\parencite{sdn_approach}, una de las ventajas de SDN con respecto a las redes
tradicionales comprende la configuración centralizada e independiente de los
fabricantes de los equipos de red, lo que permite una mayor consistencia en las
políticas de red y un favorecimiento a la escalabilidad, agilidad, flexibilidad
y posibilidad de innovación, ya que en menor tiempo las empresas pueden
desplegar o modificar servicios, aplicaciones e infraestructura de sus redes. 


Otra de las ventajas consiste en la disminución de ciertos costos. Por un lado,
una reducción del costo de capital (CAPEX) ya que el objetivo de SDN es utilizar
estándares abiertos. Las empresas pueden usar fácilmente opciones de múltiples
proveedores y no estar limitadas a un solo vendedor. Por el otro, una disminución
del costo de operación (OPEX), ya que el esfuerzo y el tiempo necesarios para
realizar tareas de modificación y/o mantenimiento se reduce. Además si las
configuraciones de red se pueden hacer de manera automatizada mejora la
trazabilidad de las operaciones.

Teniendo en cuenta las mismas fuentes de información, entre las
desventajas se encuentran la escalabilidad de la red limitada a la capacidad del
controlador y las inconsistencias que sufren las soluciones actuales en algunas de
sus características y funciones, por tratarse de un paradigma nuevo. Con respecto 
a la seguridad existe la posibilidad de dejar fuera de servicio a la red si se 
efectúa un ataque al controlador (lógica centralizada), si se intenta suplantar 
la identidad del mismo o si se trata de interrumpir la comunicación entre éste y 
los dispositivos SDN.


\section {Denegación de servicio}	

En pocas palabras, en \parencite{ddos_amir} y \parencite{ddos_kumar} se hace
referencia a la situación en la que el usuario normal no puede acceder a un
servicio debido a intentos malintencionados que impiden el acceso al mismo. Para
el caso de una red, el impedimento podría venir dado por la disminución del
ancho de banda o por el consumo de los recursos que sustentan el servicio.

\subsection {DoS (Denegación de servicio)}

Todo el tráfico creado maliciosamente proviene de una sola fuente, la cual
inunda la red con un tráfico abrumador para evitar o disminuir transacciones
legítimas \parencite{ddos_mirkovic}. El atacante puede (aleatoriamente) cambiar
la dirección IP de origen en un intento de ocultarse pero, en este tipo de
ataques, solo hay un remitente que elabora el tráfico que luego se dirige
directamente al objetivo sin que se utilice ningún \textit{host} intermedio. Por
ende, dichos ataques ya no son muy comunes debido a que son fáciles de detectar,
atribuir y mitigar \parencite{ddos_hh}.

\subsection {DDoS (Denegación de servicio distribuido)}

Consiste en la versión amplificada del anterior. En este tipo de ataque, varios
\textit{hosts} (a veces cientos de miles) se centran en un solo objetivo. Cada
computadora con acceso a Internet puede comportarse como un atacante, momento en
el que pasan a ser conocidas como \textit{bots} (\textit{zombies}) y el conjunto
de ellos forman la denominada \textit{botnet} \parencite{ddos_amir}.

En un ataque DDoS, el atacante infecta una red de computadoras (\textit{botnet})
con algún tipo de \textit{malware} que le permite manejarlas de manera remota
\parencite{ddos_kumar}. Cuando llegue la ocasión, el atacante envía la orden al
\textit{botnet}, donde cada uno de los \textit{bots} integrantes lanzan un
ataque hacia el \textit{host} víctima (ver Figura \ref{fig:ddos_flood}). De esta
forma, el verdadero atacante mantiene su identidad oculta.

Los ataques de denegación de servicio distribuido pueden intentar limitar el
ancho de banda de los enlaces de la red o consumir los recursos del
\textit{host} víctima. A continuación, se explicarán en las siguientes
subsecciones los ataques que afectan al ancho de banda, los cuales se
clasifican, de acuerdo a \parencite{ddos_amir} y \parencite{ddos_hh}, en ataques
por inundación (\textit{flood}) y en ataques por reflexión y amplificación.
\\

\begin{figure}[H]
	\centering 
	% \resizebox{.85\textwidth}{!}{\includegraphics{Figures/sdn-arch.png}}%
	\includegraphics[scale=0.7]{ddos_flood}
	\caption[Ataque DDoS]{Ataque DDoS \parencite{ddos_amir}.}
	\label{fig:ddos_flood}
\end{figure}


\subsubsection*{Inundación}

El ataque de inundación (\textit{flooding}) está diseñado para hacer que una red
o servicio no se encuentre disponible para usuarios legítimos debido a su
colapso por las grandes cantidades de tráfico que recibe proveniente de
distintos \textit{hosts} tales como computadores de oficina, portátiles,
servidores o dispositivos IoT. Un solo \textit{host} puede generar decenas de
megabits por segundo de tráfico malicioso y, mientras más \textit{hosts} generen
este tipo de tráfico, mayor será la efectividad del ataque \parencite{ddos_hh}.

Una vez que la \textit{botnet} está establecida, el atacante puede dirigir el
atentado enviando instrucciones a cada \textit{bot} vía control remoto. Tal como
se observa en la Figura \ref{fig:ddos_flood}, cuando la mencionada
\textit{botnet} fija la dirección IP de una víctima, cada \textit{bot}
responderá enviando solicitudes al objetivo, lo que posiblemente causará que el
servidor o la red desborden su capacidad, resultando en una denegación de
servicio al tráfico normal. En vista de que cada \textit{bot} es un dispositivo
de Internet genuino, puede ser difícil determinar una fuente concreta del
tráfico malicioso \parencite{ddos_flloogind}.

De acuerdo a \parencite{ddos_hh}, el primero de estos ataques por inundación fue
el denominado \verb|ping flood|, en donde se enviaron tantos paquetes de eco
ICMP (\verb|ping|) a un \textit{host} como fue posible. Luego fueron
reemplazados por la inundación de paquetes TCP explicados en la Sección
\ref{sec:flood} que, debido a la necesidad de tres paquetes para iniciar una
conexión, pueden generar más tráfico y, a su vez, permiten lograr un mayor
consumo de los recursos de las víctimas. Posteriormente, estos últimos ataques
fueron reemplazadas por inundaciones de paquetes UDP ya que, al ser un protocolo
sin estado de la conexión, permite que se falsifique la dirección IP de origen,
dificultando el rastreo y el bloqueo de dichas inundaciones.


\subsubsection*{Reflexión y amplificación}

Esta categoría de ataques de amplificación o reflexión, también llamados DRDoS
(\textit{Distributed Reflection Denial of Service}), utilizan fallas de
protocolo y otras vulnerabilidades para amplificar las cantidades de datos
transmitidos contra un sistema objetivo. Dichos ataques utilizan dispositivos
conocidos como amplificadores o reflectores, los cuales pueden ser máquinas
legítimas o dispositivos intermedios y, en comparación con las redes de
\textit{bots}, generalmente no están infectados y no están controlados
\parencite{drdos_kumar}.

El primer principio sobre el que se basa este tipo de ataques es la reflexión,
la cual se produce cuando un atacante falsifica la dirección IP de origen de los
paquetes, no con una dirección aleatoria sino simulando ser la víctima
\parencite{ddos_amir}, y enviando los mencionados paquetes a \textit{hosts} que
no pueden distinguir las solicitudes legítimas de las falsificadas (en especial
si se utiliza UDP o ICMP, al ser protocolos que no mantienen el estado de la
conexión). Por lo tanto, responden directamente a la víctima como se observa en
la Figura \ref{fig:ddos_reflexion}.

El segundo principio es la amplificación. Su concepto es hacer que el
\textit{host} intermedio, el cual no es consciente de ser un amplificador,
desencadene una respuesta con un mayor tamaño que la petición inicial. Esto
multiplica de manera efectiva el flujo de tráfico del atacante
\parencite{ddos_andre}. El impacto en estos \textit{hosts} de amplificación es
mínimo, no obstante el tráfico dirigido al objetivo podría ser mucho más
perjudicial.

Ejemplificando, uno de los ataques de reflexión y amplificación es Smurf,
descrito en la Sección \ref{sec:smurf}, que utiliza una dirección falsificada
y el direccionamiento de difusión para amplificar un flujo de paquetes. El
sistema de amplificación objetivo es una red que permite la comunicación con la
dirección de difusión (\textit{broadcast}) y posee un número relativamente alto
de \textit{hosts} activos \parencite{art_exploit}.

Otro de los ataques más frecuentemente utilizados es el ataque de amplificación
DNS, en el cual los atacantes usan servidores DNS de acceso público enviando
solicitudes de búsqueda, con el objetivo de inundar un sistema de destino con
tráfico de respuesta DNS. La solicitud demanda tanta información como sea
posible para maximizar el efecto de la amplificación \parencite{CISA}.


\begin{figure}[H]
	\centering 
	% \resizebox{.85\textwidth}{!}{\includegraphics{Figures/sdn-arch.png}}%
	\includegraphics[scale=0.7]{ddos_refle}
	\caption[Ataque DDoS por reflexión y amplificación]{Ataque DDoS por reflexión y amplificación \parencite{ddos_amir}.}
	\label{fig:ddos_reflexion}
\end{figure}


\section{Detección de intrusos}

La detección de intrusos es el arte de detectar actividad no autorizada
\parencite{ddos_hh}, como son los intentos exitosos o no de conexiones, inicios
de sesión y accesos a recursos no autorizados.

\subsection{Detección de intrusos basada en firmas}
\label{subsec:deteccion_firmas}
 
La detección de intrusos basada en firmas implica buscar un patrón de
comportamiento malicioso en el tráfico de la red como, por ejemplo, una
secuencia de paquetes, una cantidad específica de paquetes de cierto tipo o una
serie de bytes determinados al analizar el encabezado de un paquete o la carga
útil del mismo \parencite{ids_snort}. Si la actividad coincide con alguna de las
firmas de la base de datos de un sistema de detección de intrusiones, éste emite
una alarma indicando una actividad sospechosa.

Las mencionadas firmas de estos sistemas se construyen en base a las reglas
definidas en el lenguaje que cada uno proporciona. Estas reglas se encuentran en
bases de datos y se actualizan de manera progresiva. A su vez los sistemas de
detección de intrusiones, al examinar la información del encabezado del paquete,
buscan coincidencias en las direcciones, protocolos, opciones, parámetros del
fragmento, etc. Por otro lado, miran dentro de la carga útil del paquete, en
búsqueda de URLs específicas o mal formadas \parencite{ids_signatures}.

La gran desventaja es que estos sistemas solo detectan ataques conocidos. Un
ataque desconocido, al no coincidir con ninguna entrada en la base de datos, es
totalmente ignorado. Para agregar una entrada es necesario comenzar con la
comprensión del comportamiento del ataque de red para desarrollar la firma
correspondiente. Además, estas firmas también son propensas a falsos positivos,
ya que generalmente se basan en expresiones regulares y concordancia de cadenas
\parencite{ddos_foster}.


\subsection {Detección de intrusos basada en anomalías}

La técnica de detección de anomalías se centra en el concepto de un
comportamiento esperado de la red en lugar de buscar patrones en una base de
datos que se tiene que actualizar de manera progresiva. Este comportamiento de red
aceptado, es aprendido y/o determinado por los administradores de dicha red.
Cualquier comportamiento que se encuentre fuera del modelo predefinido o
aceptado (por ejemplo, una conexión inesperada a un \textit{host}, puerto)
genera eventos por detección de anomalías \parencite{ddos_foster}. Este sistema,
mientras más tiempo se encuentre operativo, más aprenderá de la red, además de
que posee el potencial de descubrir amenazas desconocidas.

Sin embargo, las desventajas de estos motores es el resultado de la detección.
Muchas veces las alarmas manifiestan más falsos positivos que la detección
basada en firmas. Además, si solamente se destina el sistema para la detección
de amenazas conocidas, la cantidad de recursos necesarios es elevada en
comparación con el sistema de la sección \ref{subsec:deteccion_firmas}
\parencite{ids_vs}.


\subsection{IDS (Sistema de detección de intrusiones)}

Los sistemas de detección de intrusiones (IDS, por sus siglas en inglés) son
programas de detección de actividades maliciosas y accesos no autorizados en un
computador o una red, generalmente utilizando una combinación de métodos basados
en el comportamiento y/o las firmas para detectar amenazas desconocidas como
conocidas, respectivamente \parencite{ids_snort}. Estos sistemas suelen tener
sensores virtuales, es decir, elementos capaces de capturar, procesar, descifrar
y descomponer paquetes de un segmento de red en tiempo real
\parencite{ids_snort}. Dichos elementos son conocidos como \textit{sniffers} de
red y, en base a ellos, el motor del IDS puede obtener datos externos sobre el
tráfico de la red y detectar la presencia o no de ataques. No obstante, estos
IDSs por sí solos no pueden cortar los mencionados ataques. Es por eso que
generalmente se los integra con un \textit{firewall}. Este último consiste en un
elemento de red que controla el cruce de paquetes a través de los límites de una
red en función de una política de seguridad específica
\parencite{fwll_springer}.

El funcionamiento de los sistemas IDS se basa en la generación de alertas cuando
el tráfico de red inspeccionado coincide con firmas de ataques conocidos o
comportamientos sospechosos (paquetes malformados, escaneos de puertos, etc.)
que se encuentran en la base de datos del correspondiente IDS. También, este
sistema no sólo analiza qué tipo de tráfico es, sino que también revisa el
contenido y su comportamiento.

Por último, la clasificación de los sistemas de detección de intrusiones es la
siguiente:
\begin{itemize}
\item \textbf{HIDS} (\textit{Host} IDS): el HIDS solo protege el sistema
  \textit{host} en el que reside, detectando modificaciones o rastros producto de
  las actividades de los intrusos en el equipo atacado cuando éstos intentan
  adueñarse del mismo \parencite{ids_toolkit}.
\item \textbf {NIDS} (\textit{Network} IDS): se trata de un tipo de IDS que
  detecta ataques que se efectúan a todo el segmento de la red. Su
  funcionamiento se detalla en la subsección siguiente y es el que se utilizará
  en el presente proyecto integrador.
\end{itemize}

\subsubsection* {Sistema de detección de intrusiones en una red} \label{sec:NIDS}

Si nuestro sistema en riesgo se trata de una red, la detección consiste
simplemente en tratar de detectar los signos de un intruso o de alguien con
malas intenciones en la red antes de que se produzcan daños, ya sea por la
pérdida de datos o por la denegación de un servicio.

Un NIDS trabaja con los datos que circulan a través de un segmento de red y el
tipo de sensor utilizado en este sistema puede ser un \textit{sniffer} de red
\parencite{ids_toolkit}\parencite{ids_snort}. A partir de dicho sensor se
analizan todos los paquetes, buscando en ellos patrones sospechosos. Los NIDS no
sólo vigilan el tráfico entrante, sino también el saliente y el tráfico local,
ya que algunos ataques podrían iniciarse desde el propio sistema protegido
\parencite{ids_snort}.


\paragraph {NIDS/Firewall.}

Como se mencionó anteriormente, en los sistemas tradicionales un NIDS no puede
cortar un ataque. Por ende suele estar acompañado de un \textit{firewall}.
Entonces, la combinación de componentes de software o de hardware de NIDS y de
\textit{firewall} se utilizan para evitar los accesos no autorizados, en donde
el NIDS actúa como sensor y el firewall como actuador. (Ver Figura
\ref{fig:firewall_NIDS}).

\begin{figure}[!]
	\centering 
	% \resizebox{.85\textwidth}{!}{\includegraphics{Figures/sdn-arch.png}}%
	\includegraphics[scale=0.7]{nids}
	\caption[Combinación de NIDS con \textit{firewall}]{Combinación de NIDS con \textit{firewall} \parencite{ids_snort}.}
	\label{fig:firewall_NIDS}
\end{figure}

\paragraph{Ubicación de NIDS en una red.}

Acorde a \parencite{ubicacion_nids}, los modos de ubicación de los dispositivos
NIDS en una red son:

\begin{itemize}
\item \textbf{Delante del firewall:} genera una gran cantidad de \textit{logs},
  muchos de ellos innecesarios debido a que algunos ataques no sobrepasan a
  dicho \textit{firewall}, por lo que no resultan efectivos.
\item \textbf{Detrás del firewall:} es la ubicación más común. Genera menor
  cantidad de \textit{logs} ya que monitorea y detecta ataques que atraviesan al
  correspondiente \textit{firewall}. A su vez, el emplazamiento del NIDS puede
  conectarse a un \textit{hub}, actuando como \textit{bridge}, o al
  puerto replicador de un \textit{switch}.

\item \textbf {Combinación de casos anteriores:} utiliza dos máquinas ya que se
  requiere la implementación de los dos ítems explicados anteriormente. Permite
  ejercer un mayor control pero usa el doble de la cantidad de hardware. (Ver Figura
\ref{fig:firewall_NIDS}).
\item \textbf {Máquina única firewall/NIDS:} utiliza una sola máquina que cumple
  con las funciones del firewall y del NIDS.
\item \textbf {Otras ubicaciones:} usada para ofrecer distintos grados de
  seguridad, como es el caso de IDSs monitoreando el tráfico de distintos
  segmentos de red.
\end{itemize}

\section {Proveedor de Servicios de Internet} \label{sec:isp}

En \parencite{isp_sun} se define a un ISP como aquel sistema que proporciona
servicios de Internet a suscriptores comerciales y residenciales (usuarios).
Además, es el encargado de conectar a los proveedores de servicios básicos como
correo electrónico, alojamiento web y noticias con los usuarios finales. Por
otra parte, los ISP ofrecen sus propios servicios de valor agregado como, por
ejemplo, la seguridad.

\subsection{Infraestructura básica}

En \parencite{isp_sun} y \parencite{isp_cisco} una infraestructura ISP consta de
las siguientes partes:

\begin{itemize}
\item \textbf{Capa central o capa núcleo.} Es la red troncal de conmutación de
  alta velocidad de la red. Un enrutador central debe poder admitir múltiples
  interfaces de alta velocidad.
\item \textbf{Capa de distribución de la red.} Es el punto de delimitación entre
  la capa central y la de acceso.
\item \textbf {Capa de acceso.} Consiste de dos partes: 
\begin{itemize} 
\item \textbf{Subcapa de servicio.} Es aquella que proporciona acceso a los
  servicios. Es comúnmente donde se encuentran los servidores, como el servidor
  web, el servidor de correo electrónico, el servidor proxy y los
  \textit{firewalls}.
\item \textbf{Subcapa de acceso.} Las funciones principales de la capa de acceso
  son conectar varias redes de área local (LAN) de usuarios finales a la capa de
  distribución. Ésta capa se caracteriza por un entorno de ancho de banda
  compartido.
\end{itemize}
\item \textbf{Capa de frontera.} Es aquella que permite la comunicación entre
  distintos proveedores de servicio.
\end{itemize}

\subsection{Tipos de enrutadores}
Los tipos de enrutadores en la infraestructura de un ISP, tal y como muestra la
Figura \ref{fig:ISP}, son los siguientes \parencite{isp_cisco}:

\begin{itemize}
\item \textbf{Enrutadores principales (Nucléo):} se ubican en la capa central y presentan
  múltiples interfaces de alta velocidad.
\item \textbf {Enrutadores de distribución:} se encuentran en la capa de
  distribución y presentan una mayor densidad de puertos y una menor velocidad
  que los enrutadores anteriores.
\item \textbf {Enrutadores de acceso:} se ubican en la subcapa de acceso. Presentan 
una mayor densidad de puertos, una menor velocidad y brindan acceso a Internet a los
usuarios finales.
\item \textbf {Enrutadores de frontera:} se encuentran en la capa de frontera y
  presentan conexiones hacia otros proveedores de servicio.
\item \textbf {Enrutadores de servicios:} se ubican en la subcapa de servicio y le 
brindan acceso a Internet a los servidores y a los distintos servicios.
\end{itemize}


\begin{figure}[H]
	\centering 
	\includegraphics[scale=0.7]{ips}
	\caption[Topología de un ISP tradicional sin capa de frontera]{Topología de un ISP tradicional sin capa de frontera \parencite{isp_cisco}.}
	\label{fig:ISP}
\end{figure}
% Chapter Template
% cSpell:words parencite onfwhitepaper includegraphics resizebox sdncomponents
% linewidth comparqui redireccionar enrutamiento subfigure toposdn Nicira toposinsdn
\chapter{Análisis de las tecnologías y planificación} % Main chapter title 

\label{Chapter3} % for referencing this chapter elsewhere, use \ref{ChapterX}

En el presente capítulo se enumeran las tecnologías y las herramientas
utilizadas en este proyecto para que, en base a los conceptos descritos en el
marco teórico, se pueda implementar, desplegar y validar una solución a la
problemática que resuelve este trabajo.

Por otra parte, se destina un fragmento del capítulo para añadir detalles acerca de
la aplicación de las buenas prácticas de la Ingeniería de Software en la
planificación, gestión y administración del mencionado proyecto.

%--------------------bitext comileArquitectura del protocolo--__------------------------
%	SECTION 1
%---------------------------------------------------------------_-----------------------
\section{Herramientas principales} \label{sec:hprincipales}

Para llevar a cabo la solución de detección y mitigación de ataques DoS y DDoS,
se debieron buscar herramientas que se encuentren en el mercado, teniendo en
cuenta que éstas fueran de carácter \textit{open source} y que permitan ahorrar tiempo de
desarrollo. A continuación se describirán en las presentes subsecciones aquellas
utilizadas en este proyecto.

\subsection {Controlador SDN}
Al comienzo del proyecto se realizó la búsqueda y selección de un controlador
SDN de naturaleza \textit{open source}, con el fin de introducirle directamente
aplicaciones para lograr soluciones rápidas y efectivas que cumplan con los
objetivos del proyecto. Existen varios controladores SDN disponibles en el
mercado, siendo algunos de ellos Ryu, el cual está íntegramente desarrollado en
Python; OpenDaylight (ODL), que es un proyecto perteneciente a \textit {The
  Linux Fundation} y \textit{Open Network Operating System} (ONOS), que
pertenece a \textit{Open Networking Foundation}. Estos dos últimos comparten la
característica de estar desarrollados en Java y, en contraste con Ryu y demás
controladores \textit{open source} como NOX, OpenContrail, Floodlight, Beacon,
POX, OpenMuL, entre otros, presentan una mayor viabilidad comercial visible, una
mejor documentación y sus funcionalidades se renuevan constantemente. Siguiendo
con la justificación de la elección, para este proyecto se escogió el
controlador SDN ONOS debido a:

\begin{itemize}
\item {Una mejor documentación. Esta es mucho más clara y simple que la de ODL,
    ya que se brindan ejemplos de aplicaciones, tutoriales, entre otras ayudas,
    que facilitan el proceso de aprendizaje. }
\item {Un mejor y mayor acceso a información y a la resolución de dudas debido al
    contacto con personas que trabajaron en proyectos que utilizaban dicho
    controlador. }
\item {ONF promociona, además de las redes definidas por software, a este controlador \parencite{wp_onos}. Además, 
    empresas tales como Google, China Unicom, AT\&T, Verizon, Cisco, Intel,
    Samsung, Huawei, Telefónica, etc., participan en esta fundación.}
\item {Su arquitectura de núcleo distribuido permite lograr una mayor
    escalabilidad, rendimiento y disponibilidad \parencite{tech_onos} (Ver
    Figura \ref{fig:distributed_core_onos}). }
\item {Está pensado para ser utilizado en los ISP \parencite{wp_onos}.}
\item {Soporta redes y capas ópticas, lo cual es muy útil teniendo en cuenta el
    presente y el futuro de los proveedores de servicios de Internet
    \parencite{wp_onos}. }
\end {itemize}

\begin{figure}[h]
	\centering 
	% \resizebox{.85\textwidth}{!}{\includegraphics{Figures/sdn-arch.png}}%
	\includegraphics[width=0.8\textwidth]{onos_distributed}
	\caption[Arquitectura distribuida del controlador ONOS]{Arquitectura distribuida del controlador ONOS \parencite{onos_arch}.}
	\label{fig:distributed_core_onos}
\end{figure}

Entrando más en detalle, ONOS es un controlador SDN de código abierto disponible
en \parencite{git_onos}. Su arquitectura se muestra en la
Figura \ref{fig:arquitectura_onos} y presenta las siguientes capas
\parencite{onos_wiki}:

\begin{enumerate}
\item {\textbf{\textit{NorthBound} (Consumidor) API:} es la interfaz que el
    controlador ONOS le expone a las aplicaciones de la capa de aplicación.}
\item {\textbf{Núcleo:} es donde se encuentran las funcionalidades principales del
    controlador, por ejemplo, la detección del cambio de estado de los enlaces,
    el descubrimiento de los dispositivos, entre otros servicios de red. Algunas
    aplicaciones del presente trabajo van a localizarse en esta capa.}
\item {\textbf{\textit{SouthBound} (Proveedor) API:} forma parte de la capa
    superior de la interfaz \textit{SouthBound} del controlador. Comunica a los
    proveedores con el núcleo de ONOS.}
\item {\textbf{Proveedores:} forma parte de la interfaz SouthBound del
    controlador. Se comunica con la red vía librerías de protocolos específicos.
    Por otra parte, utiliza la interfaz \textit{SouthBound} API para vincularse
    con el núcleo de ONOS, proporcionando al mismo descripciones y abstracciones
    de los dispositivos de red o de los eventos del plano de datos. }
\item {\textbf{Protocolos:} forma parte de la capa inferior de la interfaz
    SouthBound del controlador y contiene la implementación de los protocolos de
    control y administración que ONOS necesita para poder comunicarse con los
    dispositivos de red SDN. Ejemplos de estos protocolos son: OpenFlow,
    NETCONF, etc.}
\end {enumerate}


\begin{figure}[h]
	\centering 
	% \resizebox{.85\textwidth}{!}{\includegraphics{Figures/sdn-arch.png}}%
	\includegraphics[width=0.8\textwidth]{onos_stack}
	\caption[Arquitectura de ONOS]{Arquitectura de ONOS \parencite{onos_wiki}.}
	\label{fig:arquitectura_onos}
\end{figure}

En este proyecto se utiliza la versión del controlador ONOS 1.13.2.

\subsection{Emulación de redes SDN}

Además de un controlador SDN, se necesitó la ayuda de una herramienta que
permita emular una topología de red con dispositivos SDN que posean capacidad
OpenFlow, debido a los costos o al tiempo que demandaría desarrollar uno de
estos dispositivos si se construyera un escenario físico.

Para generar dicha emulación se utilizó ContainerNet. Se trata de un proyecto de
software de código abierto \parencite{cnet_git} que brinda un entorno de
emulación rápido, flexible, potente y liviano para redes SDN. Se encuentra
basado en un proyecto padre conocido como Mininet.

Mininet es una aplicación desarrollada por la Universidad de
Stanford, también de código abierto, que emula redes SDN. Para ello ofrece una
API escrita en Python, la cual permite crear conmutadores OpenFlow virtuales
usando Open vSwitch, \textit{hosts} con \textit{Linux network namespaces} y
enlaces emulados mediante \textit{virtual Ethernet devices} (ver Apéndice
\ref{AppendixB} para mayor información sobre estos dos últimos aspectos de
Linux) \parencite {mininet_wiki}.

Por su parte ContainerNet añade, entre otras cosas, la funcionalidad de permitir
utilizar contenedores de Docker para emular los \textit{hosts} de la topología.

A continuación se describirán las dos principales tecnologías en las que
ContainerNet basa su funcionamiento:

\paragraph{Open vSwitch (OVS).} Se trata de un software de código abierto bajo
licencia Apache 2.0 que consiste en un conmutador virtual multicapa con
capacidad OpenFlow. OVS utiliza módulos del \textit{kernel} de Linux para la 
conmutación de paquetes y, a
diferencia con los \textit{Linux brigdes} (ver Apéndice \ref{AppendixB}), puede
ejecutarse en espacio de usuario permitiendo la configuración de las
tablas de reenvío por medio del protocolo OpenFlow o el protocolo OVSDB, brindando así 
mayor flexibilidad y
portabilidad \parencite{ovs_switch} \parencite{rfc7047}.


\paragraph{Docker.} Es un proyecto muy utilizado por las empresas en la
actualidad porque permite la implementación, la distribución y el despliegue de
aplicaciones dentro de contenedores. Estos últimos son unidades estándar de
software que incluyen y mantienen la plataforma de la aplicación con sus
dependencias \parencite{dif_docker_vm}, facilitando la portabilidad. Los
contenedores comparten el sistema operativo del \textit{host}, como se observa
en la Figura \ref{fig:docker}, y se ejecutan como procesos separados en el
espacio de usuario. En otras palabras, realizan una virtualización a nivel de
sistema operativo, a diferencia de las máquinas virtuales, en donde el hardware
también se virtualiza. De este modo, los primeros son más livianos que estas
últimas y generan menos sobrecarga en los respectivos despliegues. Por su parte,
en Linux, Docker utiliza características del kernel para generar y administrar
sus contenedores, como son los \textit{namespaces}, que permiten la aislación de
recursos, y los \textit{cgroups}, utilizados para la gestión de dichos recursos
(ver Apéndice \ref{AppendixB} para más información) \parencite{wisc_docker}.

\begin{figure}[!]
	\centering 
	% \resizebox{.85\textwidth}{!}{\includegraphics{Figures/sdn-arch.png}}%
	\includegraphics[scale=0.45]{docker}
	\caption[Diferencias entre máquina virtuales y contenedores de Docker]{Diferencias entre máquina virtuales y contenedores de Docker
    \parencite{dif_docker_vm}.}
	\label{fig:docker}
\end{figure}


\subsection{Detección de Intrusos} \label{sec:herramienta_snort}

Una opción de un sistema que permita detectar los ataques DoS o DDoS podría ser
el análisis de todo el tráfico que circula por la red por parte del controlador
y, en base a ello, tomar las decisiones correspondientes al ataque.  Esto
conlleva a dos desventajas que hace a esta solución inviable:

\paragraph{Solución no escalable.} A medida que la red crece, el aumento del tráfico implica
un procesamiento mayor por parte del controlador, apartando a las demás
funciones del mismo. Además, las mejoras tecnológicas en los controladores son
onerosas. 
\paragraph{Posibilidad de ataque al controlador.} Como todo el tráfico se
analizaría en dicho controlador, se generaría un rápido agotamiento de los
recursos del mismo cuando se esté en presencia de un ataque DoS o DDoS a la red
o a algún \textit{host} por la gran cantidad de tráfico. De esta forma se
provocaría su colapso y, posiblemente, deje fuera de servicio a toda la red.

Una solución alternativa sería analizar el tráfico con sensores
(IDSs) y, en caso de detección de comportamientos sospechosos en la red, enviar
% la alerta correspondiente al controlador (Ver Figura \ref{fig:snort}). Para
la alerta correspondiente al controlador. Para
llevar a cabo esto, es necesario buscar y elegir el sistema de detección de
intrusos más adecuado que exista en el mercado. Entre los NIDS \textit{open source} más
populares se encuentra Snort.

\subsubsection*{Snort}
Es uno de los sistemas más utilizados y permite múltiples configuraciones,
siendo una de ellas la de funcionar como \textit{sniffer} para monitorear el
tráfico de la red \parencite{sniffer}. Es un sistema de detección basado en
firmas y, en contraste con su par \textbf{Suricata}, posee una abundante
documentación disponible. A diferencia de \textbf{Zeek}, Snort presenta una
instalación y un uso sencillos y goza de un gran soporte producto de su extensa
comunidad. Esto permite obtener reglas implementadas para detectar distintos
tipos de ataques. En resumen, debido a la estabilidad, soporte y robustez se
escogió a \textbf{Snort} como IDS en este proyecto.

Entrando en más detalle, este sistema de detección de intrusos presenta tres
modos de funcionamiento: como NIDS basado en firmas, como \textit{sniffer} de la
red y/o como registrador de paquetes \parencite{snort_manual}. En el presente
proyecto se lo emplea en el primer modo. En éste realiza una inspección del
enlace de la red, buscando intrusiones en base a las firmas preestablecidas y
actuando en función de éstas, generando alertas o registros de eventos
(\textit{logs}), ignorando paquetes, etc. Además, Snort tiene su propio lenguaje
para crear las firmas, el cual se detalla en el Apéndice \ref{AppendixC}.

Con respecto a la arquitectura interna de Snort, el mismo se caracteriza por
presentar el flujo de datos de la Figura \ref{fig:arch_snort}. En éste, acorde a
\parencite{ids_w_snort} y \parencite{pos_sec_sis} se observan las siguientes
etapas:

\begin{itemize}
\item \textbf{Captura de paquetes:} en esta etapa el \textit{sniffer} de red
  analiza el tráfico constantemente y transmite los mencionados paquetes hacia
  el decodificador.
\item \textbf{Decodificador de paquetes:} determina los protocolos que se usan
  en los paquetes que envía la etapa anterior. Paso siguiente, forma una trama
  en donde no sólo incluye el contenido del paquete, sino también su tamaño y
  los protocolos que utiliza. Por último, la trama se envía al preprocesador.
\item \textbf{Preprocesador:} le da forma a lo que recibe del decodificador
  (como por ejemplo, ordenando los paquetes, desfragmentándolos, decodificando
  \textit{URLs}, etc.) y lo envía al motor de detección. Esto lo realiza con el
  fin de entregar a la siguiente etapa información cuyo procesamiento sea más
  sencillo y rápido.
\item \textbf{Motor de detección:} genera la alerta en caso de que la
  información proveniente de la etapa anterior coincida con alguna de las firmas
  de ataque, creadas por este motor en base al análisis de las reglas definidas
  por el administrador del IDS.
\item \textbf{Complemento (\textit{Plug-in}) de salida:} es el encargado de
  recolectar y enviar las alertas hacia una salida determinada, como por ejemplo,
  impresión por consola, escritura en archivos (\textit{logs}) o envíos a entes
  remotos a través de \textit{sockets}. En este proyecto se utilizará \textit{sockets}
  como complemento de salida.

\end {itemize}

\begin{figure}[!]
	\centering 
	\includegraphics[width=0.8\textwidth]{arch_snort}
	\caption{Arquitectura de Snort.}
	\label{fig:arch_snort}
\end{figure}


\subsection{Generación de tráfico legítimo}
Con el objetivo de evaluar el desempeño de las aplicaciones, se debió generar
tráfico legítimo. Dentro de las herramientas utilizadas para ello se encuentran CURL y
Siege.

\textbf{CURL} es un software \textit{open source} orientado a la transferencia de
archivos que permite hacer consultas HTTP GET, POST, etc. a un servidor
web. %\parencite{curl}.

\textbf{Siege} se trata de un \textit{benchmarking} multihilo que permite a
desarrolladores testear sus aplicaciones web empleando carga HTTP o HTTPS. Su
funcionamiento consiste en realizar consultas GET a los servidores, siendo
configurables la cantidad de consultas, el tiempo entre consultas, la cantidad
de \textit{sockets} concurrentes, etc. Monitorea el tiempo de respuesta por
parte del servidor, la disponibilidad del mismo, el porcentaje de transacciones
completadas con éxito en un determinado intervalo de tiempo, entre otras
variables. % \parencite{siege}.


\subsection{Generación de tráfico malicioso}
Para poder analizar y comprobar el correcto funcionamiento de las aplicaciones
desarrolladas, se debieron generar ataques DoS y DDoS de tipo \textit{flood} y
de tipo reflexión y amplificación. En las siguientes subsecciones se explicarán
aquellos ataques que se llevaron a cabo y la herramienta correspondiente para
producirlos.

\subsubsection * {Flood} \label{sec:flood}

De acuerdo a las estadísticas sobre ataques DDoS correspondientes al último
cuatrimestre del año 2018, brindadas por SecureList de Kaspersky
Lab \parencite{ref_kalpesky}, evidencia que la inundación SYN, conocida como
SYN-flood, es uno de los ataques más comunes y predominantes, seguido
por la inundación UDP y más atrás la de TCP, tal como se observa en la Figura
\ref{fig:statistics}.

\begin{figure}[!]
	\centering 
	\includegraphics[width=0.55\textwidth]{ddos_type}
	\caption[Distribución de los ataques DDoS por tipo en el último trimestre de
    2018]{Distribución de los ataques DDoS por tipo en el último trimestre de
    2018 \parencite{ref_kalpesky}.}
	\label{fig:statistics}
\end{figure}

Teniendo en cuenta los ataques DDoS por inundación más predominantes del último
cuatrimestre del año 2018 mencionados anteriormente, es decir, SYN-Flood, UDP
flood y TCP-Flood, se obliga a escoger los siguientes ataques a realizar con el
fin de probar las aplicaciones desarrolladas:

\begin{itemize}
\item \textbf{Ataque TCP SYN-Flood.} Un cliente intercambia tres mensajes (SYN, SYN-ACK,
  ACK) para iniciar una conexión TCP con un servidor, como se observa en la
  Figura \ref{fig:tcp_hs}. En un ataque TCP SYN, el atacante envía una sucesión
  de solicitudes SYN al sistema o \textit{host} objetivo sin confirmar la
  recepción de los mensajes SYN-ACK enviados por parte del servidor, dando como
  resultado conexiones TCP incompletas, tal como se ve en la Figura
  \ref{fig:tcp_syn}. Todo esto se efectúa con el objetivo de consumir suficientes
   recursos de
  la red y del servidor para que este último no logre responder al tráfico
  legítimo, ya que dicho servidor pierde tiempo de procesamiento y puede o no,
  dependiendo su configuración, reservar recursos innecesarios en esas
  peticiones ilegítimas.
\item \textbf{Ataques TCP PUSH-ACK \textit{flood}, TCP FIN \textit{flood}, 
TCP RST \textit{flood}.} En este tipo de ataques
  el servidor víctima recibe paquetes ACK, FIN o RST falsos que no pertenecen
  a ninguna de las sesiones en la lista de transmisiones de dicho servidor. El
  \textit{host} bajo ataque desperdicia sus recursos (RAM, procesador,
  etc.) tratando de definir a qué conexión pertenecen los paquetes, afectando su
  capacidad de respuesta y de procesamiento y su disponibilidad. Por otra parte, si
  se coloca un uno en el bit que representa al \textit{flag} PUSH, se fuerza al
  servidor a procesar la información del paquete recibido, ya que apenas
  llega se envía directamente a capa de aplicación sin esperar ningún otro tipo
  de paquete en el buffer de recepción TCP \parencite{push_urg}, tal como se observa
  en la Figura \ref{fig:tcp_prf}. Esta solicitud requiere que el mencionado servidor
  haga más trabajo. En resumen, ataques
  de inundación con estos bits activados (PUSH, ACK, FIN y RST) generan un agotamiento, tanto en la
  capacidad de procesamiento y de respuesta del servidor como en el ancho de banda de los
  enlaces de la red.
\item \textbf{Ataque UDP \textit{flood}.} Un atacante envía grandes cantidades
  de tráfico compuesto por paquetes UDP hacia un \textit{host} objetivo, con el fin de
  agotar el ancho de banda de los enlaces y la capacidad de procesamiento y de
  respuesta del mencionado \textit{host}.
\end {itemize}

\begin{figure}[H]
	\centering 
  \begin{subfigure}[b]{0.4\textwidth}
    \centering
    \includegraphics[scale=0.4]{tcp_hs}
    \caption{Negociación TCP en tres pasos}
    \label{fig:tcp_hs}
  \end{subfigure}
  \begin{subfigure}[b]{0.4\textwidth}
    \centering
    \includegraphics[scale=0.4]{tcp_syn}
    \caption{Ataque TCP SYN-Flood}
    \label{fig:tcp_syn}
  \end{subfigure}
  \caption{Comparación del uso de las \textit{flags} TCP}
  \label{fig:tcp_hs_syn}
\end{figure}

\begin{figure}[H]
	\centering 
	\includegraphics[scale=0.4]{tcp_prf}
	\caption{Ataque TCP PSH-ACK, FIN, RST}
	\label{fig:tcp_prf}
\end{figure}

%\\

\subsubsection * {Reflexión y amplificación} \label{sec:smurf}

Dentro de esta categoría se encuentra el ataque Smurf, el cual se trata de un
ataque de denegación de servicio distribuido de capa tres que se utiliza para
agotar el ancho de banda de los enlaces de una red \parencite {ddos_amir}. 
Este ataque utiliza la suplantación de identidad y el
direccionamiento de difusión para amplificar un flujo de paquetes. Es decir,
 el atacante
envía paquetes ICMP relativamente pequeños y con una dirección de origen
falsificada hacia una dirección de difusión (\textit{broadcast}) como destino. 
Por lo tanto, retransmitirá estos paquetes a todos los \textit{hosts} presentes 
en dicha red de difusión, que luego 
saturarán de respuesta eco ICMP a la máquina de la víctima \parencite{art_exploit},
 como se observa en la Figura \ref{fig:smurf}.

\begin{figure}[H]
	\centering 
	\includegraphics[width=0.7\textwidth]{smurf}
	\caption{Comportamiento del ataque Smurf.}
	\label{fig:smurf}
\end{figure}

\subsubsection * {Hping3}

Para generar los ataques mencionados en las subsecciones anteriores, se utilizó
la herramienta Hping3 \parencite{hping3}\parencite{hping3_man}. Se trata de una aplicación disponible
para Linux que permite modificar los paquetes enviados a través de TCP/IP. Es
utilizado para testear la eficacia y eficiencia de los \textit{firewalls} al
generar, a través de distintos protocolos, diferentes tipos de ataques
informáticos.

Luego, para los ataques DDoS se hizo necesario comandar
la ejecución simultánea, mediante múltiples instancias de ejecución, de los
comandos que permiten generar dichos ataques a través de diferentes \textit{bots}.
(Ver comandos utilizados en la sección Apéndice \ref{AppendixA}).


\section{Aplicación de la Ingeniería de Software} \label{sec:hIngSW}

Durante el transcurso de la carrera se logró apreciar el papel importante y
fundamental que conllevan las buenas prácticas de la Ingeniería de Software. Es por ello que en las siguientes subsecciones se describirán brevemente:

\begin{itemize}
\item{El proceso de desarrollo de software empleado.}
\item{Las herramientas que fueron utilizadas a la hora de organizar y gestionar el proyecto.}
\item{Los riesgos tenidos en cuenta al comienzo del presente trabajo y sus respectivas importancias.}
\end{itemize} 


\subsection  {Metodología de trabajo}

En este proyecto se han utilizado metodologías ágiles con ayuda  de herramientas mencionadas al final del capítulo. A su vez, dentro de todos los procesos de desarrollo de software que ofrece esta área de la ingeniería, se ha utilizado el iterativo incremental, gracias a sus ventajas en cuanto a la posibilidad y el favorecimiento de la separación de un proyecto complejo en etapas o iteraciones. Dichas etapas fueron planeadas en base a los objetivos descritos en el Capítulo \ref{Chapter1}, tal y como se muestra a continuación:

\begin{itemize}
\item {\textbf{Iteración 1:} diseño e implementación de una topología de red administrada por un controlador SDN, constituida por dos \textit{hosts} y un conmutador OVS. Esta iteración se debe efectuar con el fin de familiarizarse con las herramientas a emplear en el trabajo.}
\item {\textbf{Iteración 2:} topología de red de la etapa anterior con la inclusión de sistemas de detección de intrusiones imprimiendo alertas en sus respectivas consolas frente al tráfico ICMP que atraviesa al \textit{switch} SDN correspondiente. Esta etapa posee como objetivo el aprender a configurar y utilizar Snort.}
\item {\textbf{Iteración 3:} topología administrada por ONOS que incluye \textit{hosts} con posibilidades de generar ataques y/o tráfico legítimo, además de los dispositivos de las etapas anteriores. Parte fundamental de esta iteración es la adquisición de conocimientos para la elaboración de los distintos ataques.}
\item {\textbf{Iteración 4:} diseño e implementación de una topología de red, representativa de las infraestructuras existentes, y que permita validar la solución desarrollada a lo largo del proyecto. Un posible caso podría ser la de algún ISP.}
\item {\textbf{Iteración 5:} integración de una aplicación en el controlador que detecte presencia de posibles ataques por medio de la recolección de métricas desde el plano de datos.}
\item {\textbf{Iteración 6:} integración de una aplicación en ONOS que permita filtrar el tráfico malicioso en el plano de datos, a partir de la información recibida de alertas enviadas por los sistemas de detección de intrusos.}
\item {\textbf{Iteración 7:} diseño, implementación e integración al sistema
    desarrollado, de las APIs y de la interfaz web de usuario que permitan
    configurarlo y monitorearlo de una manera sencilla.}
\end{itemize}


\subsection  {Herramientas utilizadas}

En el mercado existen herramientas disponibles y muy utilizadas que facilitan la aplicación de las buenas prácticas de la Ingeniería de Software y que se detallan a lo largo de esta sección. 

\subsubsection *{Gestión y construcción de proyectos}

De acuerdo a \parencite{apache_maven} y \parencite{java_maven}, Maven
consiste en una herramienta \textit{open source} que permite construir y
administrar cualquier proyecto basado en Java. Entre otras cosas, realiza la
compilación del código, soluciona y gestiona dependencias, ejecuta pruebas y
genera informes y/o documentación. Para lograr estas funcionalidades, utiliza un
archivo POM, el cual contiene la configuración y descripción de las
características del proyecto útiles para Maven.

\subsubsection *{Metodologías ágiles}

De acuerdo a \parencite{trello} y
\parencite{prod_trello}, Trello es una herramienta sencilla y gratuita de
administración y organización de proyectos en tableros y tarjetas visuales
(metodología Kanban), de forma tal que se puede saber cuáles tareas se están
llevando a cabo, quién del equipo tiene asignado una determinada tarea y cuál es
el estado de un proceso. Al presente proyecto se lo dividió en historias de
usuario y a cada una se la incluyó en una tarjeta. Dentro de cada tarjeta, a su
vez, se crearon \textit{checklists} con las distintas tareas que integran alguna
historia en particular.


\subsubsection *{Modelado UML}

Visual Paradigm es la herramienta utilizada en el presente proyecto para llevar a cabo los
distintos diagramas UML necesarios en la etapa de diseño. Posee una \textit{Free
  Community Edition} para uso no comercial, la cual fue utilizada en el presente
trabajo.


\subsubsection *{Control de versiones}

Uno de los aspectos fundamentales de la Ingeniería de Software es el control de versiones y la administración de códigos fuente. Para ello, en este proyecto utilizamos Git, usando un modelo de ramificación en el repositorio conocido como \textit{Trunk-Based Development}. Este último consiste en una rama principal, de la cual se desprenden otras por cada versión entregable del producto o iteración. Además, frente a cada nueva funcionalidad a implementar se crean nuevas bifurcaciones que se fusionan a la principal una vez concluidas.

%GitLab\parencite{gitlab} es una plataforma para planeamiento de proyectos, administración de códigos fuente, integración continua, entre otras funciones, que permite aplicar técnicas de la Ingeniería de Software muy utilizadas en este proyecto, como por ejemplo, el control de versiones, el desarrollo en distintas ramas y la creación de \textit{wikis}.


\subsection{Riesgos}

Para que el trabajo pueda desarrollarse y llevarse a cabo dentro de los tiempos establecidos, se debe confeccionar un estudio y administración de los distintos riesgos que pueden presentarse en diferentes etapas del proyecto integrador. Para ello se deben identificar y  analizar estos riesgos, para luego generar planes con el fin de poder afrontarlos. La metodología de gestión seguida consiste en una adaptación de la que se detalla en \parencite{Ing_software}.

\subsubsection *{Identificación de los riesgos}

De acuerdo a \parencite{Ing_software}, hay al menos seis tipos de riesgos:

\begin{itemize}
\item {\textbf{Riesgos de tecnología:} asociados a las tecnologías (hardware o software) utilizadas en el proyecto.}
\item {\textbf{Riesgos de personal:} vinculados con los miembros del equipo de trabajo.}
\item {\textbf{Riesgos organizacionales:} derivados del entorno organizacional en donde se genera el proyecto.}
\item {\textbf{Riesgos de herramientas:} producidos a partir de aquellas
    tecnologías de ingeniería de software u otras de apoyo, utilizadas para el
    presente trabajo.}
\item {\textbf{Riesgos de requerimientos:} asociados con los cambios de los requerimientos del cliente y del proceso de gestión de dichos cambios.}
\item {\textbf{Riesgos de estimación:} derivados de estimaciones administrativas en cuanto a características del sistema y a recursos utilizados en la construcción del mismo.}
\end{itemize}


Habiendo definido cada tipo, a continuación se mostrará en la Tabla \ref{tab:id_riesgos} los distintos riesgos identificados con sus respectivas clasificaciones.


% Please add the following required packages to your document preamble:
% \usepackage[table,xcdraw]{xcolor}
% If you use beamer only pass "xcolor=table" option, i.e. \documentclass[xcolor=table]{beamer}
\begin{table}[H]
\centering
\begin{tabular}{|c|
>{\columncolor[HTML]{FFFFFF}}l |c|}
  \hline
  \cellcolor[HTML]{EFEFEF}\textbf{Identificador} & \multicolumn{1}{c|}{\cellcolor[HTML]{EFEFEF}\textbf{Descripción del riesgo}} & \cellcolor[HTML]{EFEFEF}\textbf{Tipo} \\ \hline
  {\color[HTML]{000000} \textbf{R1}} & {\color[HTML]{000000} Uno de los integrantes abandona.} & {\color[HTML]{000000} Personal} \\ \hline
  {\color[HTML]{000000} \textbf{R2}} & {\color[HTML]{000000} El tamaño del software está subestimado.} & {\color[HTML]{000000} Estimación} \\ \hline
  {\color[HTML]{000000} \textbf{R3}} & {\color[HTML]{000000} Cambios en los requerimientos.} & {\color[HTML]{000000} Requerimiento} \\ \hline
  {\color[HTML]{000000} \textbf{R4}} & {\color[HTML]{000000} \begin{tabular}[c]{@{}l@{}}Las versiones del controlador y de sus\\ dependencias son incompatibles con\\ algunas de las herramientas a utilizar en\\ el proyecto.\end{tabular}} & {\color[HTML]{000000} Tecnológico} \\ \hline
  {\color[HTML]{000000} \textbf{R5}} & {\color[HTML]{000000} \begin{tabular}[c]{@{}l@{}}La versión del OVS es incompatible con \\ algunas de las herramientas y tecnologías\\ a emplear en el trabajo.\end{tabular}} & {\color[HTML]{000000} Tecnológico} \\ \hline
  {\color[HTML]{000000} \textbf{R6}} & {\color[HTML]{000000} \begin{tabular}[c]{@{}l@{}}Snort, por dificultades, no puede \\ configurarse adecuadamente.\end{tabular}} & \cellcolor[HTML]{FFFFFF}{\color[HTML]{000000} Tecnológico} \\ \hline
  {\color[HTML]{000000} \textbf{R7}} & {\color[HTML]{000000} \begin{tabular}[c]{@{}l@{}}Las herramientas para generar los \\ ataques presentan problemas en su\\ funcionamiento. (Incompatibilidad, falta\\ de documentación, etc.).\end{tabular}} & \cellcolor[HTML]{FFFFFF}{\color[HTML]{000000} Tecnológico} \\ \hline
\end{tabular}
\caption{Identificación de riesgos.}
\label{tab:id_riesgos}
\end{table}


\subsubsection *{Análisis de riesgos}

Para generar la tabla del análisis de los riesgos, se deben determinar las
probabilidades de ocurrencia de los mismos y los efectos que producen, con el
fin de distinguir cuáles son los más importantes. Para ello se definen las
tablas \ref{tab:probabilidad_riesgo} y \ref{tab:seriedad_riesgo}.

\begin{table}[htbp]
	\centering
	\begin{tabular}{|c|c|}
		\hline
		\rowcolor[HTML]{EFEFEF} 
		\textbf{Identificador de probabilidad} & \textbf{Probabilidad de ocurrencia}      \\ \hline
		\rowcolor[HTML]{FFFFFF} 
		{\color[HTML]{000000} Muy baja}            & {\color[HTML]{000000} 0\% - 25\%}    \\ \hline
		\rowcolor[HTML]{FFFFFF} 
		{\color[HTML]{000000} Baja}           & {\color[HTML]{000000} 25\% - 50\%}    \\ \hline
    \rowcolor[HTML]{FFFFFF} 
    {\color[HTML]{000000} Alta}           & {\color[HTML]{000000} 50\% - 75\%}    \\ \hline
    \rowcolor[HTML]{FFFFFF} 
		{\color[HTML]{000000} Muy alta}            & {\color[HTML]{000000} 75\% - 100\%} \\ \hline
	\end{tabular}
	\caption{Probabilidades de ocurrencia de los riesgos.}
	\label{tab:probabilidad_riesgo}
\end{table}



\begin{table}[htbp]
	\centering
	\begin{tabular}{|c|c|}
		\hline
		\rowcolor[HTML]{EFEFEF} 
    \textbf{Identificador del efecto} & \textbf{Demora estimada}      \\ \hline
		\rowcolor[HTML]{FFFFFF} 
    {\color[HTML]{000000} Despreciable}            & {\color[HTML]{000000} Menos de 8 horas de trabajo}    \\ \hline
    \rowcolor[HTML]{FFFFFF} 
  {\color[HTML]{000000} Moderado}            & {\color[HTML]{000000} Entre 8 y 24 horas de trabajo}    \\ \hline
		\rowcolor[HTML]{FFFFFF} 
		{\color[HTML]{000000} Grave}           & {\color[HTML]{000000} Demoras superiores a 24 horas}    \\ \hline
		\rowcolor[HTML]{FFFFFF} 
		{\color[HTML]{000000} Crítico}            & {\color[HTML]{000000} Se pone en juego la continuidad del proyecto} \\ \hline
	\end{tabular}
	\caption{Efectos de los riesgos.}
	\label{tab:seriedad_riesgo}
\end{table}



Luego, en la Tabla \ref{tab:matriz_criterio_riesgos} se observa la combinación entre la probabilidad de ocurrencia y el efecto producido por el riesgo. Esto origina el grado de importancia y seriedad del mismo, el cual mientras más grande sea, más perjudicial es para el proyecto.

\begin{table}[H]
	\centering
	\begin{tabular}{|
			>{\columncolor[HTML]{EFEFEF}}c |
			>{\columncolor[HTML]{FFFFFF}}c |
			>{\columncolor[HTML]{FFFFFF}}c |
			>{\columncolor[HTML]{FFFFFF}}c |
			>{\columncolor[HTML]{FFFFFF}}c |}
			\hline
			\textbf{\backslashbox {Proba-\\bilidad\\ de ocu-\\rrencia}{Efecto\\ del\\ riesgo}}                                & \cellcolor[HTML]{EFEFEF}\textbf{Despreciable} & \cellcolor[HTML]{EFEFEF}\textbf{Moderado} & \cellcolor[HTML]{EFEFEF}\textbf{Grave} & \cellcolor[HTML]{EFEFEF}\textbf{Crítico} \\ \hline
			{\color[HTML]{000000} \textbf{Muy Baja}} & {\color[HTML]{000000} Severidad 1}            & Severidad 2                               & Severidad 3                            & Severidad 4                               \\ \hline
			{\color[HTML]{000000} \textbf{Baja}}     & {\color[HTML]{000000} Severidad 2}            & Severidad 4                               & Severidad 6                            & Severidad 8                               \\ \hline
			{\color[HTML]{000000} \textbf{Alta}}     & {\color[HTML]{000000} Severidad 3}            & Severidad 6                               & Severidad 9                            & Severidad 12                              \\ \hline
			\textbf{Muy alta}                        & Severidad 4                                   & Severidad 8                               & Severidad 12                           & Severidad 16                              \\ \hline
		\end{tabular}
		\caption{Importancia de los riesgos en función de los efectos y las probabilidades de ocurrencia.}
		\label{tab:matriz_criterio_riesgos}
	\end{table}


Por último, detallamos en la tabla \ref{tab:analisis_riesgos} el análisis completo de cada uno de los riesgos identificados en la sección anterior. Como puede observarse, \textbf{R2} y \textbf{R3} son los de mayor repercusión en el proyecto.


\begin{table}[H]
	\centering
	\begin{tabular}{|c|c|c|c|}
		\hline
		\rowcolor[HTML]{EFEFEF} 
		\textbf{\begin{tabular}[c]{@{}c@{}}Identificador \\ del riesgo\end{tabular}} & \textbf{\begin{tabular}[c]{@{}c@{}}Probabilidad \\ de ocurrencia\end{tabular}} & \textbf{Efecto} & \textbf{\begin{tabular}[c]{@{}c@{}}Grado de\\ importancia\end{tabular}} \\ \hline
		\rowcolor[HTML]{FFFFFF} 
		{\color[HTML]{000000} \textbf{R1}}               & {\color[HTML]{000000} Muy baja} & {\color[HTML]{000000} Crítico}     & \cellcolor[HTML]{FFFE65}{\color[HTML]{000000} 4} \\ \hline
		\rowcolor[HTML]{FFFFFF} 
		{\color[HTML]{000000} \textbf{R2}}               & {\color[HTML]{000000} Alta}     & {\color[HTML]{000000} Grave}        & \cellcolor[HTML]{FE2E2E}{\color[HTML]{FFFFFF} 9} \\ \hline
		\rowcolor[HTML]{FFFFFF} 
		{\color[HTML]{000000} \textbf{R3}}               & {\color[HTML]{000000} Alta}     & {\color[HTML]{000000} Grave}        & \cellcolor[HTML]{FE2E2E}{\color[HTML]{FFFFFF} 9} \\ \hline
		\rowcolor[HTML]{FFFFFF} 
		{\color[HTML]{000000} \textbf{R4}}               & {\color[HTML]{000000} Baja}     & {\color[HTML]{000000} Despreciable} & \cellcolor[HTML]{89FF76}{\color[HTML]{000000} 2} \\ \hline
		\rowcolor[HTML]{FFFFFF} 
		{\color[HTML]{000000} \textbf{R5}}               & {\color[HTML]{000000} Baja}     & {\color[HTML]{000000} Despreciable} & \cellcolor[HTML]{89FF76}{\color[HTML]{000000} 2} \\ \hline
		\rowcolor[HTML]{FFFFFF} 
		{\color[HTML]{000000} \textbf{R6}}               & {\color[HTML]{000000} Alta}     & {\color[HTML]{000000} Moderado}     & \cellcolor[HTML]{FFB745}{\color[HTML]{000000} 6} \\ \hline
		\rowcolor[HTML]{FFFFFF} 
		{\color[HTML]{000000} \textbf{R7}}               & {\color[HTML]{000000} Alta}     & {\color[HTML]{000000} Moderado}     & \cellcolor[HTML]{FFB745}{\color[HTML]{000000} 6} \\ \hline
	\end{tabular}
	\caption{Análisis de los riesgos.}
	\label{tab:analisis_riesgos}
\end{table}



\subsubsection *{Estrategias de gestión de riesgos}

Por último y siguiendo los pasos que se detallan en \parencite{Ing_software}, se generó la Tabla \ref{tab:estrategias_riesgos} con el fin de solucionar los riesgos identificados y analizados en secciones anteriores.


\begin{table}[H]
	\centering
	\begin{tabular}{|c|l|l|}
		\hline
		\rowcolor[HTML]{EFEFEF} 
		\textbf{\begin{tabular}[c]{@{}c@{}}Identificador\\ del riesgo\end{tabular}} & \multicolumn{1}{c|}{\cellcolor[HTML]{EFEFEF}\textbf{Consecuencia}} & \multicolumn{1}{c|}{\cellcolor[HTML]{EFEFEF}\textbf{\begin{tabular}[c]{@{}c@{}}Estrategias de solución\\ (prevención, minimización\\ y contingencia)\end{tabular}}} \\ \hline
		\rowcolor[HTML]{FFFFFF} 
		{\color[HTML]{000000} \textbf{R1}}               & {\color[HTML]{000000} \begin{tabular}[c]{@{}l@{}}Los tiempos de desarrollo   \\ del proyecto se duplican\\ en el mejor de los casos.\end{tabular}} & {\color[HTML]{000000} \begin{tabular}[c]{@{}l@{}}Mantener una constante y\\ buena comunicación con el \\ equipo.\\ Acortar los requerimientos \\del proyecto.\end{tabular}} \\ \hline
		\rowcolor[HTML]{FFFFFF} 
		{\color[HTML]{000000} \textbf{R2}}               & {\color[HTML]{000000} \begin{tabular}[c]{@{}l@{}}Los tiempos de desarrollo   \\ del proyecto se extienden.\end{tabular}} & {\color[HTML]{000000} \begin{tabular}[c]{@{}l@{}}Estimar los tiempos de \\ desarrollo para el caso más \\ desfavorable.\end{tabular}} \\ \hline
		\rowcolor[HTML]{FFFFFF} 
		{\color[HTML]{000000} \textbf{R3}}               & {\color[HTML]{000000} \begin{tabular}[c]{@{}l@{}}Desperdicio de tiempos      \\ de desarrollo y extensión \\ de fecha de entrega del \\ proyecto.\end{tabular}} & {\color[HTML]{000000} \begin{tabular}[c]{@{}l@{}}Negociación, control y \\ revisión de los \\ requerimientos en etapas\\ iniciales del proyecto.\end{tabular}} \\ \hline
		\rowcolor[HTML]{FFFFFF} 
		{\color[HTML]{000000} \textbf{R4}}               & {\color[HTML]{000000} \begin{tabular}[c]{@{}l@{}}Problemas para cumplir      \\ con los requerimientos \\ funcionales de las\\ aplicaciones.\end{tabular}} & {\color[HTML]{000000} \begin{tabular}[c]{@{}l@{}}Utilizar versiones estables.\\ Actualizar versiones del\\ controlador y/o de sus\\ dependencias.\\ Registrar y documentar las\\ versiones utilizadas que no\\ generan problemas.\end{tabular}} \\ \hline
		\rowcolor[HTML]{FFFFFF} 
		{\color[HTML]{000000} \textbf{R5}}               & {\color[HTML]{000000} \begin{tabular}[c]{@{}l@{}}Problemas para cumplir      \\ con los requerimientos del\\ entorno de trabajo.\end{tabular}} & {\color[HTML]{000000} \begin{tabular}[c]{@{}l@{}}Utilizar versiones estables.\\ Actualizar la versión del\\ OVS.\\ Registrar y documentar las\\ versiones utilizadas que no\\ generan problemas.\\ Emplear, en caso necesario, \\ otros productos como, por \\ ejemplo, Indigo Virtual \\ Switch.\end{tabular}} \\ \hline
		\rowcolor[HTML]{FFFFFF} 
		{\color[HTML]{000000} \textbf{R6}}               & {\color[HTML]{000000} \begin{tabular}[c]{@{}l@{}}Problemas para detectar los \\ ataques.\end{tabular}} & {\color[HTML]{000000} \begin{tabular}[c]{@{}l@{}}Estudiar más a fondo los \\ manuales de Snort.\\ Utilizar otros sistemas de\\ detección de intrusiones\\ como, por ejemplo, Suricata.\end{tabular}} \\ \hline
		\rowcolor[HTML]{FFFFFF} 
		{\color[HTML]{000000} \textbf{R7}}               & {\color[HTML]{000000} \begin{tabular}[c]{@{}l@{}}Problemas para generar los  \\ ataques.\end{tabular}} & {\color[HTML]{000000} \begin{tabular}[c]{@{}l@{}}Estudiar a fondo las \\ herramientas. \\ Observar distintos tutoriales.\\ Emplear otras herramientas\\ disponibles en el mercado, \\ como las que ofrece el \\ \textit{framework} Metasploit.\end{tabular}} \\ \hline
	\end{tabular}
	\caption{Estrategias de solución para los distintos riesgos.}
	\label{tab:estrategias_riesgos}
\end{table}


% Chapter Template
% cSpell:words parencite onfwhitepaper includegraphics resizebox sdncomponents
% linewidth comparqui redireccionar enrutamiento subfigure toposdn Nicira toposinsdn
\chapter{Aplicación de detección de anomalías} % Main chapter title 

\label{sec:Chapter4} % Change X to a consecutive number; for referencing this
                     % chapter elsewhere, use \ref{ChapterX}

En el presente capítulo se explicará el diseño y la implementación de la
aplicación que se situará en el núcleo de ONOS (ver Figura
\ref{fig:arquitectura_onos}) y que estará encargada de detectar comportamientos
sospechosos en la red y de disminuir la sobrecarga de tráfico en los sistemas de
detección de intrusos.

Para ello, en primer lugar, se describen los motivos que desencadenan la
necesidad del desarrollo de esta aplicación. En segundo lugar, se detallan los
requerimientos de la misma. Y en tercer lugar, el principio de funcionamiento de
dicha aplicación se desglosa y se explica etapa por etapa, realizando el
correcto mapeo de las explicaciones con los requerimientos correspondientes. Por
último, se detalla la implementación llevada a cabo, en donde se hace foco en
las funciones más importantes.

\section{Motivación} \label{sec:motivacion_app_4}

Por lo discutido en la Sección \ref{sec:state_art}, una parte de los objetivos
principales del proyecto consiste en reducir el procesamiento de los
dispositivos de detección.

Uno de los factores más importantes es la cantidad de tráfico que analizan estos
dispositivos, debido a que existen reglas que no solo inspeccionan las cabeceras
de los paquetes sino también la carga útil de los mismos. De este modo, cuanto
mayor sea el tráfico mayor es el procesamiento que éstos realizan. A su vez,
dicha cantidad de tráfico es dependiente, entre otras cosas, de la ubicación de
los IDSs dentro de la topología.


\paragraph{Enfoques de emplazamiento del IDS.}
De acuerdo a \parencite{ddos_kumar}, cuanto más cerca esté el detector a la
posición del servicio objetivo, mayor será la cantidad de tráfico que analiza.
Tal enfoque es capaz de proporcionar una mejor precisión de detección, debido a
que posee un gran dominio para analizar el tráfico. Sin embargo, este enfoque
solamente detecta el ataque después de que llega a la víctima. Además, si bien
se puede cortar el tráfico malicioso en el enrutador más cercano a ella, el
legítimo aún no puede viajar a través de la red y llegar a su destino, debido a
que el ancho de banda está afectado. Por otro lado, si el emplazamiento es más
cercano a los dispositivos de borde directamente conectados a los \textit{hosts}
maliciosos, se puede evitar la congestión no solo del lado de la víctima, sino
también de la red, sin afectar a los demás usuarios legítimos. Sin embargo,
durante un ataque, generalmente el tráfico no proviene de una sola fuente, sino
de manera distribuida por múltiples puntos de acceso. Esto dificulta la
detección debido a la ubicación de los detectores de intrusos. Para resolver los
problemas de precisión a la hora de percibir dichos ataques y evitar el consumo
de ancho de banda por parte de éstos, se encuentra la defensa de red que ubica a
los IDSs en una zona intermedia \parencite{ddos_kumar}. Este último es el
enfoque que se utilizará en el presente proyecto.

\paragraph{Problemas de procesamiento frente a la escalabilidad de la red.} A
medida que la red escala se imposibilita que los sistemas de detección de
intrusiones analicen todo el tráfico que atraviesa a ciertos dispositivos de
red, independientemente de su ubicación. Para evitar la sobrecarga, se puede
plantear que los análisis en los IDSs sean temporales y solamente en aquellos
casos en donde se detecten comportamientos sospechosos, en base a uno esperado.
Para esto último se requiere un análisis estadístico llevado a cabo por el
controlador, el cual posee información global de la red y decide si existe o no
una anomalía.

\paragraph{} En base a lo descrito anteriormente y a lo mencionado en la Sección
\ref{sec:segnetsdn} surgen los requerimientos explicados en la siguiente
sección.

%---------------------bitext comileArquitectura del protocolo--------------------
%	SECTION 1
%---------------------------------------------------------------------------------
\section{Requerimientos} 

Los requerimientos funcionales de la aplicación son los siguientes:

\paragraph{RF01.} La aplicación debe recopilar métricas cada 10 segundos del
tráfico entrante a la zona intermedia (dispositivos de distribución) de la red.
\paragraph{RF02.} La aplicación debe almacenar las métricas totales por día y
debe mantener una ventana con los últimos 7 días.
\paragraph{RF03.} La aplicación, usando dichas métricas, debe detectar
comportamientos sospechosos en base a uno esperado.
\paragraph{RF04.} La aplicación debe identificar a los dispositivos de la
subcapa de acceso que presenten un comportamiento sospechoso.
\paragraph{RF05.} La aplicación, ante la presencia de tráfico potencialmente
malicioso, debe replicarlo temporalmente hacia los IDSs más cercanos a los
dispositivos detectados en RF04, al menos el tiempo mínimo necesario para que
Snort pueda reconocer si se está en presencia de un ataque o no.

\section{Principios de funcionamiento}

A continuación, se desarrolla una introducción a la lógica detrás de la
aplicación de detección de anomalías, para facilitar la futura comprensión de la
estructura y el funcionamiento de la misma. De manera resumida, el
comportamiento de esta aplicación se puede observar en el diagrama de la Figura
\ref{fig:diagram1}.

\begin{figure}[H]
	\centering 
	\includegraphics[scale=0.7]{activity_1}
	\caption{Diagrama de actividades de la aplicación.}
	\label{fig:diagram1}
\end{figure}

\subsection{Recolección de métricas}

De acuerdo a lo descrito en la Sección \ref{sec:control_layer}, la interfaz
SOUTHBOUND permite a los dispositivos informar el estado de los mismos usando
OpenFlow. Esto le posibilita al controlador obtener métricas de los puertos,
tales como la cantidad de paquetes y la cantidad de bytes que los atraviesan,
mediante \textbf{contadores} en el plano de datos \parencite{opf151}.

Por otro lado, basándonos en el sistema propuesto por \parencite{estado_arte_2}
para la detección de ataques distribuidos, se plantea uno propio aplicando a una
topología de red ISP la gestión SDN. Para ello, es necesario identificar los
dispositivos (de la red ISP) que proporcionan las métricas y, en conjunto con el
enfoque de emplazamiento de los sistemas de detección explicado al inicio de
esta sección, se propone la recolección de dichas métricas desde los conmutadores
identificados como \textit{distribution}, tal como se muestra en la Figura
\ref{fig:ISP}. Estos dispositivos se encuentran en una ubicación intermedia al
mantener conexiones tanto con las capas de acceso como con las capas del núcleo.

Como se mencionó anteriormente, gracias a OpenFlow se pueden obtener desde el
plano de datos las métricas de todos los dispositivos y de todos sus puertos
(\textbf{RF01}). No obstante, todas ellas no son necesarias para el análisis
posterior. Sólo son indispensables aquellas pertenecientes a los puertos de los
conmutadores SDN de distribución cuyos enlaces se conecten a los dispositivos de
borde de la subcapa de acceso. Así sólo se deben obtener métricas del tráfico
entrante a la red, tal como se observa en la Figura \ref{fig:trafico_entrante}.


\begin{figure}[H]
	\centering 
	\includegraphics[scale=1.5]{intraffic_1}
	\caption{Tráfico entrante desde los dispositivos de acceso a los de
    distribución.}
	\label{fig:trafico_entrante}
\end{figure}


\subsection{Detección de  anomalías}

La detección de anomalías se resuelve realizando un análisis estadístico basado
en la prueba de bondad de ajuste chi-cuadrado. Ésta se utiliza para comprobar si
una muestra de datos sigue una distribución determinada. La manera de definirlo
es realizando una prueba de hipótesis para verificar si dicha muestra difiere de
una población de datos esperada. La forma de cuantificar las diferencias es
usando (\ref{eq:chi_cuadrado}).

\begin{equation}\label{eq:chi_cuadrado}
\chi^2={\sum_{i=0}^{k} \left [ \frac{(O_{i} - E_{i})^2}{E_{i}} \right ]}
\end{equation}\\

En donde \(O_{i}\) representa los datos observados y \(E_{i}\) es el valor
esperado para cada observación.

Para nuestro proyecto estas variables están definidas por la cantidad de
paquetes y la cantidad de bytes pertenecientes al tráfico que ingresa a la red.
El primero se utiliza para saber si un ataque afecta a los recursos de los
dispositivos de la red, mientras que el segundo se usa para determinar si existe
un impacto en el ancho de banda \parencite{estado_arte_2}. Los datos anteriores
se comparan con valores esperados. Si éstos difieren estadísticamente, se
considera un tráfico con un comportamiento sospechoso (\textbf{RF03}).

Por otro lado, los valores esperados no pueden ser estáticos. De ser así, no se
considerarían las variaciones del tráfico legítimo a largo plazo y se podría
estar etiquetando a dicho tráfico legítimo como sospechoso, dando lugar a las
detecciones conocidas como \textbf{falsos positivos}. Esto explica el por qué
del requerimiento \textbf{RF02}, ya que es necesario conocer la cantidad de
tráfico que circulaba en la red anteriormente.

Finalmente, una vez considerado el tráfico sospechoso, el diseño del
\textbf{RF04} se basa en identificar los dispositivos de distribución que
superan una cierta diferencia estadística con respecto a sus valores esperados.
Luego, se deben determinar cuáles de los conmutadores de la subcapa de acceso
conectados a dichos \textit{distribution} presentan un porcentaje de tráfico que
compromete a la red.

\subsection{Duplicación del tráfico}

Una vez que se clasifica el tráfico entrante como sospechoso, es necesario
enviarlo a un IDS para su inspección.

Una primera solución sería desviar el tráfico completamente hacia el detector de
ataques, pero como el marcado es sobre los puertos del dispositivo
\textit{distribution}, el tráfico proveniente de éstos también incluye tráfico
legítimo. En este caso, si se realiza el completo redireccionamiento, se estaría
cortando el servicio a clientes legítimos.

Otra solución es la duplicación temporal del tráfico, usando las directivas de
la interfaz SOUTHBOUND para aplicar reglas con el fin de enviar este tráfico
tanto a su destino original como al IDS, tal como se observa en la Figura
\ref{fig:dupl_trafico}. Finalmente, ésta es la solución elegida para el
\textbf{RF05}.

\begin{figure}[H]
	\centering 
	\includegraphics[scale=1.5]{intraffic_2}
	\caption{Duplicación de tráfico sospechoso.}
	\label{fig:dupl_trafico}
\end{figure}

Un aspecto importante a tener en cuenta es que si se duplica todo el tráfico
entrante a un dispositivo \textit{distribution}, se estaría enviando información
innecesaria al sistema de detección de intrusos. Es por ello que solo se debe
reproducir aquel tráfico proveniente de los puertos con comportamientos que
difieren en una cierta proporción a lo esperado y cuyos enlaces se conecten a
conmutadores de la subcapa de acceso, tal y como se mencionó anteriormente.

\section {Detalles de la implementación}

En esta sección se hace hincapié en los detalles de la implementación haciendo
foco en las funciones más importantes con el fin de cumplir con los
requerimientos funcionales. Para ello se tiene en cuenta y se adapta lo
propuesto en \parencite{estado_arte_2}. A su vez, una visión estática y global
de lo efectuado se encuentra en las Figuras \ref{fig:diagrama_clases_1} y
\ref{fig:diagrama_clases_1_2}.

\begin{figure}[H]
	\centering 
	\includegraphics[width=0.9\textwidth]{Clases1}
	\caption{Diagrama de clases para la detección de anomalías.}
	\label{fig:diagrama_clases_1}
\end{figure}

Para la recolección de las métricas (\textbf{RF01}), el controlador ONOS
mediante la capa Proveedores (ver Figura \ref{fig:arquitectura_onos}) ofrece
estadísticas de los puertos de los conmutadores SDN mediante el servicio
\verb|deviceService|. Éste brinda dos métodos importantes:

\begin{itemize}
\item \verb|getPortDeltaStatistics (DeviceId deviceId)|. Obtiene una lista con
  todas las estadísticas de un dispositivo en un intervalo de tiempo, cuyo valor
  por defecto es de 10 segundos.
\item \verb|getPortStatistics (DeviceId deviceId)|. Obtiene una lista con el
  total de las estadísticas de cada puerto de un dispositivo desde que empieza a
  enviar y recibir tráfico.
\end{itemize}

El primer método permite obtener métricas de los dispositivos cada 10 segundos.
El segundo facilita el diseño e implementación del \textbf{RF02}, ya que basta
con almacenar el valor dado por este método al inicio y al final del día para
luego realizar una diferencia. Luego se conforma una cola con estas diferencias
en donde también se almacenan los valores obtenidos de los 7 días anteriores.

Para la implementación del \textbf{RF03}, primero se obtienen los valores
estadísticos esperados bajo circunstancias normales para el tráfico entrante a
un dispositivo \textit{distribution} \(i\), desde la subcapa de acceso, para un
intervalo de tiempo de 10 segundos (\textbf{\(G_i\)}), a partir de las colas
mencionadas anteriormente. Esto se realiza tanto para la cantidad de paquetes
como para la cantidad de bytes, tal y como se observa en (\ref{eq:gicount}) y
(\ref{eq:gisize}).

\begin{equation}
	cant\_paquetes\_semana_i={\sum_{j=1}^7 cant\_paquetes\_dia_{j}}
\end{equation}

\begin{equation}\label{eq:gicount}
    G_{i\_paquetes}={\frac{cant\_paquetes\_semana_i}{\frac{7*24*60*60}{10}}}
\end{equation}

\begin{equation}
	cant\_bytes\_semana_i={\sum_{j=1}^7 cant\_bytes\_dia_{j}}
\end{equation}

\begin{equation}\label{eq:gisize}
	G_{i\_bytes}={\frac{cant\_bytes\_semana_i}{cant\_intervalos}}
\end{equation}

\begin{equation}\label{eq:intervalo}
  cant\_intervalos=\frac{7\ dias * 24\frac{horas}{dia} * 60\frac{minutos}{hora} * 60\frac{segundos}{minuto}}{10 \ segundos}
\end{equation} \\

Luego se ejecuta la prueba de bondad de ajuste chi-cuadrado, la cual está dada
por (\ref{eq:Xcount}) y (\ref{eq:Scount}) y se calcula cada vez que se obtienen
métricas nuevas cada 10 segundos.

\begin{equation}\label{eq:Xcount}
	\chi^2 = {\sum_{i=1}^{n}   \frac{N_{i\_paquetes}-G_{i\_paquetes}}{G_{i\_paquetes}}}
\end{equation}

\begin{equation}\label{eq:Scount}
	\chi^2 = {\sum_{i=1}^{n}   \frac{N_{i\_bytes}-G_{i\_bytes}}{G_{i\_bytes}}}
\end{equation}\\


En donde \(N_{i}\) son los datos recolectados de los últimos 10 segundos por el
dispositivo \textit{distribution}. Luego, los resultados se comparan con los
valores de la tabla de distribución chi cuadrado usando \textbf{\(n\)} grados de
libertad, que representan la cantidad de estos dispositivos de distribución, y
un nivel de significancia (denotado como \(\alpha\)) de 0.05. (Un nivel de
significancia de 0.05 indica un riesgo del 5\% de concluir que existe tráfico
sospechoso cuando no lo hay). Si alguna de las comparaciones arroja que el valor
de la tabla es menor, se considera que en la red existe un potencial ataque de
denegación de servicio. El comportamiento del algoritmo se puede observar en el
diagrama de la Figura \ref{fig:diagram2}.

\begin{figure}[H]
	\centering 
	\includegraphics[width=\textwidth]{sequence_1}
	\caption{Diagrama de secuencia para la detección de anomalías.}
	\label{fig:diagram2}
\end{figure}

Una vez detectada la anomalía en la red se deben determinar los dispositivos de
borde comprometidos (\textbf{RF04}). Para ello primero es necesario encontrar
los dispositivos de distribución que superen un umbral en cuanto a la diferencia
entre el comportamiento observado y el esperado correspondiente. Esto se hace
mediante (\ref{eq:ineq_1}) y (\ref{eq:ineq_2}).

\begin{equation}\label{eq:ineq_1}
	{\frac{(N_{i\_paquetes}-G_{i\_paquetes})^2}{G_{i\_paquetes}}} > umbral_{paquetes}
\end{equation}

\begin{equation}\label{eq:ineq_2}
	{\frac{(N_{i\_bytes}-G_{i\_bytes})^2}{G_{i\_bytes}}} > umbral_{bytes}
\end{equation}\\

En donde el valor del umbral se toma igual a 3.5, de acuerdo a
\parencite{ddos_amir}, con el objetivo de tener un balance entre la cantidad de
falsos positivos y falsos negativos en la detección.

Determinados los dispositivos de distribución se deben buscar cuáles de sus
\(k\) enlaces entrantes desde la subcapa de acceso presentan un nivel de tráfico
mayor o igual al 80\% \parencite{estado_arte_2} en relación a todo el que
ingresa a dicho dispositivo, tal y como se muestra en (\ref{eq:ineq_3}) y
(\ref{eq:ineq_4}). En caso de que ningún enlace cumpla con alguna de estas
condiciones planteadas en las inecuaciones, se los considera a todos
sospechosos.

\begin{equation}\label{eq:ineq_3}
	{\frac{cant\_paquetes\_ultimos\_10\_segundos\_enlace_i}{\sum_{i=1}^{k}
      cant\_paquetes\_ultimos\_10\_segundos\_enlace_i}} \geq 0.8
\end{equation}

\begin{equation}\label{eq:ineq_4}
  {\frac{cant\_bytes\_ultimos\_10\_segundos\_enlace_i}{\sum_{i=1}^{k}
      cant\_bytes\_ultimos\_10\_segundos\_enlace_i}} \geq 0.8
\end{equation}\\

Luego, a partir de los mencionados enlaces se obtienen los dispositivos de borde
solicitados por \textbf{RF04}. Un diagrama para este requerimiento se puede
observar en la Figura \ref{fig:diagram2.1}.

\begin{figure}[H]
	\centering 
	\includegraphics[width=\textwidth]{sequence_2_1}
	\caption{Diagrama de secuencia para RF04.}
	\label{fig:diagram2.1}
\end{figure}

Un aspecto importante a aclarar es que los términos negativos no son tenidos en
cuenta en (\ref{eq:Xcount}), (\ref{eq:Scount}), (\ref{eq:ineq_1}) y
(\ref{eq:ineq_2}), ya que no se busca detectar anomalías ante niveles de
tráfico por debajo de lo esperado.

\begin{figure}[H]
	\centering 
	\includegraphics[width=\textwidth]{Clases2}
	\caption{Diagrama de clases para la detección de anomalías.}
	\label{fig:diagrama_clases_1_2}
\end{figure}


Por otra parte, para implementar \textbf{RF05} se usa el servicio de ONOS llamado
\verb|IntentService|, el cual permite conectar dos puntos en la red a alto
nivel. Cuando llega un paquete nuevo a la red, si las tablas OpenFlow (ver
Figura \ref{fig:openflow_2}) en los dispositivos no tienen reglas establecidas
para tomar acciones sobre el mismo, se envían al plano de control. El
controlador, mediante una aplicación de enrutamiento que usa el servicio antes
mencionado, se encarga de instalar las reglas OpenFlow en los dispositivos
intermedios estableciendo la comunicación entre los puntos.

La clase \verb|SinglePointToMultiPointIntent| permite conectar un punto
de conexión de entrada a dos o más puntos de salida. Llegado el momento de
realizar la duplicación de tráfico se obtienen todas las conexiones hechas por
el servicio \verb|IntentService| y se agrega como nuevo destino el IDS, a
aquellas que provengan del dispositivo de borde detectado como sospechoso,
tal y como se mencionó en párrafos anteriores. Así se logra el comportamiento
deseado de la Figura \ref{fig:dupl_trafico}.

En cuanto al aspecto del \textbf{RF05} que solicita el envío del tráfico al IDS
más cercano, previamente se debe poseer una lista con todos los detectores de la
red. Luego, ONOS provee el servicio \verb|topologyService| para obtener todos
los caminos entre dos puntos de conexión. Se usa este servicio para adquirir las
rutas más cortas entre los dispositivos de la subcapa de acceso devueltos por el
\textbf{RF04} y los IDSs más cercanos.


% Chapter Template
% cSpell:words parencite onfwhitepaper includegraphics resizebox sdncomponents
% linewidth comparqui redireccionar enrutamiento subfigure toposdn Nicira toposinsdn
\chapter{Aplicación de filtrado del tráfico} % Main chapter title 

\label{sec:Chapter5} % Change X to a consecutive number; for referencing this
                     % chapter elsewhere, use \ref{ChapterX}

En el presente capítulo se explicará el diseño y la implementación de la
aplicación ubicada en el núcleo del controlador y encargada de mitigar y filtrar
el tráfico que proviene de ataques DoS y DDoS.

Para ello se describe la motivación de la aplicación, sus requerimientos, las
etapas constituyentes y la mencionada implementación de sus funciones más
importantes.

\section {Motivación}

Una vez realizada la duplicación de tráfico mencionada en el Capítulo
\ref{sec:Chapter4}, los dispositivos IDSs deben realizar la inspección de dicho
tráfico en búsqueda de aspectos maliciosos, en base a reglas predefinidas. De
acuerdo a la Sección \ref{sec:herramienta_snort}, la herramienta encargada de
realizar esta tarea es Snort.

Teniendo en cuenta su descripción, en el caso de la detección de tráfico
sospechoso el mismo puede enviar las alertas a través de \textit{sockets}. Una
limitación que se presenta es que éstos son de tipo \textit{Unix}, por lo tanto
para poder enviarlas al controlador se debió implementar un retransmisor a
\textit{sockets} TCP.

La aplicación que se describe en este capítulo es la que recibe estas alertas en
el controlador SDN y, en base a ellas, decide la acción a tomar, como por ejemplo,
filtrar el tráfico. Una vez definida esta acción, es necesario determinar los
dispositivos en donde se instalarán las reglas correspondientes, pudiendo ser el
más cercano a la fuente si ésta puede ser determinada o, en caso contrario, en
todos los dispositivos de borde o de la red, dependiendo la potencia del ataque.

En base a lo descrito anteriormente surgen los requerimientos detallados a
continuación.

%-----------------------------bitext comileArquitectura del protocolo--------------------
%	SECTION 1
%----------------------------------------------------------------------------------------
\section{Requerimientos} \label{sec:reqs_app_2}

Los requerimientos funcionales de la aplicación son los siguientes:

\paragraph{RF06.} La aplicación en el controlador ONOS debe aceptar alertas a
través de \textit{sockets}, provenientes de múltiples IDSs. Además, la
aplicación debe interactuar por lo menos con 4 IDSs al mismo tiempo.

\paragraph{RF07.} La aplicación debe ir procesando y descomponiendo las alertas
en sus partes constituyentes a medida que van llegando desde los
\textit{sockets}. Además, debe reconocer dichas alertas enviadas por Snort en
función del identificador \textit{sid} (ver Apéndice \ref{AppendixC} para
más información) a partir de una base de datos interna de la aplicación que
señala aquellas que son válidas. Para las que no lo sean, la aplicación
solamente debe registrar el mensaje que contiene de manera informativa.

\paragraph{RF08.} La aplicación, al recibir las alertas correspondientes, debe
detener el tráfico de los ataques DoS o DDoS de tipo \textit{\textbf{TCP SYN
    flood}}, \textit{\textbf{TCP PUSH-ACK flood}},\textit{ \textbf{TCP SYN-FIN
    flood}}, \textit{\textbf{TCP FIN flood}} y \textit{\textbf{TCP RESET
    flood}}, instalando las reglas OpenFlow del tipo \textit{drop} en los
dispositivos de borde desde donde provenga el respectivo ataque dirigido a
cualquier cliente del ISP.

\paragraph{RF09.} La aplicación, al recibir las alertas correspondientes, debe
detener el tráfico de los ataques DoS o DDoS de tipo \textit{\textbf{UDP
    flood}}, instalando las reglas OpenFlow de \textit{drop} en los dispositivos
de borde desde donde provenga el ataque dirigido a cualquier cliente del ISP.

\paragraph{RF10.} La aplicación, al recibir las alertas correspondientes, debe
detener el tráfico de los ataques DDoS de tipo \textbf{\textit{Smurf}} dirigidos
a cualquier cliente del ISP. Debe además instalar las reglas OpenFlow de
\textit{drop} en todos los dispositivos.

\paragraph{RF11.} Las reglas de \textit{drop} que se deben instalar según el
conjunto de requerimientos que va desde \textbf{RF08} hasta \textbf{RF10} deben
ser eliminadas de los dispositivos a los 30 segundos de inactividad de las
mismas, ya que se considera el ataque como mitigado. La inactividad hace
referencia al hecho de que en un intervalo de tiempo dado, ningún paquete de
ningún flujo genera una coincidencia (\textit{match}) con dicha regla.

\section{Principios de funcionamiento}

A continuación, el comportamiento de la aplicación de filtrado y su relación con
la Sección \ref{sec:Chapter4} se puede observar en el diagrama de la Figura
\ref{fig:diagram_flujo_filtrado}.

\begin{figure}[H]
	\centering 
	\includegraphics[width=0.95\textwidth]{activity_2}
	\caption{Diagrama de actividades de la aplicación de filtrado de tráfico.}
	\label{fig:diagram_flujo_filtrado}
\end{figure}


\subsection {Reconocimiento de la alerta}

En primer lugar, para resolver los requerimientos antes mencionados Snort debe
enviar las alertas a la aplicación que se describe en este capítulo, la cual
presenta un \textit{socket} servidor que escucha peticiones de apertura de
sesión por parte de los distintos IDSs. A medida que la aplicación acepta
conexiones, genera \textit{sockets} independientes para poder recibir las
alertas de los distintos detectores (\textbf{RF06}). Por ende, para este
requerimiento se necesita de un entorno multihilo.

Por otra parte, para que dicha aplicación pueda reconocer las alertas de Snort
(\textbf{RF07}), es imprescindible conocer la estructura de las mismas. Esta
estructura es conocida como \verb|Alertpkt Struct| y es dependiente de la
versión del detector. En este proyecto se emplea la 2.9.11 y su formato se puede
ver en el Código Fuente \ref{lst:alert_struct} \parencite{alerpkt}.

\begin{lstlisting} [label=lst:alert_struct, caption=Estructura Alertpkt, captionpos=b, language=C]
  #define ALERTMSG_LENGTH 256
  #define SNAPLEN         65535 
  typedef struct _Alertpkt
  {
    uint8_t alertmsg[ALERTMSG_LENGTH];
    struct pcap_pkthdr32 pkth;
    uint32_t dlthdr;    /* Desplazamiento cabecera de capa de enlace */
    uint32_t nethdr;    /* Desplazamiento cabecera de capa de red */
    uint32_t transhdr;  /* Desplazamiento cabecera de capa de transporte */
    uint32_t data;
    uint32_t val;
    uint8_t pkt[SNAPLEN];
    Event event; /* Total bytes Event: 36 */
  } Alertpkt;
\end{lstlisting}

De acuerdo al Código Fuente \ref{lst:alert_struct}, uno de los campos más
destacados es \verb|alertmsg|, el cual representa el mensaje de la alerta. Por
otro lado, \verb|pkt| contiene el paquete que produjo el suceso, es decir, aquel
que fue interceptado por el IDS y dio origen a la mencionada alerta. Luego, la
estructura \verb|Event| (ver Código Fuente \ref{lst:event_struct}), contiene
toda la información relevante sobre el evento, tales como la identificación de
la regla de Snort que lo detectó (\verb|sid| ó \verb|sig_id|), la clasificación
(\verb|classification|), revisión (\verb|sig_rev|) y prioridad (\verb|priority|)
de dicha regla, etc. Todos las demás varaibles de ambas estructuras no presentan
relevancia para este proyecto.

\begin{lstlisting} [label=lst:event_struct, caption=Estructura Event, captionpos=b, language=C]
  typedef struct _Event
  {
    uint32_t sig_generator;   /* ID del IDS */
    uint32_t sig_id;          /* ID de la regla detectora */
    uint32_t sig_rev;         /* Revision de la regla detectora */
    uint32_t classification;  /* Clasificacion del evento */
    uint32_t priority;        /* Prioridad del evento */
    uint32_t event_id;        /* ID del evento */
    uint32_t event_reference; /* Referencia a otros eventos */
    struct sf_timeval32 ref_time; /* Tiempo de referencia para el evento */
  } Event;
  typedef struct sf_timeval32
  {
    uint32_t tv_sec;      /* Segundos */
    uint32_t tv_usec;     /* Microsegundos */
  };
\end{lstlisting}

Además de todo lo explicado hasta el momento, una vez que se obtiene el
\verb|sig_id| de la alerta, se lo debe comparar con los valores considerados
como válidos por parte de la aplicación. La Sección \ref{sec:install_rules} se
llevará a cabo solamente si dichas alertas son válidas, de lo contrario no se
realizará nada más que la impresión del contenido del campo \verb|alertmsg|.

\subsection{Instalación de reglas} \label {sec:install_rules}

Con todo lo descrito anteriormente es posible procesar una alerta en el
controlador. Luego, mediante OpenFlow se definen reglas de filtrado utilizando
la acción de desechar el paquete (\textit{drop}) (ver Figura
\ref{fig:openflow_2}).

Una vez construidas estas reglas, la pregunta es en dónde se instalan. Esto
depende si el origen del ataque se puede identificar dentro de la red o no. Si
resulta el primer caso, entonces se aplican en las tablas de flujo del
dispositivo más cercano a la fuente, tal como se observa en la Figura
\ref{fig:filtrado_trafico_1}. En caso contrario se aplican en todos los
dispositivos de borde, tal y como se observa en la Figura
\ref{fig:filtrado_trafico_2}.

\begin{figure}[H]
	\centering 
	\begin{subfigure}[b]{0.49\textwidth}
		\centering
		\includegraphics[width=\textwidth]{aislar_1}
		\caption{Origen del ataque identificado.}
		\label{fig:filtrado_trafico_1}
	\end{subfigure}
	\begin{subfigure}[b]{0.49\textwidth}
		\centering
		\includegraphics[width=\textwidth]{aislar_2}
		\caption{Origen del ataque sin identificar.}
		\label{fig:filtrado_trafico_2}
	\end{subfigure}
	\caption{Filtrado del tráfico.}
	\label{fig:filtrado_trafico_1_2}
\end{figure}

Los requerimientos \textbf{RF08}, \textbf{RF09} y \textbf{RF10} realizan una
separación y clasificación en cuanto a la mitigación de los ataques debido a que
difieren en la regla de \textit{drop} a instalar en los dispositivos. El primero
y el segundo tratan sobre las inundaciones TCP y UDP, respectivamente, mientras
que el tercero sobre Smurf. Todos los ataques que generen una falsificación de
la dirección IP de origen dificultan la localización de la fuente, como por
ejemplo, el mencionado Smurf. Para detener este potente ataque se hace
indispensable instalar las reglas no solo en los de borde, sino también en todos
los dispositivos afectados de la red.

Por otro lado, estas reglas deben ser temporales debido a los recursos limitados
en almacenamiento que presentan tales dispositivos. Es por ello que el
requerimiento \textbf{RF11} expresa que se eliminarán aquellas reglas en
los conmutadores en donde cumplan con 30 segundos de inactividad.

\section {Detalles de la implementación}

Para la implementación, tal como se mencionó anteriormente, es necesario
configurar Snort para enviar las alertas a través de un \textit{socket}. Para
lograr esto, se debe modificar su archivo de configuración \verb|snort.conf|. El
mismo habilita el envío a través de \textit{UNIX sockets}, tal como lo
especifica la documentación \parencite{snort_manual}, por lo que se debió
implementar en Python un nodo retransmisor que utilice \textit{sockets TCP} para
que el controlador pueda recibir dichas alertas. (Ver Figura
\ref{fig:bloques_relay}).

\begin{figure}[htbp]
	\centering 
	\includegraphics[width=0.75\textwidth]{unix_sock}
	\caption{Comunicación Snort-ONOS.}
	\label{fig:bloques_relay}
\end{figure}

Abriendo un puerto TCP en el controlador, el siguiente paso es reconocer el
mensaje enviado por Snort siguiendo el formato de las estructuras
\ref{lst:alert_struct} y \ref{lst:event_struct}. Utilizando la clase
\verb|DataInputStream| de Java 8 se recibe la información y se la almacena en un
objeto para su posterior uso. Antes de seguir con la explicación, es importante
mencionar que una visión estática y global de lo explicado en este capítulo se
observa en la Figura \ref{fig:diagrama_de_clases_2}.

\begin{figure}[th]
	\centering 
	\includegraphics[width=\textwidth]{Clases3}
	\caption{Diagrama de clases de la aplicación de filtrado del tráfico.}
	\label{fig:diagrama_de_clases_2}
\end{figure}

Para el armado del campo \verb|pkt| de la estructura de la alerta (ver Código Fuente
\ref{lst:alert_struct}), se utiliza el paquete \verb|org.onlab.packet|, el cual
contiene clases por cada protocolo de la pila TCP/IP, tales como IP, TCP, ICMP,
etc. Esto es de utilidad ya que permite almacenar la información completa en
objetos, ayudando a resolver el \textbf{RF07}.

Por otra parte, para la implementación de \textbf{RF06} se sigue el patrón de
cliente-servidor, creando un hilo nuevo por cada conexión aceptada o IDS.
Además, utilizando el servicio \verb|ExecutorService| y la clase
\verb|Executors| de Java 8, se puede limitar la cantidad de hilos que puede
crear un proceso.

Una vez reconocida la alerta, en base al \verb|sig_id| ó \verb|sid| de la regla
que la generó, se debe decidir si la misma forma o no parte de un ataque. En el
caso de tratarse de un ataque, se la debe clasificar en función de si el ataque
que representa pertenece al requerimiento \textbf{RF08}, \textbf{RF09} ó
\textbf{RF10}.

Si la alerta no es válida o no forma parte de un ataque, solamente se registra
un mensaje en el \verb|logger| del controlador. En caso contrario, se debe
mitigar dicho ataque. Para aplicar una regla de filtrado a través de OpenFlow,
se hace uso de la interfaz \verb|TrafficSelector|. Esta API de ONOS permite
definir un criterio de selección para un tipo particular de tráfico, usando los
bits de las cabeceras de Ethernet, IPv4, TCP, UDP, etc. Por otro lado, para
definir acciones sobre un paquete de dicho tráfico seleccionado se utiliza la
interfaz \verb|TrafficTreatment|.

De este modo, se conforman las reglas OpenFlow como se definió en la Sección
\ref{sec:opflow} con los campos \verb|match| (\verb|TrafficSelector|) y
\verb|action| (\verb|TrafficTreatment|). Ejemplos de la implementación de éstos
se puede observar en los Bloques de Código \ref{lst:traffic_selector} y
\ref{lst:traffic_treatment}, respectivamente.\\ \\
% En este caso, como se intenta implementar un \textit{firewall}, las acciones
% son siempre de \textit{drop}. \\

\begin{lstlisting} [label=lst:traffic_selector, caption= Ejemplo de un selector
  de tráfico., captionpos=b, language= Java]
/**
 * Selector de trafico TCP en IPv4 que se emite desde la direccion de
 * origen DireccionIpOrigen con destino hacia DireccionIpDestino 
 * al puerto port.
*/
selector = DefaultTrafficSelector.builder()
                   .matchEthType    (Ethernet.TYPE_IPV4)
                   .matchIPSrc      (DireccionIpOrigen)
                   .matchIPDst      (DireccionIpDestino)
                   .matchIPProtocol (IPv4.PROTOCOL_TCP)
                   .matchTcpDst     (port).build();
\end{lstlisting}

\begin{lstlisting} [label=lst:traffic_treatment, caption= Ejemplo del
  tratamiento \textit{drop} del tráfico., captionpos=b, language= Java]
/**
* 	Tratamiento de drop del trafico.
*/
TrafficTreatment drop = DefaultTrafficTreatment.builder().drop().build();

\end{lstlisting}

Luego, se aplican las reglas sobre los dispositivos. Si la fuente del ataque se
puede aislar, entonces la dirección de origen se obtene del campo \verb|pkt| (ver Código
Fuente \ref{lst:alert_struct}) de la alerta enviada por Snort. Utilizando el
servicio \verb|hostService| de ONOS se obtiene el dispositivo SDN que mantiene
conexión directa con el origen del mencionado ataque. Para el caso en el que no
se pueda identificar la fuente, entonces se necesita encontrar más de un
dispositivo de red, por lo que se usa el servicio \verb|deviceService|.

Finalmente, es importante aclarar más en detalle como se efectuó la
implementación de los requerimientos \textbf{RF08}, \textbf{RF09}, \textbf{RF10}
y \textbf{RF11}.

Para los primeros dos, se busca el dispositivo SDN directamente conectado al
\textit{host} de donde proviene el ataque o, en su defecto, se localizan todos
los \textit{switches} de la subcapa de acceso, con el fin de insertarles las
reglas OpenFlow. Mientras que para el tercero, dada la potencia del ataque que
describe, dichas reglas se escribirán en todos los conmutadores SDN de la red,
incluidos los del núcleo.
 

Por otra parte, a la hora de escribir las reglas OpenFlow y darles una
característica temporal (\textbf{RF11}), se utilizó el servicio de ONOS
\verb|flowObjectiveService|, tal como se visualiza en el Código Fuente
\ref{lst:write_rule}.\\

\begin{lstlisting} [label=lst:write_rule, caption= Escritura de las reglas
  OpenFlow mediante el servicio \textit{flowObjectiveService}., captionpos=b, language= Java]
  /** * Creacion de la regla OpenFlow e insercion en el dispositivo SDN con
  identificacion switchId. */

flowObjectiveService.forward (switchId, DefaultForwardingObjective.builder()
          .fromApp       (appId)
          .withSelector  (selectorDelTrafico)
          .withTreatment (tratamientoDelTrafico)
          .withPriority  (prioridadDeLaRegla)
          .makeTemporary (timeoutDeLaRegla)
          .add ());
\end{lstlisting}


En el caso del requerimiento \textbf{RF08}, el tráfico a filtrar en los OVS es
teniendo en cuenta el Código Fuente \ref{lst:traffic_selector}, es decir, aquel
de tipo TCP en IPv4 que se emite desde una dirección IP de origen hacia otra de
destino a un puerto específico. Lo mismo abarca a \textbf{RF09}, con la única
diferencia de que el protocolo de capa de transporte es UDP y no TCP. En cambio,
para \textbf{RF10}, el tráfico es ICMP de tipo 8 desde la dirección IP de la
víctima hacia aquella de difusión utilizada por el ataque Smurf.


\subsection *{Observación}

Ante la presencia de ataques en los que se pretenda falsificar la dirección IP
de origen y cuyas direcciones de destino no sean las de difusión, el controlador
encuentra errores al intentar armar el camino del flujo correspondiente a través
del plano de datos. Esto se debe a que el punto de conexión del origen del
tráfico difiere con el destino de las respuestas, por lo que ONOS se encarga de
entregar por su cuenta los respectivos paquetes.

La problemática de lo anterior se traduce en una posible denegación de servicio
al controlador, debido a la inundación con dichos paquetes. Para evitar esto, se
instalan automáticamente reglas de \textit{drop} en los dispositivos por los que
ingresa el tráfico, de manera tal que estos flujos se descartan en el plano de
datos.
 
% Chapter Template
% cSpell:words parencite onfwhitepaper includegraphics resizebox sdncomponents
% linewidth comparqui redireccionar enrutamiento subfigure toposdn Nicira toposinsdn
\chapter{Interfaz gráfica de usuario} % Main chapter title 

\label{Chapter6} % Change X to a consecutive number; for referencing this chapter elsewhere, use \ref{ChapterX}

En este capítulo se hará una breve explicación de la API y de la aplicación que
posibilitan al administrador del sistema una configuración y un uso sencillos e
intuitivos.

En primer lugar, se detallarán los requerimientos. Luego, se hará una breve
descripción del proceso de diseño y de implementación de este subsistema que
utiliza la interfaz NORTHBOUND del controlador. A su vez, tendrá su lugar la
explicación de las herramientas utilizadas para el desarrollo de la interfaz
gráfica de usuario.

%-----------------------------bitext comileArquitectura del protocolo--------------------
%	SECTION 1
%----------------------------------------------------------------------------------------
\section{Requerimientos} \label{sec:reqs_gui}

Los requerimientos funcionales de la aplicación y de la API son los siguientes:

\paragraph{RF12.} La aplicación y la API deben permitirle al usuario
introducirle a las aplicaciones de filtrado del tráfico y de detección de
anomalías %\ref{sec:Chapter4} y \ref{sec:Chapter5}
las direcciones IP de los distintos IDSs.

\paragraph{RF13.} En cuanto a la aplicación de detección de anomalías, la
aplicación y la API deben posibilitarle al usuario la introducción de las
etiquetas de los dispositivos de la red (acceso, distribución, núcleo, servicio
y frontera) para el correcto funcionamiento de la misma.
\paragraph{RF14.} El usuario debe poder ingresar a la aplicación de detección de
anomalías los parámetros del modelo de comportamiento esperado de su red, los
cuales son los siguientes por cada dispositivo OVS de distribución:

\begin{itemize}
\item{Cantidad de paquetes entrantes al OVS desde los enlaces conectados
    directamente a los dispositivos de la subcapa de acceso.}
\item{Cantidad de bytes entrantes al OVS desde los enlaces conectados
    directamente a los dispositivos de la subcapa de acceso.}
\end{itemize}

\paragraph{RF15.} La aplicación y la API deben permitirle al usuario la
visualización de los identificadores de los conmutadores de borde cuyo tráfico
entrante presente comportamiento sospechoso.
\paragraph{RF16.} La aplicación y la API deben mostrarle al usuario información actualizada, en formato gráfico, de la evolución en el tiempo de los
valores que se originan como resultado de (\ref{eq:Xcount}) y (\ref{eq:Scount}).
\paragraph{RF17.} El usuario debe poder ver, a través de la aplicación y de la
API, un gráfico con la evolución en el tiempo de los valores del tráfico global entrante a la red y observado cada 10 segundos, etiquetado con la letra \textbf{\(N\)} en
(\ref{eq:Xcount}) y (\ref{eq:Scount}).
\paragraph{RF18.} La aplicación y la API deben lograr que el usuario pueda visualizar información sobre las alertas recibidas de los distintos IDSs. Dicha información debe estucturarse de la manera presentada a continuación:
\begin{itemize}
	\item{Tipo de ataque. (Smurf o inundación).}
	\item{Mensaje de la alerta.}
	\item{Dirección IP de origen del paquete que provocó el evento.}
	\item{Dirección IP de destino del paquete que provocó el evento.}
	\item{IDS que generó la alerta.}
	\item{\verb|sid| de la regla que provocó la alerta.}
	\item{Fecha y hora de la generación del evento.}
	\item{Si el ataque se encuentra solucionado o no.}
	\item{Dispositivo de red en donde se encuentra la solución a la alerta.}
\end{itemize}

\paragraph{RF19.} El usuario debe poder ver, a través de la aplicación, las reglas de \textit{drop} activas en los distintos OVS. La información a mostrar de las distintas reglas se estructura de la manera expuesta a continuación:
\begin{itemize}
\item {Identificación del dispositivo en donde está instalada la regla.}
\item {\textit{Timeout} de la regla.}
\item {Información del protocolo utilizado por el \textit{Traffic Selector}.}
\item {Dirección IP de origen indicada por la regla.}
\item {Dirección IP de destino indicada por la regla.}
\item {El número del puerto de destino indicado por la regla, si el protocolo es TCP o UDP.}
\item {El tipo ICMP indicado por la regla, en caso de que el protocolo sea el correspondiente.}
\end{itemize}

\section {Diseño e implementación de la API}

A la hora de diseñar la API, se debieron tener en cuenta cuáles métodos eran
necesarios para lograr los distintos requerimientos. Para ello se construyó la
Tabla \ref{tab:api_methods}.
\\
\\

\begin{longtable}[c]{|c|
		>{\columncolor[HTML]{3166FF}}l |l|}
		\hline
		
		\cellcolor[HTML]{EFEFEF}\textbf{Requerimiento} & \multicolumn{1}{c|}{\cellcolor[HTML]{EFEFEF}\textbf{Método}} & \multicolumn{1}{c|}{\cellcolor[HTML]{EFEFEF}\textbf{Descripción}} \\ \hline
		\endfirsthead
		%
		\multicolumn{3}{c}%
		{{\bfseries Table \thetable\ continued from previous page}} \\
		\endhead
		%
		\cellcolor[HTML]{FFFFFF}                                & {\color[HTML]{FFFFFF} 1 - GET}                                & \cellcolor[HTML]{FFFFFF}Obtener direcciones IP de los distintos \textit{hosts}. \\ \cline{2-3} 
		\cellcolor[HTML]{FFFFFF}                                & {\color[HTML]{FFFFFF} 2 - GET}                                & Obtener direcciones IP de los IDSs.                                    \\ \cline{2-3} 
		\cellcolor[HTML]{FFFFFF}                                & \cellcolor[HTML]{32CB00}{\color[HTML]{FFFFFF} 3 - POST}       & Cargar dirección IP de un IDS.                                        \\ \cline{2-3} 
		\multirow{-4}{*}{\cellcolor[HTML]{FFFFFF}\textbf{RF12}} & \cellcolor[HTML]{FE0000}{\color[HTML]{FFFFFF} 4 - DELETE}     & Eliminar dirección IP de un IDS.                                      \\ \hline
		                                                        & {\color[HTML]{FFFFFF} 5 - GET}                                & Obtener descripciones de los \textit{switches}.                                 \\ \cline{2-3} 
		\multirow{-2}{*}{\textbf{RF13}}                         & \cellcolor[HTML]{32CB00}{\color[HTML]{FFFFFF} 6 - POST}       & Configurar etiquetas.                                                      \\ \hline
		                                                        & {\color[HTML]{FFFFFF} 7 - GET}                                & Obtener \textit{switches} de distribución.                                     \\ \cline{2-3} 
		                                                        & {\color[HTML]{FFFFFF} 8 - GET}                                & Obtener valores del comportamiento esperado.                           \\ \cline{2-3} 
		\multirow{-3}{*}{\textbf{RF14}}                         & \cellcolor[HTML]{32CB00}{\color[HTML]{FFFFFF} 9 - POST}       & Cargar valores del comportamiento esperado.                            \\ \hline
		\textbf{RF15}                                           & {\color[HTML]{FFFFFF} 10 - GET}                               & Obtener dispositivos de borde sospechosos.                             \\ \hline
		\textbf{RF16}                                           & {\color[HTML]{FFFFFF} 11 - GET}                               & Obtener historial de valores de chi cuadrado.                          \\ \hline
		\textbf{RF17}                                           & {\color[HTML]{FFFFFF} 12 - GET}                               & Obtener valores del tráfico observado.                                \\ \hline
		\textbf{RF18}                                           & {\color[HTML]{FFFFFF} 13 - GET}                               & Obtener historial de alertas.                                          \\ \hline
		\textbf{RF19}                                           & {\color[HTML]{FFFFFF} 14 - GET}                               & Obtener reglas OpenFlow de \textit{drop}.                                       \\ \hline
		\caption{Métodos de la API.}
		\label{tab:api_methods}\\
	\end{longtable}


	De todos los métodos que figuran en la Tabla \ref{tab:api_methods}, una gran parte puede obtenerse desde la API que ofrece por defecto el controlador ONOS, tal como se observa en la Tabla \ref{tab:api_onos}.\\ 



	\begin{longtable}[c]{|c|
		>{\columncolor[HTML]{3166FF}}l |}
		\hline
		\cellcolor[HTML]{EFEFEF}\textbf{Método} & \multicolumn{1}{c|}{\cellcolor[HTML]{EFEFEF}\textbf{\begin{tabular}[c]{@{}c@{}}API de ONOS\end{tabular}}} \\ \hline
		\endfirsthead
		%
		\multicolumn{2}{c}%
		{{\bfseries Table \thetable\ continued from previous page}} \\
		\endhead
		%
		1                                        & {\color[HTML]{FFFFFF} GET /hosts}                                                     \\ \hline
		5                                        & {\color[HTML]{FFFFFF} GET /devices}                                                   \\ \hline
		6                                        & \cellcolor[HTML]{32CB00}{\color[HTML]{FFFFFF} POST /network/configuration}            \\ \hline
		7                                        & {\color[HTML]{FFFFFF} GET /devices}                                                   \\ \hline
		10                                       & {\color[HTML]{FFFFFF} GET /devices}                                                   \\ \hline
		14                                       & {\color[HTML]{FFFFFF} GET /flows}                                                     \\ \hline
		\caption{Métodos de la API de ONOS.}
		\label{tab:api_onos}\\
	\end{longtable}

	Agregando a la Tabla \ref{tab:api_onos}, los métodos 7 y 10 utilizan la misma consulta a la API de ONOS. Su diferencia radica en el procesamiento posterior para obtener los diferentes \textit{switches} de red. Dicho procesamiento se basa en consultar las distintas etiquetas, interesando para la obtención de los dispositivos de distribución aquella denominada \textit{distribution}, siendo  \textit{inspect} para los sospechosos.   

	Por otra parte, se debieron implementar aquellos métodos no cubiertos por defecto por el controlador. Es por ello que se presenta la Tabla \ref{tab:api_tesis} con lo efectuado en el presente proyecto en materia de APIs.

% Please add the following required packages to your document preamble:
% \usepackage[table,xcdraw]{xcolor}
% If you use beamer only pass "xcolor=table" option, i.e. \documentclass[xcolor=table]{beamer}
\begin{table}[]
	\centering
	\begin{tabular}{|c|
			>{\columncolor[HTML]{3166FF}}l |}
			\hline
			\cellcolor[HTML]{EFEFEF}{\color[HTML]{000000} \textbf{Método}} & \multicolumn{1}{c|}{\cellcolor[HTML]{EFEFEF}{\color[HTML]{000000} \textbf{API generada}}} \\ \hline
			2                                                               & {\color[HTML]{FFFFFF} GET /ids/ips}                                                       \\ \hline
			3                                                               & \cellcolor[HTML]{32CB00}{\color[HTML]{FFFFFF} POST /ids/ip}                               \\ \hline
			4                                                               & \cellcolor[HTML]{FE0000}{\color[HTML]{FFFFFF} DELETE /ids/ip}                              \\ \hline
			8                                                               & {\color[HTML]{FFFFFF} GET /statistic/download}                                            \\ \hline
			9                                                               & \cellcolor[HTML]{32CB00}{\color[HTML]{FFFFFF} POST /statistic/upload}                     \\ \hline
			11                                                              & {\color[HTML]{FFFFFF} GET /statistic/chi/values}                                          \\ \hline
			12                                                              & {\color[HTML]{FFFFFF} GET /statistic/chi/values}                                          \\ \hline
			13                                                              & {\color[HTML]{FFFFFF} GET /fm/alarms}                                                     \\ \hline
		\end{tabular}
		\caption{Métodos de la API implementados.}
		\label{tab:api_tesis}
	\end{table}



	Agregando información a la mencionada Tabla \ref{tab:api_tesis}, cabe destacar
  que los métodos 2, 3 y 4 pertenecen a la aplicación de filtrado del tráfico,
  la cual debió incluir una leve modificación para permitir que los IDSs de la
  topología de la red se puedan autenticar a través del envío de mensajes
  iniciales por medio de sus \textit{sockets} conectados al controlador. En
  dicho proceso, los detectores envían sus direcciones IP a ONOS, el cual los
  acepta o no si desde la GUI se cargaron previamente dichas direcciones por
  medio del método 3.

	Por otro lado, los métodos 8, 9, 11 y 12 corresponden a la aplicación de detección de anomalías, de donde es posible obtener los valores estadísticos que se solicitan.

	Y por último, el método 13 exigió la utilización, en la aplicación de filtrado del tráfico, de dos servicios que ofrece ONOS: \texttt{AlarmService} y \texttt{AlarmProviderService}. Ambos se utilizan para interactuar con el manejador de alarmas de los dispositivos. En este caso, lo único que se requiere es mantener un historial de aquellas alarmas o alertas provistas por los IDSs, las cuales presentan reglas de \textit{drop}, instaladas en los dispositivos de red, que las atienden y solucionan. Además, se hace necesario agregar o eliminar dichas alertas, lo cual se efectúa de manera sencilla por medio de estos servicios. A su vez, éstos ofrecen una API, de la cual se utilizará el método \textit{GET /alarms} para obtenerlas.



\section {Diseño e implementación de la interfaz gráfica de usuario}

Para el desarrollo de la interfaz web de usuario se hizo necesario el
aprendizaje de determinados lenguajes, librerías y herramientas.

Una de esas herramientas que se utilizó es el \textit{microframework} web Flask \parencite{flask}, el cual se encuentra escrito en Python e incluye un servidor HTTP. Este \textit{framework} permite el desarrollo y despliegue de aplicaciones web bajo un estilo de arquitectura de software Modelo-Vista-Controlador. La vista en este caso sería la página HTML, la cual fue confeccionada con la ayuda de Bootstrap.
Además, el mencionado servidor actúa de cliente con respecto a las APIs de ONOS.

Bootstrap \parencite{bootstrap} se trata de un conjunto de herramientas \textit{open source} utilizadas para el desarrollo \textit{front-end} de sitios web con los lenguajes HTML, CSS y Javascript. Con ayuda de sus \textit{templates} es posible construir diseños amigables, estéticos y de rápido prototipado.

A su vez, con el objetivo de insertar datos y textos determinados y dinámicos dentro de un documento HTML desde la aplicación de Flask, se debió recurrir a un lenguaje de \textit{templates} escrito en Python denominado Jinja2 \parencite{jinja}. Por otra parte, para generar los gráficos se usó la librería Pygal \parencite{pygal}, que los efectúa a partir de la creación de imágenes con formato vectorial SVG.


Habiendo definido las principales herramientas utilizadas, se comenzará a mostrar la implementación de los distintos requerimientos. En primer lugar, la interfaz web cuenta con dos vistas, una de configuración y otra de control.

En la primer vista las direcciones IP de los IDSs pueden cargarse y eliminarse, tal como se observa en la Figura \ref{fig:post_ids}. Además, las etiquetas de los dispositivos pueden modificarse a valores elegidos por el usuario o a aquellos por defecto que se encuentran configurados en un archivo. Por último, pueden ingresarse los valores del modelo de comportamiento esperado en cuanto a la cantidad de paquetes y de bytes que ingresan por ciertos enlaces a los dispositivos de distribución (Ver Figura \ref{fig:set_comportamiento_esperado}).

\begin{figure}[H]
	\centering 
	\includegraphics[width=\textwidth]{ids}
	\caption{Carga y eliminación de la IP de los IDSs.}
	\label{fig:post_ids}
\end{figure}

\begin{figure}[H]
	\centering 
	\includegraphics[width=\textwidth]{valoresesperados}
	\caption{Carga del comportamiento esperado de la red.}
	\label{fig:set_comportamiento_esperado}
\end{figure}

En la segunda vista se pueden observar las alertas de los ataques y las reglas
OpenFlow utilizadas para detenerlos, tal como se observa en las Figuras
\ref{fig:alertas} y \ref{fig:reglas_drop}. A su vez, se encuentra una tabla con
los dispositivos de borde con tráfico sospechoso y el gráfico que muestra la
evolución de los valores de chi cuadrado en el tiempo (ver Figura
\ref{fig:Chi_cuadrado_graph}). Por último, existe otro gráfico que deja
visualizar la forma en que varía el tráfico observado cada 10 segundos en la red
a lo largo del tiempo. Este gráfico se muestra en la Figura
\ref{fig:trafico_obs_10}.

\begin{figure}[H]
	\centering 
	\includegraphics[width=\textwidth]{alerta}
	\caption{Alertas de ataques.}
	\label{fig:alertas}
\end{figure}


\begin{figure}[H]
	\centering 
	\includegraphics[width=\textwidth]{reglaOpenflowDrop}
	\caption{Reglas OpenFlow de \textit{drop} en el sistema.}
	\label{fig:reglas_drop}
\end{figure}


\begin{figure}[H]
	\centering 
	\includegraphics[width=\textwidth]{chicuadrado}
	\caption{Gráfico de la evolución de los valores de chi cuadrado en el tiempo.}
	\label{fig:Chi_cuadrado_graph}
\end{figure}

\begin{figure}[H]
	\centering 
	\includegraphics[width=\textwidth]{trafico}
	\caption{Gráfico de la evolución del tráfico observado en la red a través del tiempo.}
	\label{fig:trafico_obs_10}
\end{figure}


 
% Chapter Template cSpell:words parencite onfwhitepaper includegraphics
% resizebox sdncomponents linewidth comparqui redireccionar enrutamiento
% subfigure toposdn Nicira toposinsdn
\chapter{Diseño e implementación del entorno de emulación}
\label{Chapter7} % Change X to a consecutive number; for referencing this
                 % chapter elsewhere, use \ref{ChapterX}

En el presente capítulo se detalla el entorno de emulación utilizado como
vehículo de prueba para validar y evaluar las aplicaciones desarrolladas. Se
describen sus requerimientos, la topología empleada con sus respectivos agentes,
el escenario de generación de tráfico legítimo y las reglas del IDS utilizadas
para detectar los ataques.

Este entorno se llevó a cabo en una plataforma con las siguientes
características:

\begin{itemize}
	\item{Sistema operativo:} Ubuntu 18.04 (64 bits).
	\item{Procesador:} Intel® Core™ i7-3770 CPU @ 3.40GHz × 8.
	\item{Memoria RAM:} DDR3 con capacidad igual a 15,6 GiB.
\end{itemize}


\section {Requerimientos}

Como se mencionó anteriormente se realizó una emulación con una arquitectura
característica de un ISP, tal como se define en la Sección \ref{sec:isp}.
Partiendo de esta topología, los requerimientos funcionales del entorno de
emulación son los siguientes:

\paragraph{RF20.} Cuando la red se ecuentre bajo un determinado ataque generado por la herramienta \verb|hping3|, los usuarios legítimos no podrán interactuar con los servidores HTTP de la topología.

\paragraph{RF21.} Los servidores no deben perder su disponibilidad frente al
escenario de tráfico legítimo, explicado en la Sección
\ref{sec:traffic_legitim}.

\paragraph{RF22.} La topología debe poseer IDSs Snort conectados directamente a
los dispositivos SDN \textit{distribution}, de acuerdo al enfoque explicado en
la Sección \ref{sec:motivacion_app_4}.

\paragraph{RF23.} Los IDSs deben contener las reglas para detectar los ataques
que se generan con los comandos que figuran en el Apéndice \ref{AppendixA}. \\

\section {Topología SDN}

La topología utilizada en el presente proyecto es la que se observa en la Figura
\ref{fig:topologia_final}. Controlada por ONOS y administrada por ContainerNet, contiene conmutadores OVS y
\textit{hosts} emulados por contenedores de Docker.\\ %Figure

\begin{figure}[H]
	\centering 
	\includegraphics[width=\textwidth]{topo_h}
	\caption{Topología utilizada con la clasificación de los distintos tipos de
    \textit{hosts}.}
	\label{fig:topologia_final}
\end{figure}

En dicha topología se utiliza una distribución y clasificación de los
conmutadores SDN de manera similar a la de las arquitecturas actuales de los
ISP, tal como se muestra en la Figura \ref{fig:ovs_classif}. \\

\begin{figure}[H]
	\centering 
	\includegraphics[width=0.5\textwidth]{topo_s}
	\caption{Clasificación y distribución de los conmutadores SDN teniendo en
    cuenta la arquitectura de un ISP.}
	\label{fig:ovs_classif}
\end{figure}

A su vez, en la Sección \ref{sec:agentes_emulacion} se muestran las
características de los \textit{hosts} que forman parte de esta topología y, en
la Tabla \ref{tab:bw_links}, lo que se observa son los anchos de banda de los
diferentes enlaces.

\begin{table}[H]
	\centering
	\begin{tabular}{|c|c|c|}
		\hline
		\cellcolor[HTML]{EFEFEF}\textbf{Extremo A del enlace}  & \cellcolor[HTML]{EFEFEF}\textbf{Extremo B del enlace}  & \cellcolor[HTML]{EFEFEF}\textbf{Ancho de banda [Mbits/s]}  \\ \hline
		OVS del núcleo      & OVS del núcleo         & 1000 \\ \hline
		OVS del núcleo      & OVS de distribución    & 500  \\ \hline
		OVS de distribución & OVS de acceso          & 300  \\ \hline
		OVS de acceso       & \textit{Host} servidor & 100  \\ \hline
		OVS de acceso       & \textit{Host} cliente  & 12   \\ \hline
		OVS de distribución & IDS                    & 300  \\ \hline
		OVS de distribución & OVS de distribución    & 500  \\ \hline
	\end{tabular}
	\caption{Anchos de banda de los diferentes enlaces de la topología.}
	\label{tab:bw_links}
\end{table}


Los valores elegidos en la Tabla \ref{tab:bw_links} presentan una explicación. A medida que nos acercamos más al núcleo del ISP, mayor es el ancho de banda necesario en los enlaces debido a que contienen paquetes de múltiples sectores de la red. En cambio, en capas cercanas a las de borde, el tráfico a manejar es de muy pocos \textit{hosts}. Además, los servidores van a requerir un mayor ancho de banda que sus clientes, para poder responderles de manera adecuada. A su vez, cabe destacar que para el caso de los IDSs, sus enlaces deben presentar un elevado ancho de banda debido a que se conectan directamente a los OVS de distribución, los cuales manipulan tráfico de distintos dispositivos de acceso. 


\section {Agentes del entorno de emulación} \label{sec:agentes_emulacion}

En la Figura \ref{fig:topologia_final} se observan referencias que indican
distintos comportamientos de los \textit{hosts} que integran la topología. A
continuación se comenzarán a explicar en detalle cada uno de ellos.

\paragraph{IDS.} Los contenedores que actúan como detectores están basados en la
imagen de Docker \textit{ubuntu}. A dicha imagen se le agrega el software
Snort (versión 2.9.11.1), su archivo de configuración y las reglas para detectar
los ataques (ver Sección \ref{sec:snort_reglas}). Además, se encuentran
conectados directamente a los OVS de distribución, tal como menciona el enfoque
elegido en el Capítulo \ref{sec:Chapter4}.

\paragraph{Usuario legítimo.} Estos agentes también están basados en la imagen
de Docker \textit{ubuntu}. Participan en el escenario de generación de tráfico
legítimo por medio de consultas HTTP GET a los servidores vía las herramientas
Siege o Curl. También se utilizan para funciones de monitoreo y recolección de
datos con el fin de confeccionar determinados gráficos. Se encuentran conectados
a los OVS de la subcapa de acceso.

\paragraph{Bot.} Estos contenedores presentan la misma imagen base que
los usuarios legítimos. Además, al igual que ellos, están conectados a los OVS de la subcapa de acceso y participan en el escenario de generación de tráfico mencionado anteriormente. Sin
embargo, llevan en su sistema de archivos los \textit{scripts} necesarios para ejecutar
los comandos con los cuales se generan los ataques (ver Apéndice
\ref{AppendixA}).


\paragraph{Servidor HTTP.} Basados en la imagen \textit{httpd}, permiten poner
en marcha un servidor web Apache (versión 2.4) que contiene una página web la
cual, por cada consulta HTTP GET, realiza un cálculo de la serie de Fibonacci para incrementar el procesamiento en el servidor y dar mayor efectividad a los ataques. Se ubican detrás de los dispositivos de la subcapa de servicio.

\paragraph{Servidor DNS.} En base a la imagen de Docker \textit{sameersbn/bind}, se construyen los \textit{hosts} que permiten levantar este tipo de servidores, los cuales son utilizados en el presente proyecto únicamente para la resolución de nombres de dominio. Sus ubicaciones en la red son idénticas a la de los otros servidores explicados anteriormente.

\section {Escenario de generación de tráfico
  legítimo} \label{sec:traffic_legitim}

Para detectar comportamientos sospechosos primero hace falta tener el modelo de
un comportamiento esperado. Pero antes de comenzar con la explicación de esta
sección, es necesario introducir conceptos del área de probabilidad y
estadística útiles para los tiempos entre consultas dirigidas a los servidores.
Uno de ellos es la distribución de Poisson, que se utiliza para fenómenos
discretos, como por ejemplo, la cantidad de consultas a un servidor web por
minuto. Esta distribución, de acuerdo a \parencite{poisson}, permite conseguir
la probabilidad de ocurrencia de una determinada cantidad de eventos discretos
en cierto intervalo de tiempo o región específica, a partir de una frecuencia de
ocurrencia media de dicho evento. Por otra parte, la extensión del intervalo
entre la ocurrencia de dos eventos sucesivos discretos distribuidos en base a
Poisson se modela a través de la distribución exponencial
\parencite{distribucion_exponencial}. Entonces, para el tiempo entre las
consultas hacia un servidor web se tendría que utilizar esta última
distribución, la cual presenta las siguientes características, de acuerdo a
\parencite{distribucion_exponencial}:

\begin{itemize}
\item{Distribución continua de probabilidad.}
\item{Parámetro \(\lambda\) mayor a cero.}
\item{Valor esperado igual a la inversa de \(\lambda\).}
\item{Varianza igual al valor esperado al cuadrado.}  
\end{itemize}

Los conceptos anteriores son útiles a la hora de diseñar y construir un
escenario de generación de tráfico legítimo. Cabe destacar que dicho escenario puede ser cualquiera, no necesariamente el que se propone en esta sección. Esto es debido a que la aplicación de detección de anomalías se adapta a diferentes modelos de comportamiento esperado.

El escenario propuesto en este trabajo hace hincapié en un ISP cuyos servicios son otorgados a una ciudad con jornada laboral que va desde las 8 hasta las 12 horas y luego desde las 16 hasta las 20 hs. El tráfico es efectuado por 18 usuarios (6 empresas y 12 clientes domésticos) y la duración de la emulación es de 1 hora, en donde cada intervalo de 2,5 minutos equivale a una hora de un día
hábil (ver correspondencia en Tabla \ref{tab:minutos_horas}). En ese tiempo, se efectuarán consultas HTTP GET a los servidores
utilizando la distribución exponencial para definir el tiempo entre las mismas.


\begin{table}[H]
	\centering
	\begin{tabular}{|c|c|c|c|c|}
		\hline
		\cellcolor[HTML]{EFEFEF}\textbf{Etapa} & \cellcolor[HTML]{EFEFEF}\textbf{\begin{tabular}[c]{@{}c@{}}Inicio del intervalo\\ de emulación\end{tabular}} & \cellcolor[HTML]{EFEFEF}\textbf{\begin{tabular}[c]{@{}c@{}}Fin del intervalo\\ de emulación\end{tabular}} & \cellcolor[HTML]{EFEFEF}\textbf{\begin{tabular}[c]{@{}c@{}}Hora de inicio\\ en día hábil\end{tabular}} & \cellcolor[HTML]{EFEFEF}\textbf{\begin{tabular}[c]{@{}c@{}}Hora de fin\\ en día hábil\end{tabular}}\\ \hline
        1 & 0 minutos & 10 minutos & 00:00 hs & 04:00 hs\\ \hline		
        2 & 10 minutos & 20 minutos & 04:00 hs & 08:00 hs\\ \hline		
        3 & 20 minutos & 30 minutos & 08:00 hs & 12:00 hs\\ \hline		
        4 & 30 minutos & 40 minutos & 12:00 hs & 16:00 hs\\ \hline		
        5 & 40 minutos & 50 minutos & 16:00 hs & 20:00 hs\\ \hline		
        6 & 50 minutos & 60 minutos & 20:00 hs & 00:00 hs\\ \hline		
	\end{tabular}
	\caption{Conversión de los minutos de la emulación a horas de un día hábil.}
	\label{tab:minutos_horas}
\end{table}

Por otra parte, los comportamientos de los distintos tipos de usuarios en los
diferentes intervalos anteriormente definidos se muestran en la Tabla
\ref{tab:comportamientos}. Esto se lo efectúa para dotar de mayor realidad al
escenario, ya que se tienen en cuenta las horas pico de consumo por parte de las
empresas y las correspondientes a la de los usuarios domésticos.

\begin{table}[H]
	\centering
	\begin{tabular}{|c|c|c|c|}
		\hline
        \cellcolor[HTML]{EFEFEF}\textbf{Etapa} & \cellcolor[HTML]{EFEFEF}\textbf{Intervalo [min, min)} & \cellcolor[HTML]{EFEFEF}\textbf{Clientes domésticos} & \cellcolor[HTML]{EFEFEF}\textbf{Empresas}\\ \hline
        1 & [0, 10)  & Modo inactivo & Modo inactivo \\ \hline
        2 & [10, 20) & Modo inactivo & Modo inactivo \\ \hline
        3 & [20, 30) & Modo inactivo & Modo activo   \\ \hline
        4 & [30, 40) & Modo activo   & Modo inactivo \\ \hline
        5 & [40, 50) & Modo inactivo & Modo activo   \\ \hline
        6 & [50, 60) & Modo activo   & Modo inactivo \\ \hline
						
	\end{tabular}
	\caption{Comportamientos de los distintos tipos de usuarios 
en los diferentes intervalos de la emulación.}
	\label{tab:comportamientos}
\end{table}

Cabe destacar que los dos tipos de comportamiento para cada \textit{host}
cliente se definen a continuación:

\begin{itemize}
\item{\textbf{Modo activo:}} la decisión de enviar (80\% de probabilidad) o no
  (20\% de probabilidad) una consulta HTTP GET al servidor se toma cada una
  fracción de tiempo que sigue un comportamiento de una distribución exponencial
  con un valor esperado igual a 5 segundos.
\item{\textbf{Modo inactivo:}} la decisión de enviar (20\% de probabilidad) o no
  (80\% de probabilidad) una consulta HTTP GET al servidor se toma cada una
  fracción de tiempo que sigue un comportamiento de una distribución exponencial
  con un valor esperado igual a 30 segundos.
\end{itemize}


Una vez diseñado este escenario de generación de tráfico legítimo, se lo debió
implementar. Para ello se utilizó el lenguaje de programación Python, con el
cual se efectuó un \textit{script} maestro que comanda mediante hilos los comportamientos
de los diferentes clientes utilizados a través de \textit{scripts} esclavos, presentes en
el sistema de archivos de los mismos. Para explicar mejor lo anterior se muestra
el diagrama de actividad \ref{fig:diagrama_escenario_legitimo_1}. \\

\begin{figure}[H]
	\centering 
	\includegraphics[width=0.9\textwidth]{sim_etapa}
	\caption{Diagrama de actividad de la administración de los comportamientos de
    los distintos clientes.}
	\label{fig:diagrama_escenario_legitimo_1}
\end{figure}

Por otra parte, la Figura \ref{fig:diagrama_escenario_legitimo_2} permite
visualizar la ejecución del modo inactivo de la primera etapa en un \textit{bot}
o cliente doméstico. \\

\begin{figure}[H]
	\centering 
	\includegraphics[width=\textwidth]{sim_atack}
	\caption{Diagrama del flujo de ejecución de la primera etapa (modo inactivo)
    de un bot o un cliente doméstico.}
	\label{fig:diagrama_escenario_legitimo_2}
\end{figure}


Además, es importante mencionar que este escenario posibilitó la obtención, para
la aplicación de detección de anomalías, de los valores del modelo de
comportamiento esperado de la red, en cuanto a la cantidad de paquetes y de
bytes en los conmutadores de distribución SDN, los cuales deben ser ingresados a
la GUI (para más información ver la Sección \ref{sec:reqs_gui}).

Por último, a la hora de validar el sistema desarrollado en este proyecto (ver Capítulo \ref{sec:Resultados}),
a la mencionada aplicación de detección de anomalías se le debió modificar el tiempo durante el cual almacena métricas globales. Es decir, en vez de ser un día ese intervalo, se lo tomó como una hora, con el fin de agilizar el proceso de prueba. A su vez, comparado con 10 segundos y si se considera el tiempo entre consultas, una hora sigue siendo, estadísticamente, una cantidad bastante grande.

\section {Reglas de Snort utilizadas}\label{sec:snort_reglas}

Las reglas que se insertaron en los IDSs de la topología instanciada son las que
se presentan en el Código Fuente \ref{lst:snort_rules}, en función de los
distintos ataques utilizados (\textbf{RF23}).


\begin{lstlisting} [label=lst:snort_rules, caption= Reglas de Snort utilizadas.,
  captionpos=b]
// TCP SYN flood.
alert tcp any any -> any 80 (msg: "TCP: SYN flood track by src"; flags: S; flow: stateless; threshold: type both, track by_src, count 1000, seconds 30; sid: 1000041; rev: 1;)

// UDP flood.
alert udp any any -> any any (msg: "UDP: UDP flood track by src"; flow: stateless; threshold: type both, track by_src, count 5000, seconds 30; classtype: attempted-dos; sid: 1000040; rev: 1;)

// Smurf.
alert icmp any any -> any any (msg: "ICMP: SMURF Attack"; itype: 8; threshold: type both, track by_src, count 40, seconds 30; sid: 1000005; rev: 1;) 

// TCP RESET flood.
alert tcp any any -> any any (msg: "TCP: RESET flood track by src"; flags: R;  threshold: type both, track by_src, count 1000, seconds 30; sid: 1000048; rev: 1;)

// TCP FIN flood.
alert tcp any any -> any any (msg: "TCP: FIN flood track by src"; flags: F; threshold: type both, track by_src, count 1000, seconds 30; sid: 1000044; rev: 1;)

// TCP SYN FIN flood.
alert tcp any any -> any any (msg: "TCP: SYN FIN flood track by src"; flags: SF; threshold: type both, track by_src, count 1000, seconds 30; sid: 1000047; rev: 1;)

// TCP PUSH ACK flood.
alert tcp any any -> any any (msg: "TCP: PUSH ACK flood track by src"; flags: PA; threshold: type both, track by_src, count 1000, seconds 30; sid: 1000046; rev: 1;)
\end{lstlisting}

A continuación se explicarán algunas de las reglas más representativas de esta
sección:

\begin{itemize}
\item {\textbf{Regla de ataque TCP SYN flood.}} Hace referencia a que se
  generará una alerta con una frecuencia de 30 segundos, si durante ese
  intervalo de tiempo se cuentan 1000 paquetes TCP con el \textit{flag} SYN
  activado que se envían desde alguna red y puerto hacia cualquier red destino
  al puerto 80. En dichas alertas se imprimirá el mensaje \textit{TCP: SYN flood
    track by src}. Por otra parte, la regla tendrá un identificador de Snort
  igual a 1000041, un número de revisión de 1 y será aplicada a cualquier tipo
  de tráfico. Cabe destacar además que se lleva un conteo de paquetes por cada
  dirección de red de origen.
\item {\textbf{Regla de ataque UDP flood.}} Se expresa que se generará una
  alerta con una frecuencia de 30 segundos, si durante ese intervalo de tiempo
  se cuentan 5000 paquetes UDP que se envían desde alguna red y puerto hacia
  cualquier red destino y puerto. En dichas alertas se imprimirá el mensaje
  \textit{UDP: UDP flood}. Por otra parte, la regla tendrá un identificador de
  Snort igual a 1000040, un número de revisión de 1, será aplicada a cualquier
  tipo de tráfico y pertenece a la categoría \textit{attempted-dos}. Cabe
  destacar además que se lleva un conteo de paquetes por cada dirección de red
  de origen.
\item {\textbf{Regla de ataque Smurf.}} Indica que se generará una alerta con
  una frecuencia de 30 segundos, si durante ese intervalo de tiempo se cuentan
  40 paquetes ICMP de tipo 8 que se envían desde alguna red y puerto hacia
  cualquier red destino y puerto. En dichas alertas se imprimirá el mensaje
  \textit{ICMP: SMURF Attack}. Por otra parte, la regla tendrá un identificador
  de Snort igual a 1000005 y un número de revisión de 1. Cabe destacar además
  que se lleva un conteo de paquetes por cada dirección de red de origen.

\end{itemize}

Si se observan en detalle las reglas, se podrán visualizar diferentes límites en
los conteos de los paquetes que podrían indicar indicios de ataques. Esto se
debe a la diferencia de potencia en los distintos tipos de estos ataques, además de
que algunos tardan más que otros en perjudicar al sistema.

A su vez, es necesario explicar el origen de estos límites o el proceso de su obtención. Los mismos están basados en un umbral de detección del ataque en un tiempo inferior a 5 segundos desde que el IDS lo analiza. Para esto se construyó en los inicios del proyecto una topología simple con un \textit{host} atacante, un IDS y un \textit{switch} OVS que los unía. Desde dicho IDS, por cada uno de los ataques mencionados en el Apéndice \ref{AppendixA}, se colocaban reglas con diferentes umbrales. Se escogía aquella con baja probabilidad de producir un falso positivo y cuya detección sea en un tiempo de unos pocos segundos, tal y como se mencionó anteriormente.


% Chapter Template
% cSpell:words parencite onfwhitepaper includegraphics resizebox sdncomponents
% linewidth comparqui redireccionar enrutamiento subfigure toposdn Nicira toposinsdn
\chapter{Resultados} % Main chapter title 

\label{sec:Resultados} % Change X to a consecutive number; for referencing this chapter elsewhere, use \ref{ChapterX}

En el presente capítulo se verificará que los resultados obtenidos por las soluciones brindadas en este proyecto cumplan con los requerimientos enunciados a lo largo del mismo.

Se detallarán las matrices de trazabilidad que relacionan dichos requerimientos con los diferentes casos de \textit{tests} y la documentación de estos últimos aparecerá al final de este capítulo.

A su vez, la nomenclatura utilizada para identificar los diferentes casos de prueba es la siguiente, considerando a \textit{n} igual a un número natural:

\begin{itemize}
\item{\textbf{T-E-n:} \textit{tests} del entorno de emulación.}
\item{\textbf{T-A-n:} casos de prueba solamente de la aplicación de detección de anomalías sin el uso de la interfaz gráfica de usuario.}
\item{\textbf{T-F-n:} \textit{tests} solamente de la aplicación de filtrado de
    tráfico o de la combinación de ésta y la de detección de
    anomalías, pero sin el uso de la GUI en ninguno de los casos.}
\item{\textbf{T-G-n:} hace referencia a aquellos casos de prueba en los cuales se utiliza la interfaz gráfica de usuario.}
\end{itemize}

\section {Matrices de trazabilidad}

En esta sección se detallan distintas matrices de trazabilidad. Una de ellas
corresponde a la que relaciona los requerimientos detallados en el Capítulo
\ref{sec:Chapter4} con los casos de \textit{tests} de la Sección
\ref{sec:Tests}, tal como se observa en las Tablas
\ref{tab:matriz_trazabilidad_4_a} y \ref{tab:matriz_trazabilidad_4_b}.

Otra de las matrices es la que tiene en cuenta los requerimientos de la
aplicación de filtrado de tráfico que puede verse en la Tabla
\ref{tab:matriz_trazabilidad_5}.

Luego, en la Matriz que se puede ver en la Tabla \ref{tab:matriz_trazabilidad_6}
los casos de \textit{tests} se contrastan con los requerimientos de la Sección
\ref{sec:reqs_gui}.

Por último, en la Tabla \ref{tab:matriz_trazabilidad_emulacion} se hace hincapié
en la matriz que presenta como requerimientos los detallados en el Capítulo
\ref{Chapter7}.


\begin{table}[H]
	\centering
	\begin{tabular}{|c|c|c|c|c|c|c|c|c|}
        \hline
		\cellcolor[HTML]{EFEFEF}\textbf{\begin{tabular}[c]{@{}c@{}}\backslashbox {Reque-\\rimien-\\tos}{Casos \\de \\prueba}\end{tabular}} & \cellcolor[HTML]{EFEFEF}\textbf{T-A-1} & \cellcolor[HTML]{EFEFEF}\textbf{T-A-2} & \cellcolor[HTML]{EFEFEF}\textbf{T-F-3} & \cellcolor[HTML]{EFEFEF}\textbf{T-F-4} & \cellcolor[HTML]{EFEFEF}\textbf{T-F-5} & \cellcolor[HTML]{EFEFEF}\textbf{T-F-6} &\cellcolor[HTML]{EFEFEF}\textbf{T-F-7} & \cellcolor[HTML]{EFEFEF}\textbf{T-F-8} \\ \hline
		\textbf{RF01}  & X  &  &   &    & & & &  \\ \hline
\textbf{RF02}                                                      & X   &   &   &   & & & &  \\ \hline
\textbf{RF03}                                                      &   &  & X  & X  & X& X&X &X  \\ \hline
\textbf{RF04}                                                      &   & X  &   &   &  & X & &  \\ \hline
\textbf{RF05}                                                      &   &   & X  &   &  & & &  \\ \hline
	\end{tabular}
	\caption{Matriz de trazabilidad de la aplicación de detección de anomalías.}
	\label{tab:matriz_trazabilidad_4_a}
\end{table}

\begin{table}[H]
	\centering
	\begin{tabular}{|c|c|}
		\hline
		\cellcolor[HTML]{EFEFEF}\textbf{\begin{tabular}[c]{@{}c@{}}\backslashbox {Requerimientos}{Casos de prueba}\end{tabular}} & \cellcolor[HTML]{EFEFEF}\textbf{T-G-2}\\ \hline
		\textbf{RF01}                                            &     \\ \hline
		\textbf{RF02}                                            &     \\ \hline
		\textbf{RF03}                                            & X \\ \hline
		\textbf{RF04}                                            &    \\ \hline
		\textbf{RF05}                                            &    \\ \hline
	\end{tabular}
	\caption{Matriz de trazabilidad de la aplicación de detección de anomalías.}
	\label{tab:matriz_trazabilidad_4_b}
\end{table}


\begin{table}[H]
	\centering
	\begin{tabular}{|c|c|c|c|c|c|c|c|c|}
		\hline
		\cellcolor[HTML]{EFEFEF}\textbf{\begin{tabular}[c]{@{}c@{}}\backslashbox{Reque-\\rimien-\\tos}{Casos \\de \\prueba}\end{tabular}} & \cellcolor[HTML]{EFEFEF}\textbf{T-F-1} & \cellcolor[HTML]{EFEFEF}\textbf{T-F-2} & \cellcolor[HTML]{EFEFEF}\textbf{T-F-3} & \cellcolor[HTML]{EFEFEF}\textbf{T-F-4} & \cellcolor[HTML]{EFEFEF}\textbf{T-F-5} &\cellcolor[HTML]{EFEFEF} \textbf{T-F-6} & \cellcolor[HTML]{EFEFEF}\textbf{T-F-7} & \cellcolor[HTML]{EFEFEF}\textbf{T-F-8} \\ \hline
		\textbf{RF06}                                                      &  X  &   &   &   &  & & & \\ \hline
		\textbf{RF07}                                                      &   & X &   &   &  & & & \\ \hline
		\textbf{RF08}                                                      &   &   & X  &   &  & X  &  X & X\\ \hline
		\textbf{RF09}                                                      &   &   &   & X  &   & X & X & X\\ \hline
        \textbf{RF10}                                                      &   &   &   &   & X & X  &  X & X \\ \hline
        \textbf{RF11}                                                      &   &   & X  & X   & X &    & & \\ \hline
	\end{tabular}
	\caption{Matriz de trazabilidad de la aplicación de filtrado de tráfico.}
	\label{tab:matriz_trazabilidad_5}
\end{table}

\begin{table}[H]
	\centering
	\begin{tabular}{|c|c|c|}
		\hline
		\cellcolor[HTML]{EFEFEF}\textbf{\begin{tabular}[c]{@{}c@{}}\backslashbox{Requerimientos}{Casos de prueba}\end{tabular}} & \cellcolor[HTML]{EFEFEF}\textbf{T-G-1} & \cellcolor[HTML]{EFEFEF}\textbf{T-G-2} \\ \hline
		
		\textbf{RF12}                                           & X  &   \\ \hline
		\textbf{RF13}                                           & X  &  \\ \hline
		\textbf{RF14}                                           & X  & X \\ \hline
		\textbf{RF15}                                           & X  & \\ \hline
		\textbf{RF16}                                           & X  & \\ \hline
		\textbf{RF17}                                           & X  & \\ \hline
		\textbf{RF18}                                           & X  & \\ \hline
		\textbf{RF19}                                           & X  & \\ \hline
	\end{tabular}
	\caption{Matriz de trazabilidad de la interfaz gráfica de usuario.}
	\label{tab:matriz_trazabilidad_6}
\end{table}


\begin{table}[H]
	\centering
	\begin{tabular}{|c|c|c|c|}
		\hline
		\cellcolor[HTML]{EFEFEF}\textbf{\begin{tabular}[c]{@{}c@{}}\backslashbox{Requerimientos}{Casos de prueba}\end{tabular}} & \cellcolor[HTML]{EFEFEF}\textbf{T-E-1} & \cellcolor[HTML]{EFEFEF}\textbf{T-E-2} & \cellcolor[HTML]{EFEFEF}\textbf{T-E-3} \\ \hline
		\textbf{RF20}                                                      & X &   &   \\ \hline
		\textbf{RF21}                                                      &  & X  &   \\ \hline
		\textbf{RF22}                                                      &   &   & X \\ \hline
		\textbf{RF23}                                                      &   &   & X \\ \hline
	\end{tabular}
	\caption{Matriz de trazabilidad del entorno de emulación.}
	\label{tab:matriz_trazabilidad_emulacion}
\end{table}


\section {Casos de prueba} \label{sec:Tests}

Los \textit{tests} que se efectuaron sobre el sistema, se encuentran documentados a continuación en las siguientes subsecciones.
Además, tal y como se mencionó anteriormente, para este proceso de verificación de los resultados obtenidos se debió modificar la aplicación de detección de anomalías.

\subsection {Casos de prueba del entorno de emulación}

Los casos de \textit{tests} referidos a los requerimientos del entorno de
emulación se describen en las Tablas \ref{tab:test_E_1}, \ref{tab:test_E_2} y
\ref{tab:test_E_3}.

En dichos Tablas, en la entrada denominada \textit{Aplicaciones a instalar en el
  controlador ONOS} se hace referencia a \texttt{\textbf{OpenFlow Provider Suite}}. Esta
aplicación se encuentra de manera predeterminada en el mencionado controlador y
es la que le permite comunicarse con los dispositivos OVS utilizando el
protocolo OpenFlow.

\subsection {Casos de prueba de la aplicación de detección de anomalías}

Por otra parte, los casos de prueba para la aplicación de detección de anomalías
se describen en las Tablas \ref{tab:test_A_1}, \ref{tab:test_A_2},
\ref{tab:test_F_3}, \ref{tab:test_F_4}, \ref{tab:test_F_5}, \ref{tab:test_F_6},
\ref{tab:test_F_7}, \ref{tab:test_F_8} y \ref{tab:test_G_2}.


\subsection {Casos de prueba de la aplicación de filtrado del tráfico}

Los casos de \textit{tests} que involucran y permiten verificar el cumplimiento de los requerimientos de la aplicación de filtrado de tráfico (ver Capítulo \ref{sec:Chapter5}) se describen en las Tablas \ref{tab:test_F_1}, \ref{tab:test_F_2}, \ref{tab:test_F_3}, \ref{tab:test_F_4}, \ref{tab:test_F_5}, \ref{tab:test_F_6}, \ref{tab:test_F_7} y \ref{tab:test_F_8}.


\subsection {Casos de prueba de la API NORTHBOUND del controlador y de la interfaz gráfica de usuario}

Los casos de prueba que involucran y permiten verificar el cumplimiento de los requerimientos mencionados en el Capítulo \ref{Chapter6} se describen en los Tablas \ref{tab:test_G_1} y \ref{tab:test_G_2}.


% Please add the following required packages to your document preamble:
% \usepackage[table,xcdraw]{xcolor}
% If you use beamer only pass "xcolor=table" option, i.e. \documentclass[xcolor=table]{beamer}
\begin{table}[H]
	\centering
	\begin{tabular}{|c|l|}
		\hline
		\rowcolor[HTML]{EFEFEF} 
		\textbf{\begin{tabular}[c]{@{}c@{}}Identificador                          \\ del test\end{tabular}} & \multicolumn{1}{c|}{\cellcolor[HTML]{EFEFEF}\textbf{T-E-1}} \\ \hline
		\rowcolor[HTML]{FFFFFF} 
		\textbf{Título}                                                          & \begin{tabular}[c]{@{}l@{}} DDoS de tipo Smurf anula disponibilidad de los enlaces del ISP. \end{tabular}  \\ \hline
		\rowcolor[HTML]{EFEFEF} 
		\textbf{Objetivo} & \begin{tabular}[c]{@{}l@{}}Utilizando la herramienta HPING3 realizar un ataque DDoS de\\ tipo Smurf a algún servidor web Apache de la topología, con el\\ fin de saturar los enlaces de la red. \end{tabular} \\ \hline
		\rowcolor[HTML]{FFFFFF} 
		\textbf{\begin{tabular}[c]{@{}c@{}}Aplicaciones a \\ instalar en el\\ controlador\\ ONOS\end{tabular}} & \begin{tabular}[c]{@{}l@{}}Openflow Provider Suite  (\texttt{org.onosproject.openflow}).\\ Aplicación de detección de anomalías.\end{tabular} \\ \hline
		\rowcolor[HTML]{EFEFEF} 
		\textbf{Pasos previos}                                                    &                                                                                            -             \\ \hline
		\rowcolor[HTML]{FFFFFF} 
		\textbf{Procedimiento} & \begin{tabular}[c]{@{}l@{}}
			\textbf{1.} Se debe poner en marcha el controlador.\\\hline  
			\textbf{2.} Se debe instanciar la topología del ISP descrita en el Capítulo \\ \ref{Chapter7} con la herramienta ContainerNet. Dicha topología cuenta con\\ 15 dispositivos Open vSwitch y 28 \textit{hosts}.\\\hline 
			\textbf{3.} Se debe ejecutar el comando de ContainerNet \texttt{pingall} para \\verificar la conectividad entre todos los \textit{hosts}.\\\hline 
			\textbf{4.} Se debe configurar el tiempo máximo que el controlador\\ ONOS verifica el estado de los enlaces.\\\hline 
			\textbf{5.} Se debe ingresar a la CLI del controlador vía el comando \texttt{ssh}\\ \texttt{-p 8101 \textless{}nombre\_host\_controlador\textgreater @\textless{}ip\_controlador\textgreater{}}. \\ Dentro de la CLI, con los comandos \textit{devices} y \textit{hosts} deben \\figurar todos los dispositivos de la red que se instanciaron y\\ que reconoce dicho controlador.\\\hline 
			\textbf{6.} Ingresar a la interfaz web del controlador ONOS a través\\ del ingreso de la URL:\\ \url{http://[IP_controlador]:8181/onos/ui/index.html}. \\ Observar la topología que allí figura. \\\hline 
			\textbf{7.} Levantar el servidor web Apache con el comando\\ \texttt{httpd-foreground}.\\\hline 
			\textbf{8.} Ejecutar en un \textit{bot} el comando \texttt{hping3 -{}-icmp -{}-flood} \\ \texttt{-d 36 -{}-spoof <ip\_servidor>~ <ip\_broadcast>}. (Ataque \\ Smurf). \\\hline 
			\textbf{9.} Desde un usuario legítimo realizar iterativamente \textit{tests} de \\conectividad.
			\end{tabular} \\ \hline

		\rowcolor[HTML]{EFEFEF}
		\textbf{\begin{tabular}[c]{@{}c@{}}Resultado \\ esperado\end{tabular}} & \begin{tabular}[c]{@{}l@{}}El resultado que se espera es que el usuario legítimo no tenga\\ conectividad con el servidor, debido a la saturación de los\\ enlaces.
		\end{tabular} \\ \hline

		\textbf{\begin{tabular}[c]{@{}c@{}}Resultado \\ obtenido\end{tabular}} & \multicolumn{1}{c|}{\textbf{{\color[HTML]{036400} APROBADO}}} \\ \hline 
	\end{tabular}
	\caption{Test T-E-1.}
	\label{tab:test_E_1}
\end{table}




% Please add the following required packages to your document preamble:
% \usepackage[table,xcdraw]{xcolor}
% If you use beamer only pass "xcolor=table" option, i.e. \documentclass[xcolor=table]{beamer}
\begin{table}[th]
	\centering
	\begin{tabular}{|c|l|}
        \hline
        \rowcolor[HTML]{EFEFEF}
		\textbf{\begin{tabular}[c]{@{}c@{}}Identificador \\ del test\end{tabular}} & \multicolumn{1}{c|}{\textbf{T-E-2}} \\ \hline
		 
        \textbf{Título}   & \begin{tabular}[c]{@{}l@{}} Escenario de tráfico legítimo no anula disponibilidad del\\ servidor del ISP.\end{tabular} \\ \hline
        \rowcolor[HTML]{EFEFEF}
		\textbf{Objetivo} & \begin{tabular}[c]{@{}l@{}}Utilizando el escenario de la Sección \ref{sec:traffic_legitim} para generar \\tráfico legítimo, verificar que el servidor web Apache \\que lo recibe no pierde su disponibilidad.\end{tabular} \\ \hline
      
        \textbf{\begin{tabular}[c]{@{}c@{}}Aplicaciones a\\ instalar en el\\ controlador\\ ONOS\end{tabular}} & \begin{tabular}[c]{@{}l@{}}Openflow Provider Suite  (\texttt{org.onosproject.openflow}).\\ Aplicación de detección de anomalías.\end{tabular} \\ \hline 
        \rowcolor[HTML]{EFEFEF}
        \textbf{Pasos previos} & Pasos 1 a 4 del procedimiento de T-E-1. \\ \hline
         
        \textbf{Procedimiento} & \begin{tabular}[c]{@{}l@{}}\textbf{1.} Levantar el servidor web Apache con el comando\\ \texttt{httpd-foreground}.\\\hline \textbf{2.} Generar con la ayuda de los scripts correspondientes \\el escenario de  tráfico legítimo descrito en la Sección \ref{sec:traffic_legitim}. \\\hline \textbf{3.} Visualizar el estado del proceso del servidor web.\\ \hline \textbf{4.} Desde 1 usuario legítimo realizar iterativamente \textit{tests}\\ de conectividad.\end{tabular} \\ \hline
		\rowcolor[HTML]{EFEFEF}
		\textbf{\begin{tabular}[c]{@{}c@{}}Resultado \\ esperado\end{tabular}} & \begin{tabular}[c]{@{}l@{}}El resultado que se espera es que el servidor no pierda \\ la disponibilidad en ningún momento (proceso caído) \\y pueda prestarle normalmente servicio a todos los \\usuarios legítimos (100\% de disponibilidad en \textit{tests} de\\ conectividad).\end{tabular} \\ \hline
        \textbf{\begin{tabular}[c]{@{}c@{}}Resultado \\ obtenido\end{tabular}} & \multicolumn{1}{c|}{\textbf{{\color[HTML]{036400} APROBADO}}} \\ \hline 
      

	\end{tabular}
	\caption{Test T-E-2.}
	\label{tab:test_E_2}
\end{table}

\begin{table}[th]
	\centering
	\begin{tabular}{|c|l|}
				\hline
		\rowcolor[HTML]{EFEFEF}
		\textbf{\begin{tabular}[c]{@{}c@{}}Identificador \\ del test\end{tabular}} & \multicolumn{1}{c|}{\textbf{T-E-3}} \\ \hline
				
		\textbf{Título}   & \begin{tabular}[c]{@{}l@{}} Ubicación y alertas del IDS.\end{tabular} \\ \hline
		\rowcolor[HTML]{EFEFEF}
		\textbf{Objetivo} & \begin{tabular}[c]{@{}l@{}}Verificar que los IDSs se encuentren conectados directamente a\\ los dispositivos SDN de distribución y que produzcan las\\ alertas en base a los ataques generados por los comandos del \\Apéndice \ref{AppendixA}.\end{tabular} \\ \hline
			
		\textbf{\begin{tabular}[c]{@{}c@{}}Aplicaciones a\\ instalar en el\\ controlador\\ ONOS\end{tabular}} & \begin{tabular}[c]{@{}l@{}}Openflow Provider Suite  (\texttt{org.onosproject.openflow}).\\ Aplicación de detección de anomalías. \end{tabular} \\ \hline 
		\rowcolor[HTML]{EFEFEF}
		\textbf{Pasos previos} & Pasos 1 a 6 del procedimiento de T-E-1. \\ \hline
				
		\textbf{Procedimiento} & \begin{tabular}[c]{@{}l@{}}\textbf{1.} Asegurar que los IDSs tengan en sus archivos de\\ configuración las reglas correspondientes para detectar los\\ ataques que se generan con los comandos del Apéndice \ref{AppendixA}.\\\hline \textbf{2.} Poner en marcha los procesos de Snort en los diferentes \\IDSs. Colocar la salida de dichos IDSs a sus respectivas \\consolas.\\\hline \textbf{3.} Pasos 1 a 2 del procedimiento de T-E-2.\\\hline \textbf{4.} Realizar los distintos ataques mencionados anteriormente \\desde un solo \textit{bot}.\\ \end{tabular} \\ \hline
		\rowcolor[HTML]{EFEFEF}
		\textbf{\begin{tabular}[c]{@{}c@{}}Resultado \\ esperado\end{tabular}} & \begin{tabular}[c]{@{}l@{}}En la topología observada, los IDSs deben estar directamente\\ conectados a los dispositivos SDN de distribución. A su vez,\\ en la consola del proceso de Snort que capte el tráfico\\ malicioso deben figurar los mensajes indicativos de las alertas\\ correspondientes.\end{tabular} \\ \hline
		\textbf{\begin{tabular}[c]{@{}c@{}}Resultado \\ obtenido\end{tabular}} & \multicolumn{1}{c|}{\textbf{{\color[HTML]{036400} APROBADO}}} \\ \hline 
	\end{tabular}
	\caption{Test T-E-3.}
	\label{tab:test_E_3}
\end{table}





\begin{table}[th]
	\centering
	\begin{tabular}{|c|l|}
		\hline
		\rowcolor[HTML]{EFEFEF}
		\textbf{\begin{tabular}[c]{@{}c@{}}Identificador \\ del test\end{tabular}} & \multicolumn{1}{c|}{\textbf{T-A-1}} \\ \hline
		\textbf{Título}   & \begin{tabular}[c]{@{}l@{}} Recolección de métricas.\end{tabular} \\ \hline

\rowcolor[HTML]{EFEFEF}
\textbf{Objetivo} & \begin{tabular}[c]{@{}l@{}}Verificar que la recolección de las métricas de la cantidad de\\ paquetes y de la cantidad de bytes para detectar\\ comportamientos sospechosos se haga cada 10 segundos.\\ También corroborar que se mantengan aquellas globales de\\ los últimos 7 días.\\\end{tabular} \\ \hline
					
\textbf{\begin{tabular}[c]{@{}c@{}}Aplicaciones a\\ instalar en el\\ controlador\\ ONOS\end{tabular}} & \begin{tabular}[c]{@{}l@{}}Openflow Provider Suite  (\texttt{org.onosproject.openflow}).\\Aplicación de detección de anomalías. \end{tabular} \\ \hline 
\rowcolor[HTML]{EFEFEF}
\textbf{Pasos previos} & Pasos 1 a 2 del procedimiento de T-E-2. \\ \hline
						
\textbf{Procedimiento} & \begin{tabular}[c]{@{}l@{}}\textbf{1.} A los 70 segundos, ejecutar en un bot un ataque DoS del\\ tipo \textit{TCP SYN flood} hacia el puerto 80 del servidor.\\ \hline \textbf{2.} Observar los \textit{logs} de ONOS.\\ \end{tabular} \\ \hline
\rowcolor[HTML]{EFEFEF}
\textbf{\begin{tabular}[c]{@{}c@{}}Resultado \\ esperado\end{tabular}} & \begin{tabular}[c]{@{}l@{}} En los \textit{logs} de ONOS debe figurar que en un tiempo como\\ máximo de 10 segundos, dado por el intervalo de\\ recolección de métricas, se debe haber detectado el ataque y \\generado la duplicación de tráfico hacia el IDS\\ correspondiente. Por otra parte, a los 120 segundos, los\\ contadores diarios de la cantidad de paquetes y de bytes\\ deben haber modificado su valor ya que pasados los 70\\ segundos el tráfico aumenta gracias al ataque.\\ \end{tabular} \\ \hline
\textbf{\begin{tabular}[c]{@{}c@{}}Resultado \\ obtenido\end{tabular}} & \multicolumn{1}{c|}{\textbf{{\color[HTML]{036400} APROBADO}}} \\ \hline 												 
								
								
	\end{tabular}
	\caption{Test T-A-1.}
	\label{tab:test_A_1}
\end{table}

\begin{table}[th]
	\centering
	
	\begin{tabular}{|c|l|}
		\hline
		\rowcolor[HTML]{EFEFEF}
		\textbf{\begin{tabular}[c]{@{}c@{}}Identificador \\ del test\end{tabular}} & \multicolumn{1}{c|}{\textbf{T-A-2}} \\ \hline
		\textbf{Título}   & \begin{tabular}[c]{@{}l@{}} Identificación del dispositivo de la subcapa de acceso de\\ donde proviene el ataque.\end{tabular} \\ \hline

\rowcolor[HTML]{EFEFEF}
\textbf{Objetivo} & \begin{tabular}[c]{@{}l@{}}Verificar que la detección del dispositivo OVS de la subcapa\\ de acceso de donde proviene el ataque se haga de forma\\ correcta.\\\end{tabular} \\ \hline
					
\textbf{\begin{tabular}[c]{@{}c@{}}Aplicaciones a\\ instalar en el\\ controlador\\ ONOS\end{tabular}} & \begin{tabular}[c]{@{}l@{}}Openflow Provider Suite  (\texttt{org.onosproject.openflow}).\\Aplicación de detección de anomalías. \end{tabular} \\ \hline 
\rowcolor[HTML]{EFEFEF}
\textbf{Pasos previos} & Pasos 1 a 2 del procedimiento de T-E-2. \\ \hline
						
\textbf{Procedimiento} & \begin{tabular}[c]{@{}l@{}}\textbf{1.} Ejecutar en un \textit{bot} un ataque DoS del tipo \textit{TCP SYN flood}\\ hacia el puerto 80 del servidor.\\ \hline \textbf{2.} Observar los \textit{logs} de ONOS.\\ \hline \textbf{3.} Observar la interfaz web de ONOS.\\ \end{tabular} \\ \hline
\rowcolor[HTML]{EFEFEF}
\textbf{\begin{tabular}[c]{@{}c@{}}Resultado \\ esperado\end{tabular}} & \begin{tabular}[c]{@{}l@{}}\textbf{1.} En el \textit{log} de ONOS debe figurar un mensaje en donde se\\ exprese el ID del dispositivo EDGE de donde proviene el\\ ataque.\\\hline\textbf{2.} En la GUI de ONOS el ID del dispositivo obtenido debe\\ concordar con el OVS directamente conectado al \textit{host}\\ atacante.\\ \end{tabular} \\ \hline
\textbf{\begin{tabular}[c]{@{}c@{}}Resultado \\ obtenido\end{tabular}} & \multicolumn{1}{c|}{\textbf{{\color[HTML]{036400} APROBADO}}} \\ \hline 										 
								
								
	\end{tabular}
	\caption{Test T-A-2.}
	\label{tab:test_A_2}
\end{table}


\begin{table}[th]
	\centering
	\begin{tabular}{|c|l|}
		\hline
		\rowcolor[HTML]{EFEFEF}
		\textbf{\begin{tabular}[c]{@{}c@{}}Identificador  \\ del test\end{tabular}} & \multicolumn{1}{c|}{\textbf{T-F-1}} \\ \hline
		\textbf{Título}   & \begin{tabular}[c]{@{}l@{}} Conexión de IDSs al controlador mediante \textit{sockets}.\end{tabular} \\ \hline

		\rowcolor[HTML]{EFEFEF}
		\textbf{Objetivo} & \begin{tabular}[c]{@{}l@{}}Verificar que por lo menos 4 IDSs puedan abrir una conexión\\ \textit{socket} TCP con el controlador ONOS.\end{tabular} \\ \hline
					
		\textbf{\begin{tabular}[c]{@{}c@{}}Aplicaciones a\\ instalar en el\\ controlador\\ ONOS\end{tabular}} & \begin{tabular}[c]{@{}l@{}}Openflow Provider Suite  (\texttt{org.onosproject.openflow}).\\ Aplicación de filtrado del tráfico. \end{tabular} \\ \hline 
		\rowcolor[HTML]{EFEFEF}
		\textbf{Pasos previos} & Pasos 1 a 4 del procedimiento de T-E-1. \\ \hline
						
		\textbf{Procedimiento} & \begin{tabular}[c]{@{}l@{}}\textbf{1.} Poner en marcha los procesos de Snort en los diferentes \\IDSs. Colocar la salida de dichos IDSs a sus \textit{sockets} Unix.\\\hline \textbf{2.} En cada IDS, ejecutar el \textit{script} de Python que realiza la\\ retransmisión a un \textit{socket} TCP conectado al controlador.\\ \end{tabular} \\ \hline
		\rowcolor[HTML]{EFEFEF}
		\textbf{\begin{tabular}[c]{@{}c@{}}Resultado \\ esperado\end{tabular}} & \begin{tabular}[c]{@{}l@{}}El controlador ONOS debe mostrar en sus \textit{logs} la apertura \\de las 4 conexiones vía \textit{sockets} TCP.\end{tabular} \\ \hline
		\textbf{\begin{tabular}[c]{@{}c@{}}Resultado \\ obtenido\end{tabular}} & \multicolumn{1}{c|}{\textbf{{\color[HTML]{036400} APROBADO}}} \\ \hline 		 
		
		
	\end{tabular}
	\caption{Test T-F-1.}
	\label{tab:test_F_1}
\end{table}

\begin{table}[th]
	\centering
	\begin{tabular}{|c|l|}
		\hline
		\rowcolor[HTML]{EFEFEF}
		\textbf{\begin{tabular}[c]{@{}c@{}}Identificador \\ del test\end{tabular}} & \multicolumn{1}{c|}{\textbf{T-F-2}} \\ \hline
		\textbf{Título}   & \begin{tabular}[c]{@{}l@{}} Procesamiento de las alertas de los IDSs en el controlador ONOS.\end{tabular} \\ \hline

\rowcolor[HTML]{EFEFEF}
\textbf{Objetivo} & \begin{tabular}[c]{@{}l@{}}Verificar que el controlador ONOS procesa de manera correcta\\ las alertas generadas por los IDSs.\end{tabular} \\ \hline
					
\textbf{\begin{tabular}[c]{@{}c@{}}Aplicaciones a\\ instalar en el\\ controlador\\ ONOS\end{tabular}} & \begin{tabular}[c]{@{}l@{}}Openflow Provider Suite  (\texttt{org.onosproject.openflow}).\\ Aplicación de detección de anomalías. \\Aplicación de filtrado del tráfico. \end{tabular} \\ \hline 
\rowcolor[HTML]{EFEFEF}
\textbf{Pasos previos} & Pasos 1 a 4 del procedimiento de T-E-1. \\ \hline
						
\textbf{Procedimiento} & \begin{tabular}[c]{@{}l@{}}\textbf{1.} Asegurar que los IDSs tengan en sus archivos de\\ configuración la regla para detectar ataques \textit{TCP SYN flood}.\\\hline\textbf{2.} Poner en marcha los procesos de Snort en los diferentes \\IDSs. Colocar la salida de dichos IDSs a sus \textit{sockets} Unix.\\\hline \textbf{3.} En cada IDS, ejecutar el \textit{script} de Python que realiza la \\retransmisión a un \textit{socket} TCP conectado al controlador.\\\hline \textbf{4.} Pasos 1 a 2 del procedimiento de T-E-2.\\ \hline \textbf{5.} Ejecutar en un bot un ataque DoS del tipo \textit{TCP SYN flood}\\ hacia el puerto 80 del servidor.\\ \end{tabular} \\ \hline
\rowcolor[HTML]{EFEFEF}
\textbf{\begin{tabular}[c]{@{}c@{}}Resultado \\ esperado\end{tabular}} & \begin{tabular}[c]{@{}l@{}}El controlador ONOS debe mostrar en su \textit{log} un mensaje de \\reconocimiento de la alerta generada por el IDS \\correspondiente. A su vez, se deben poder visualizar campos\\ importantes de la misma, como el \texttt{sid}, la dirección IP origen\\ del paquete que provocó el evento, el mensaje de la alerta, etc.\end{tabular} \\ \hline
\textbf{\begin{tabular}[c]{@{}c@{}}Resultado \\ obtenido\end{tabular}} & \multicolumn{1}{c|}{\textbf{{\color[HTML]{036400} APROBADO}}} \\ \hline 		 				 
				
				
	\end{tabular}
	\caption{Test T-F-2.}
	\label{tab:test_F_2}
\end{table}


\begin{table}[th]
	\centering
	\begin{tabular}{|c|l|}
		\hline
		\rowcolor[HTML]{EFEFEF}
		\textbf{\begin{tabular}[c]{@{}c@{}}Identificador \\ del test\end{tabular}} & \multicolumn{1}{c|}{\textbf{T-F-3}} \\ \hline
		\textbf{Título}   & \begin{tabular}[c]{@{}l@{}} Detección y detención de ataques DoS por inundación TCP.\end{tabular} \\ \hline

\rowcolor[HTML]{EFEFEF}
\textbf{Objetivo} & \begin{tabular}[c]{@{}l@{}}Verificar la detección y detención de ataques DoS por\\ inundación TCP al servidor como, por ejemplo, \textit{TCP SYN} \\\textit{flood}, \textit{TCP FIN flood}, \textit{TCP SYN FIN flood}, \textit{TCP RESET flood} y\\ \textit{TCP PUSH ACK flood}.\end{tabular} \\ \hline
					
\textbf{\begin{tabular}[c]{@{}c@{}}Aplicaciones a\\ instalar en el\\ controlador\\ ONOS\end{tabular}} & \begin{tabular}[c]{@{}l@{}}Openflow Provider Suite  (\texttt{org.onosproject.openflow}).\\ Aplicación de detección de anomalías. \\Aplicación de filtrado del tráfico. \end{tabular} \\ \hline 
\rowcolor[HTML]{EFEFEF}
\textbf{Pasos previos} & Pasos 1 a 4 del procedimiento de T-F-2. \\ \hline
						
\textbf{Procedimiento} & \begin{tabular}[c]{@{}l@{}}\textbf{1.} Ejecutar en un \textit{bot} un ataque DoS por inundación TCP\\ hacia el puerto 80 del servidor.\\\hline\textbf{2.} Lanzar desde el mismo \textit{bot} atacante una consulta HTTP\\ GET hacia el \textit{host} víctima.\\\hline \textbf{3.} Después de 20 segundos, cortar el ataque.\\\hline \textbf{4.} Esperar 40 segundos y volver a lanzar otro tipo de ataque\\ por inundación TCP.\\ \hline \textbf{5.} Repetir los pasos 2 a 4. \end{tabular} \\ \hline
\rowcolor[HTML]{EFEFEF}
\textbf{\begin{tabular}[c]{@{}c@{}}Resultado \\ esperado\end{tabular}} & \begin{tabular}[c]{@{}l@{}}El controlador ONOS detecta el ataque por medio del control\\ de las estadísticas de tráfico y genera un bloqueo de los\\ paquetes TCP provenientes del \textit{host} atacante con destino\\ hacia un puerto específico del \textit{host} víctima. Este bloqueo\\ impide las consultas GET desde el mencionado atacante\\ hacia la correspondiente víctima. Luego de 30 segundos de\\ haberse cortado el ataque, la regla OpenFlow de \textit{drop} debe\\ ser eliminada del OVS y la replicación de tráfico hacia el IDS\\ debe detenerse. Esto se puede observar desde los \textit{logs} de\\ ONOS o desde la interfaz web del controlador.\end{tabular} \\ \hline
\textbf{\begin{tabular}[c]{@{}c@{}}Resultado \\ obtenido\end{tabular}} & \multicolumn{1}{c|}{\textbf{{\color[HTML]{036400} APROBADO}}} \\ \hline 		 				 
				
								 
				
				
	\end{tabular}
	\caption{Test T-F-3.}
	\label{tab:test_F_3}
\end{table}

\begin{table}[th]
	\centering
	\begin{tabular}{|c|l|}
		\hline
		\rowcolor[HTML]{EFEFEF}
		\textbf{\begin{tabular}[c]{@{}c@{}}Identificador \\ del test\end{tabular}} & \multicolumn{1}{c|}{\textbf{T-F-4}} \\ \hline
		\textbf{Título}   & \begin{tabular}[c]{@{}l@{}} Detección y detención de ataques DoS por inundación UDP.\end{tabular} \\ \hline

\rowcolor[HTML]{EFEFEF}
\textbf{Objetivo} & \begin{tabular}[c]{@{}l@{}}Verificar la detección y detención de ataques DoS por\\ inundación UDP al servidor.\end{tabular} \\ \hline
					
\textbf{\begin{tabular}[c]{@{}c@{}}Aplicaciones a\\ instalar en el\\ controlador\\ ONOS\end{tabular}} & \begin{tabular}[c]{@{}l@{}}Openflow Provider Suite  (\texttt{org.onosproject.openflow}).\\ Aplicación de detección de anomalías. \\Aplicación de filtrado del tráfico. \end{tabular} \\ \hline 
\rowcolor[HTML]{EFEFEF}
\textbf{Pasos previos} & Pasos 1 a 4 del procedimiento de T-F-2. \\ \hline
						
\textbf{Procedimiento} & \begin{tabular}[c]{@{}l@{}}\textbf{1.} Ejecutar en un \textit{bot} un ataque DoS por inundación UDP\\ hacia el puerto 80 del servidor.\\\hline\textbf{2.} Lanzar desde el mismo \textit{bot} atacante una consulta HTTP\\ GET hacia el \textit{host} víctima.\\\hline \textbf{3.} Después de 20 segundos, cortar el ataque.\\ \hline \textbf{4.} Después de 20 segundos, lanzar nuevamente el ataque.\\ \end{tabular} \\ \hline
\rowcolor[HTML]{EFEFEF}
\textbf{\begin{tabular}[c]{@{}c@{}}Resultado \\ esperado\end{tabular}} & \begin{tabular}[c]{@{}l@{}}El controlador ONOS detecta el ataque por medio del control\\ de las estadísticas de tráfico y genera un bloqueo de los\\ paquetes UDP provenientes del \textit{host} atacante con destino\\ hacia un puerto específico del \textit{host} víctima (se puede ver en la\\ interfaz web de ONOS). Este bloqueo no impide las\\ consultas GET desde el mencionado atacante hacia la\\ correspondiente víctima, debido a que éstas se encuentran\\ sobre TCP. Luego de 20 segundos de haberse cortado el\\ ataque, la regla OpenFlow de \textit{drop} no debe ser eliminada del\\ OVS, ya que se necesitan 30 segundos para ello (ver \textit{logs} del\\ controlador). \end{tabular} \\ \hline
\textbf{\begin{tabular}[c]{@{}c@{}}Resultado \\ obtenido\end{tabular}} & \multicolumn{1}{c|}{\textbf{{\color[HTML]{036400} APROBADO}}} \\ \hline 		 				 
										 
						
						
	\end{tabular}
	\caption{Test T-F-4.}
	\label{tab:test_F_4}
\end{table}


\begin{table}[th]
	\centering
	\begin{tabular}{|c|l|}
		\hline
		\rowcolor[HTML]{EFEFEF}
		\textbf{\begin{tabular}[c]{@{}c@{}}Identificador \\ del test\end{tabular}} & \multicolumn{1}{c|}{\textbf{T-F-5}} \\ \hline
		\textbf{Título}   & \begin{tabular}[c]{@{}l@{}} Detección y detención de ataques Smurf lanzados desde un\\ \textit{bot}.\end{tabular} \\ \hline

\rowcolor[HTML]{EFEFEF}
\textbf{Objetivo} & \begin{tabular}[c]{@{}l@{}}Verificar la detección y detención de ataques Smurf al\\ servidor.\end{tabular} \\ \hline
					
\textbf{\begin{tabular}[c]{@{}c@{}}Aplicaciones a\\ instalar en el\\ controlador\\ ONOS\end{tabular}} & \begin{tabular}[c]{@{}l@{}}Openflow Provider Suite  (\texttt{org.onosproject.openflow}).\\ Aplicación de detección de anomalías. \\Aplicación de filtrado del tráfico. \end{tabular} \\ \hline 
\rowcolor[HTML]{EFEFEF}
\textbf{Pasos previos} & Pasos 1 a 4 del procedimiento de T-F-2. \\ \hline
						
\textbf{Procedimiento} & \begin{tabular}[c]{@{}l@{}}\textbf{1.} Ejecutar en un \textit{bot} un ataque DoS del tipo Smurf hacia el\\ servidor. (\texttt{hping3 -{}-icmp -{}-flood -d 36 -{}-spoof} \\  \texttt{<ip\_servidor>~<ip\_broadcast>}).\\\hline\textbf{2.} Lanzar desde el mismo \textit{bot} atacante un ping hacia el \textit{host}\\ víctima.\\\hline \textbf{3.} Observar en la GUI web de ONOS el recorrido del tráfico\\ generado por el \textit{host} atacante.\\ \hline \textbf{4.} Observar en los \textit{logs} de ONOS en cuál dispositivo Open\\ vSwitch queda finalmente instalada la regla OpenFlow de\\ \textit{drop} y en cuáles se eliminan.\\ \end{tabular} \\ \hline
\rowcolor[HTML]{EFEFEF}
\textbf{\begin{tabular}[c]{@{}c@{}}Resultado \\ esperado\end{tabular}} & \begin{tabular}[c]{@{}l@{}}El controlador ONOS detecta el ataque por medio del control\\ de las estadísticas de tráfico y genera un bloqueo de los\\ paquetes ICMP de tipo 8 provenientes del \textit{host} víctima con\\ destino a la dirección de \textit{broadcast} de la red. En la interfaz web\\ del controlador se debe observar que el tráfico malicioso es\\ cortado por el dispositivo de borde más cercano al \textit{host}\\ atacante. A su vez, este último debe poder realizar el ping con\\ normalidad hacia el servidor, como todos los demás usuarios\\ legítimos y \textit{bots} (solamente se bloquean los paquetes ICMP\\ que actúan como combustible para el ataque Smurf). Por otra\\ parte, la regla de \textit{drop}, instalada en todos los conmutadores,\\ debe quedar solamente, después de 30 segundos, en el OVS\\ de acceso mencionado anteriormente debido a la inactividad\\ de la misma en los demás dispositivos. 
\end{tabular} \\ \hline
\textbf{\begin{tabular}[c]{@{}c@{}}Resultado \\ obtenido\end{tabular}} & \multicolumn{1}{c|}{\textbf{{\color[HTML]{036400} APROBADO}}} \\ \hline 		 				 
										 
												 
						
						
	\end{tabular}
	\caption{Test T-F-5.}
	\label{tab:test_F_5}
\end{table}




\begin{table}[th]
	\centering
	\begin{tabular}{|c|l|}
		\hline
		\rowcolor[HTML]{EFEFEF}
		\textbf{\begin{tabular}[c]{@{}c@{}}Identificador \\ del test\end{tabular}} & \multicolumn{1}{c|}{\textbf{T-F-6}} \\ \hline
		\textbf{Título}   & \begin{tabular}[c]{@{}l@{}} Detección y detención de ataques DDoS  desde distintos OVS de \\acceso.\end{tabular} \\ \hline

\rowcolor[HTML]{EFEFEF}
\textbf{Objetivo} & \begin{tabular}[c]{@{}l@{}}Verificar la detección y detención de ataques DDoS, ya sea de\\ tipo Smurf, inundación TCP o inundación UDP, dirigidos al\\ servidor y lanzados desde 6 \textit{bots} ubicados en distintos OVS de\\ acceso.\end{tabular} \\ \hline
					
\textbf{\begin{tabular}[c]{@{}c@{}}Aplicaciones a\\ instalar en el\\ controlador\\ ONOS\end{tabular}} & \begin{tabular}[c]{@{}l@{}}Openflow Provider Suite  (\texttt{org.onosproject.openflow}).\\ Aplicación de detección de anomalías. \\Aplicación de filtrado del tráfico. \end{tabular} \\ \hline 
\rowcolor[HTML]{EFEFEF}
\textbf{Pasos previos} & Pasos 1 a 4 del procedimiento de T-F-2. \\ \hline
						
\textbf{Procedimiento} & \begin{tabular}[c]{@{}l@{}}\textbf{1.} Ejecutar en 6 \textit{bots} ubicados en distintos OVS de acceso un\\ ataque DDoS, ya sea de tipo Smurf, inundación TCP o\\ inundación UDP hacia el servidor. \\\hline\textbf{2.} Lanzar desde los mismos \textit{bots} atacantes mensajes ICMP o\\ consultas HTTP GET dependiendo del tipo de ataque realizado.\\ (Ver Tablas \ref{tab:test_F_3}, \ref{tab:test_F_4} y \ref{tab:test_F_5}).\\ \end{tabular} \\ \hline
\rowcolor[HTML]{EFEFEF}
\textbf{\begin{tabular}[c]{@{}c@{}}Resultado \\ esperado\end{tabular}} & \begin{tabular}[c]{@{}l@{}}El comportamiento del controlador es similar a lo descrito en\\ los \textit{Tests} T-F-3, T-F-4 y T-F-5, dependiendo del tipo de ataque, ya\\ que lo detecta por medio del control de las estadísticas de\\ tráfico y genera un bloqueo de los paquetes provenientes de los\\ \textit{hosts} atacantes con destino hacia el \textit{host} víctima (visible desde la\\ interfaz web de ONOS). Todas las demás condiciones en cuanto\\ a las consultas HTTP GET y los mensajes ICMP se mantienen\\ constantes. La única diferencia reside en la cantidad de \\dispositivos en donde permanecerán activas las reglas\\ OpenFlow, cuyo contenido es el mismo que en los \textit{tests}\\ anteriores. Además, la aplicación de detección de anomalías\\ debe determinar de manera correcta cuáles son los conmutadores\\ de borde directamente conectados a los atacantes, en caso de\\ inundaciones UDP o TCP.
\end{tabular} \\ \hline
\textbf{\begin{tabular}[c]{@{}c@{}}Resultado \\ obtenido\end{tabular}} & \multicolumn{1}{c|}{\textbf{{\color[HTML]{036400} APROBADO}}} \\ \hline 						 
						
						
	\end{tabular}
	\caption{Test T-F-6.}
	\label{tab:test_F_6}
\end{table}



\begin{table}[th]
	\centering
	\begin{tabular}{|c|l|}
		\hline
		\rowcolor[HTML]{EFEFEF}
		\textbf{\begin{tabular}[c]{@{}c@{}}Identificador \\ del test\end{tabular}} & \multicolumn{1}{c|}{\textbf{T-F-7}} \\ \hline
		\textbf{Título}   & \begin{tabular}[c]{@{}l@{}} Detección y detención de ataques DDoS  desde un mismo OVS\\ de acceso.\end{tabular} \\ \hline

\rowcolor[HTML]{EFEFEF}
\textbf{Objetivo} & \begin{tabular}[c]{@{}l@{}}Verificar la detección y detención de ataques DDoS, ya sea de\\ tipo Smurf, inundación TCP o inundación UDP, dirigidos al\\ servidor y lanzados desde 2 \textit{bots} ubicados detrás del mismo\\ OVS de acceso.\end{tabular} \\ \hline
					
\textbf{\begin{tabular}[c]{@{}c@{}}Aplicaciones a\\ instalar en el\\ controlador\\ ONOS\end{tabular}} & \begin{tabular}[c]{@{}l@{}}Openflow Provider Suite  (\texttt{org.onosproject.openflow}).\\ Aplicación de detección de anomalías. \\Aplicación de filtrado del tráfico. \end{tabular} \\ \hline 
\rowcolor[HTML]{EFEFEF}
\textbf{Pasos previos} & Pasos 1 a 4 del procedimiento de T-F-2. \\ \hline
						
\textbf{Procedimiento} & \begin{tabular}[c]{@{}l@{}}\textbf{1.} Ejecutar en 2 \textit{bots} ubicados detrás del mismo OVS de acceso\\ un ataque DDoS, ya sea de tipo Smurf, inundación TCP o\\ inundación UDP hacia el servidor. \\\hline\textbf{2.} Lanzar desde los mismos \textit{bots} atacantes mensajes ICMP o\\ consultas HTTP GET dependiendo del tipo de ataque realizado.\\ (Ver Tablas \ref{tab:test_F_3}, \ref{tab:test_F_4} y \ref{tab:test_F_5}).\\ \end{tabular} \\ \hline
\rowcolor[HTML]{EFEFEF}
\textbf{\begin{tabular}[c]{@{}c@{}}Resultado \\ esperado\end{tabular}} & \begin{tabular}[c]{@{}l@{}}El comportamiento del controlador es idéntico a lo descrito en\\ la Tabla \ref{tab:test_F_6}.
\end{tabular} \\ \hline
\textbf{\begin{tabular}[c]{@{}c@{}}Resultado \\ obtenido\end{tabular}} & \multicolumn{1}{c|}{\textbf{{\color[HTML]{036400} APROBADO}}} \\ \hline 						 
						
												 
						
						
	\end{tabular}
	\caption{Test T-F-7.}
	\label{tab:test_F_7}
\end{table}





\begin{table}[th]
	\centering
	\begin{tabular}{|c|l|}
		\hline
		\rowcolor[HTML]{EFEFEF}
		\textbf{\begin{tabular}[c]{@{}c@{}}Identificador \\ del test\end{tabular}} & \multicolumn{1}{c|}{\textbf{T-F-8}} \\ \hline
		\textbf{Título}   & \begin{tabular}[c]{@{}l@{}} Detección y detención de ataques DDoS  hacia un cliente \\ cualquiera del ISP.\end{tabular} \\ \hline

\rowcolor[HTML]{EFEFEF}
\textbf{Objetivo} & \begin{tabular}[c]{@{}l@{}}Verificar la detección y detención de ataques DDoS, ya sea de\\ tipo Smurf, inundación TCP o inundación UDP, dirigidos a\\ cualquier cliente del ISP y lanzados desde 6 \textit{bots} ubicados o no\\ en distintos OVS de acceso.\end{tabular} \\ \hline
					
\textbf{\begin{tabular}[c]{@{}c@{}}Aplicaciones a\\ instalar en el\\ controlador\\ ONOS\end{tabular}} & \begin{tabular}[c]{@{}l@{}}Openflow Provider Suite  (\texttt{org.onosproject.openflow}).\\ Aplicación de detección de anomalías. \\Aplicación de filtrado del tráfico. \end{tabular} \\ \hline 
\rowcolor[HTML]{EFEFEF}
\textbf{Pasos previos} & Pasos 1 a 4 del procedimiento de T-F-2. \\ \hline
						
\textbf{Procedimiento} & \begin{tabular}[c]{@{}l@{}}\textbf{1.} Ejecutar en 6 \textit{bots} un ataque DDoS, ya sea de tipo Smurf,\\ inundación TCP o inundación UDP hacia cualquier cliente\\ del ISP. \\\hline\textbf{2.} Lanzar desde los mismos \textit{bots} atacantes mensajes ICMP o\\ consultas HTTP GET dependiendo del tipo de ataque realizado.\\ (Ver Tablas \ref{tab:test_F_3}, \ref{tab:test_F_4} y \ref{tab:test_F_5}).\\ \end{tabular} \\ \hline
\rowcolor[HTML]{EFEFEF}
\textbf{\begin{tabular}[c]{@{}c@{}}Resultado \\ esperado\end{tabular}} & \begin{tabular}[c]{@{}l@{}} 
	El comportamiento del controlador es idéntico a lo descrito en\\ la Tabla \ref{tab:test_F_6}. 
\end{tabular} \\ \hline
\textbf{\begin{tabular}[c]{@{}c@{}}Resultado \\ obtenido\end{tabular}} & \multicolumn{1}{c|}{\textbf{{\color[HTML]{036400} APROBADO}}} \\ \hline 						 
												 
						
						
	\end{tabular}
	\caption{Test T-F-8.}
	\label{tab:test_F_8}
\end{table}




\begin{table}[th]
	\centering
	\begin{tabular}{|c|l|}
		\hline
		\rowcolor[HTML]{EFEFEF}
		\textbf{\begin{tabular}[c]{@{}c@{}}Identificador  \\ del test\end{tabular}} & \multicolumn{1}{c|}{\textbf{T-G-1}} \\ \hline
		\textbf{Título}   & \begin{tabular}[c]{@{}l@{}} Control de funcionalidades de la interfaz gráfica de usuario y de\\ la API.\end{tabular} \\ \hline		
	\rowcolor[HTML]{EFEFEF}
\textbf{Objetivo} & \begin{tabular}[c]{@{}l@{}}Verificar que el subsistema del Capítulo \ref{Chapter6} cumpla con los\\ requerimientos de la Sección \ref{sec:reqs_gui}.\end{tabular} \\ \hline
					
\textbf{\begin{tabular}[c]{@{}c@{}}Aplicaciones a\\ instalar en el\\ controlador\\ ONOS\end{tabular}} & \begin{tabular}[c]{@{}l@{}}Openflow Provider Suite  (\texttt{org.onosproject.openflow}).\\ Aplicación de detección de anomalías. \\Aplicación de filtrado del tráfico. \end{tabular} \\ \hline 
\rowcolor[HTML]{EFEFEF}
\textbf{Pasos previos} & Pasos 1 y 2 del procedimiento de T-F-2. \\ \hline
						
\textbf{Procedimiento} & \begin{tabular}[c]{@{}l@{}} \textbf{1.} Enviar por la GUI las 4 direcciones IP de los IDSs hacia el\\ controlador.\\ \hline \textbf{2.} Paso 3 del procedimiento de T-F-2.\\ \hline \textbf{3.} Enviar las etiquetas de los dispositivos al controlador a través\\ de dicha GUI.\\ \hline \textbf{4.} Enviar los parámetros del comportamiento esperado de la\\ red, obtenidos en la Sección \ref{sec:traffic_legitim} por medio de la interfaz gráfica\\ de usuario.\\ \hline \textbf{5.} Paso 4 del procedimiento de T-F-2.\\ \hline \textbf{6.} A los 20 segundos de haber efectuado el paso 5 del \\procedimiento ejecutar en un \textit{bot} un ataque DoS del tipo \textit{TCP}\\ \textit{SYN flood} hacia el puerto 80 del servidor.\\ \hline \textbf{7.} A los 20 segundos de haber realizado el ataque detenerlo.\\ \hline \textbf{8.} A los 40 segundos reanudar de vuelta el ataque pero con 3\\ bots de distintos OVS de la subcapa de acceso.\\ \hline \textbf{9.} A los 20 segundos de haber realizado el ataque detenerlo.\\ \end{tabular} \\ \hline
\rowcolor[HTML]{EFEFEF}
\textbf{\begin{tabular}[c]{@{}c@{}}Resultado \\ esperado\end{tabular}} & \begin{tabular}[c]{@{}l@{}} 
Al efectuar el primer ítem, en la GUI deben figurar los nuevos\\ detectores agregados. Además, luego del tercer y cuarto paso,\\ la interfaz gráfica de usuario debe exponer los datos agregados\\ en el controlador.
Por otra parte, en los últimos ítems, los\\ distintos ataques deben mostrar sus correspondientes alertas en\\ la interfaz gráfica de usuario. A su vez, las reglas de \textit{drop} que se\\ visualizan en la GUI deben actualizarse dinámicamente, ya que\\ son temporales. Es decir, a los 30 segundos de realizado el paso\\ 7 del procedimiento la regla que mitiga el ataque \textit{TCP SYN flood}\\ debe eliminarse. Además, en dicha GUI se debe poder observar\\ el dispositivo de acceso en donde la mencionada aplicación de\\ detección de anomalías localizó el tráfico sospechoso. Por\\ último, ambos gráficos de la interfaz gráfica de usuario, al\\ momento de la detección de los ataques y antes de que se\\ instalen las reglas OpenFlow, deben presentar un valor alto en\\ los valores que allí se visualizan. 

\end{tabular} \\ \hline
\textbf{\begin{tabular}[c]{@{}c@{}}Resultado \\ obtenido\end{tabular}} & \multicolumn{1}{c|}{\textbf{{\color[HTML]{036400} APROBADO}}} \\ \hline
	\end{tabular}
	\caption{Test T-G-1.}
	\label{tab:test_G_1}
\end{table}


\begin{table}[th]
	\centering
	\begin{tabular}{|c|l|}
		\hline
		\rowcolor[HTML]{EFEFEF}
		\textbf{\begin{tabular}[c]{@{}c@{}}Identificador  \\ del test\end{tabular}} & \multicolumn{1}{c|}{\textbf{T-G-2}} \\ \hline
		\textbf{Título}   & \begin{tabular}[c]{@{}l@{}} Configuración de distintos comportamientos esperados de la\\ red.\end{tabular} \\ \hline		
    \rowcolor[HTML]{EFEFEF}
    \textbf{Objetivo} & \begin{tabular}[c]{@{}l@{}}Verificar que la aplicación de detección de anomalías detecte\\ como sospechoso o no a un determinado comportamiento de la\\ red en función de los parámetros pasados a través de la GUI.\end{tabular} \\ \hline
    \textbf{\begin{tabular}[c]{@{}c@{}}Aplicaciones a\\ instalar en el\\ controlador\\ ONOS\end{tabular}} & \begin{tabular}[c]{@{}l@{}}Openflow Provider Suite  (\texttt{org.onosproject.openflow}).\\ Aplicación de detección de anomalías. \end{tabular} \\ \hline 
    \rowcolor[HTML]{EFEFEF}
    \textbf{Pasos previos} & Paso 1 del procedimiento de T-E-2. \\ \hline
    \textbf{Procedimiento} & \begin{tabular}[c]{@{}l@{}} \textbf{1.} Enviar las etiquetas de los dispositivos al controlador a través\\ de dicha GUI.\\ \hline \textbf{2.} Enviar los parámetros del comportamiento esperado de la\\ red, obtenidos en la Sección \ref{sec:traffic_legitim}, por medio de la interfaz gráfica\\ de usuario.\\ \hline \textbf{3.} Generar el escenario de la mencionada Sección \ref{sec:traffic_legitim}.\\ \hline \textbf{4.} A los 7 minutos modificar los parámetros del comportamiento\\ esperado por un valor en la cantidad de paquetes igual a 10\\ paquetes por minuto en cada OVS de distribución.\\ \hline \textbf{5.} Realizar el paso 3.\\ \end{tabular} \\ \hline
    \rowcolor[HTML]{EFEFEF}
    \textbf{\begin{tabular}[c]{@{}c@{}}Resultado \\ esperado\end{tabular}} & \begin{tabular}[c]{@{}l@{}} 
                                                                               Al generar el escenario de la Sección \ref{sec:traffic_legitim}, con los parámetros del\\ ítem 2 la aplicación no detecta ninguna anomalía. Sin embargo,\\ lo debe hacer cuando bajo el mismo patrón de generación de\\ tráfico legítimo dichos parámetros se modifican. 
\end{tabular} \\ \hline
\textbf{\begin{tabular}[c]{@{}c@{}}Resultado \\ obtenido\end{tabular}} & \multicolumn{1}{c|}{\textbf{{\color[HTML]{036400} APROBADO}}} \\ \hline
	\end{tabular}
	\caption{Test T-G-2.}
	\label{tab:test_G_2}
\end{table}

% Chapter Template
% cSpell:words parencite onfwhitepaper includegraphics resizebox sdncomponents
% linewidth comparqui redireccionar enrutamiento subfigure toposdn Nicira toposinsdn
\chapter{Conclusiones} % Main chapter title 

\label{sec:Conclusiones} % Change X to a consecutive number; for referencing this chapter elsewhere, use \ref{ChapterX}

En el último capítulo del proyecto integrador se enuncian las conclusiones propias del trabajo realizado, ciertas reflexiones en cuanto al crecimiento profesional alcanzado, los posibles trabajos que se pueden desencadenar en un futuro y las limitaciones presentadas.


\section{Conclusiones del proyecto.} \label{sec:conclusionesproyecto}

En los inicios del proyecto siempre se buscó generar una solución escalable y fácilmente adaptable a la naturaleza dinámica que el mercado exige sobre las redes. Esta solución debía consitir en un mecanismo de detección y mitigación de ataques DDoS que se pueda acoplar de manera sencilla a distintas empresas, entre ellos, los proveedores de servicios de Internet. Para lograr esto se buscó entrar en el campo de las redes definidas por software, consiguiendo los siguientes resultados:

\begin{itemize}
    \item{La solución obtenida tiene en cuenta a uno de los controladores SDN con mayor visibilidad comercial.}
    \item{La implementación se verificó a partir del uso de herramientas de
        generación de ataques ampliamente utilizadas.}
    \item{Los sistemas de detección de intrusos usados constan de un software
        con un soporte provisto por una gran comunidad, por lo que se da a
        entender su aceptación y reconocimiento. Esto permite delegar el
        análisis profundo del tráfico sospechoso a estos sistemas por parte del
        controlador SDN.}
    \item{La solución es fácilmente acoplable a cualquier empresa, ya que
        solamente tiene noción del tráfico y de la topología de la misma.}
    \item{La escalabilidad es un punto a favor en este proyecto integrador. Más
        allá de las ventajas que ofrece SDN en este tema, se presenta la
        capacidad de monitorear el tráfico malicioso en el
        plano de datos y no en el de control. Esto posibilita la libertad de
        crecimiento de la red sin contrarrestar la disponibilidad del
        controlador.}
    \item{La solución estadística obtenida consume un ancho de banda mínimo, ya
        que solamente se transmiten métricas cada diez segundos entre los
        dispositivos de red SDN y ONOS.}
    \item{Los sistemas de detección de intrusos no se encuentran permanentemente
        monitoreando la red. Únicamente lo hacen en caso de que el controlador
        detecte una posibilidad de ataque. Esto produce una disminución en el
        número de estos equipos para una misma red.}
    \item{La implementación que se llevó a cabo utiliza como topología de prueba la de un ISP, la cual, frente a determinados cambios, puede seguir siendo protegida por esta solución gracias a las ventajas que ofrece SDN y a las aplicaciones desarrolladas en este proyecto. Ante dichas modificaciones, solamente es necesario reconfigurar el rol de los dispositivos de red, ya sea como puntos de recolección de métricas o como \textit{firewalls}.} 
    
\end{itemize}



\section{Conclusiones personales.} \label{sec:conclusionespersonales}

Un proyecto integrador, como su nombre lo indica, busca integrar los conceptos adquiridos durante los años de la carrera. Durante la misma, se logra entender que los problemas que se nos presentan en cualquier proyecto se deben resolver de manera escalable, segura, económicamente viable y en forma competitiva. Además la solución se debe validar con argumentos y conocimientos técnicos y verificar con casos de prueba acorde a los requerimientos de dicho proyecto. En este trabajo hemos aplicado estos principios y puesto en funcionamiento los conceptos y prácticas adquiridas en las materias relacionadas con redes de computadoras, sistemas operativos, probabilidad y estadística, ingeniería de software, modelos y simulación, entre otras.

Si se hace un recorrido por la vida del proyecto, se apreciará la cantidad de capas de conocimiento que fueron necesarias adquirir para poder realizar y verificar el correcto funcionamiento de las aplicaciones. A continuación se listan algunas reflexiones sobre estas etapas:

\begin{itemize}
    \item {Se logró entender este nuevo paradigma de las redes definidas por software y las ventajas que trae. A su vez, fue necesario adquirir conceptos en cuanto a los controladores SDN disponibles en el mercado y optar por uno. Al tomar la decisión, se debió investigar a fondo su arquitectura y la forma de implementar aplicaciones que logren utilizar sus servicios.}
    \item {Se pudieron obtener los ataques DDoS más utilizados en el último tramo del año 2018. Dichos ataques fueron los que se utilizaron en el entorno de emulación para probar las aplicaciones desarrolladas.}
    \item {En tercer lugar, la explicación sobre Snort brindada en este trabajo es un resumen de todo lo que se aprendió sobre este IDS.}
    \item {Se logró generar los ataques y el tráfico legítimo a partir de la combinación de herramientas y conceptos sobre probabilidad y estadística, protocolos de red, etc.}
    \item {Se consiguió obtener una propuesta superadora respecto a los trabajos que se han detallado en la Sección \ref{sec:state_art}.}
    \item {Para el desarrollo de las aplicaciones se logró adquirir los conocimientos sobre las clases y métodos de Java que ofrece el controlador ONOS. Este proceso fue ayudado en parte por la herramienta Maven.}
    \item {Se ha obtenido conocimiento en herramientas nuevas de Ingeniería de Software como, por ejemplo, Trello.}
    \item {Se logró profundizar la práctica en los lenguajes de programación Java y Python.}
    \item {Se ha podido construir un entorno de emulación con una herramienta flexible como lo es ContainerNet.}
    \item {Utilizando Docker se han conseguido ampliar los conocimientos adquiridos durante el cursado de la carrera.}
    \item {Se adquirió experiencia en la resolución de problemas debido a los diversos desafíos que se nos fueron presentando.}
    \item {Se consiguió obtener un cierto aprendizaje sobre la construcción de APIs.}
    \item{Se logró construir una interfaz gráfica web de usuario, por medio de distintas herramientas y lenguajes que fue necesario aprender.}
\end{itemize}


\section{Posibles trabajos futuros.}

A lo largo del presente proyecto han ido surgiendo y se han propuesto trabajos futuros a realizar. Entre ellos se destacan:

\begin{itemize}
    
\item{\textbf{Publicación científica del trabajo.} Una de los pasos siguientes que se podría realizar sería publicarlo en forma de \textit{paper}.}
\item{\textbf{Mejoras a la aplicación de filtrado.} La posibilidad de detectar más tipos de ataques, debido a que solamente se tienen en cuenta muy pocos, es uno de los factores de mejora.}
\item{\textbf{Mejoras a la aplicación de detección de anomalías.} La introducción de redes neuronales en esta aplicación sería uno de los próximos pasos a realizar para mejorar la detección de los comportamientos sospechosos.}
\item{\textbf{Mejoras a la interfaz gráfica de usuario.} Entre estas mejoras, se destacan la de permitir la clasificación de los clientes en distintos grupos para ofrecerles diferentes niveles de seguridad.}
\item{\textbf{Utilizar equipos SDN físicos.} Dado el costo de los equipos, se hizo imposible adquirir \textit{switches} SDN físicos con capacidad OpenFlow. Sin embargo, agregaría un mayor valor agregado al proyecto y un incremento en su relevancia si se lo aplicara sobre los mencionados dispositivos.}
\item {\textbf{Mejorar las limitaciones.} Posiblemente, una de las cosas más importantes a mejorar son los problemas detalladas en la Sección \ref{sec:limitaciones}.}

\end{itemize}


\section{Limitaciones.} \label{sec:limitaciones}

En esta sección se mostrarán las limitaciones que presenta y que se han encontrado en el proyecto realizado.

\begin{itemize}
\item{\textbf{Ataques de día cero.} Un ataque desconocido es ignorado por las aplicaciones y por los IDS, por lo que la red no tendría forma de mitigarlo.}

\item{\textbf{Desvío de tráfico desde dispositivo OVS de acceso y no desde enlace con tráfico sospechoso.} En la aplicación de detección de anomalías el desvío del mencionado tráfico hacia los IDSs en caso de comportamientos que difieren estadísticamente del esperado se realiza a partir de todos los flujos entrantes a un dispositivo de borde. La idea inicial era, a partir de la detección del OVS de distribución comprometido y de la obtención de su enlace sospechoso que lo conecta con un dispositivo de acceso, duplicar al correspondiente IDS el tráfico de ese enlace solamente. Esto no se puede realizar debido a que los flujos se establecen a través de \textit{intents} (ver Capítulo \ref{sec:Chapter4} para mayor información). Estos son independientes de los enlaces por los que circulan los paquetes, pudiendo establecerse varias rutas para un mismo flujo. Además se definen en base a un punto de origen y uno de destino. Por lo tanto, en un enlace de la zona intermedia de la red no es factible el hecho de determinar cuáles de estos \textit{intents} se encuentran en dicho enlace.}

\item{\textbf{Conmutadores de frontera.} Para disminuir en un principio la complejidad, se optó por enfocarse únicamente en el tráfico que se genera dentro del ISP y no del que proviene desde otro. Para lograr esta última funcionalidad, se podría realizar un análisis estadístico también sobre el OVS de frontera, tratándolo como si fuera uno de distribución.}

\end{itemize}

%----------------------------------------------------------------------------------------
%	THESIS CONTENT - APPENDICES
%----------------------------------------------------------------------------------------

\appendix % Cue to tell LaTeX that the following "chapters" are Appendices

% Include the appendices of the thesis as separate files from the Appendices folder
% Uncomment the lines as you write the Appendices

% Appendix Template

\chapter{Comandos de Hping3 utilizados} % Main appendix title

\label{AppendixA} % Change X to a consecutive letter; for referencing this appendix elsewhere, use \ref{AppendixX}
Los comandos de Hping3 utilizados para la generación de los ataques son los que se muestran en la Tabla \ref{tabla:comandos_hping3}. Para mayor información, consultar en \parencite{hping3_man}.\\


\begin{table}[htbp]
	\centering
	\begin{tabular}{|l|l|}
		\hline
		\cellcolor[HTML]{EFEFEF}\textbf{Ataque}    & \cellcolor[HTML]{EFEFEF}\textbf{Comando}                                                                               \\ \hline
		TCP SYN flood      & 
	\begin{tabular}[c]{@{}l@{}}\texttt{hping3 -S -{}-flood {[}-p puerto destino{]} \textless{}IP} \\ \texttt{host víctima\textgreater {[}-d tamaño del paquete{]}}\end{tabular}         \\ \hline
		UDP flood          & 
	\begin{tabular}[c]{@{}l@{}}\texttt{hping3 -{}-udp -{}-flood {[}-p puerto destino{]} \textless{}IP} \\ \texttt{host víctima\textgreater {[}-d tamaño del paquete{]}}\end{tabular}         \\ \hline
		Smurf              & 
	\begin{tabular}[c]{@{}l@{}}\texttt{hping3 -{}-icmp -{}-flood -{}-spoof \textless{}IP host} \\ \texttt{víctima\textgreater \textless{}IP broadcast\textgreater  {[}-d tamaño del paquete{]}} \\ \texttt{{[} -{}-interval tiempo de espera entre envío de}\\ \texttt{paquetes{]}}\end{tabular} 
		\\ \hline
		TCP RESET flood    & 
	\begin{tabular}[c]{@{}l@{}}\texttt{hping3 -R -{}-flood {[}-p puerto destino{]} \textless{}IP} \\ \texttt{host víctima\textgreater {[}-d tamaño del paquete{]}}\end{tabular}         \\ \hline
		TCP FIN flood      & 
	\begin{tabular}[c]{@{}l@{}}\texttt{hping3 -F -{}-flood {[}-p puerto destino{]} \textless{}IP} \\ \texttt{host víctima\textgreater  {[}-d tamaño del paquete{]}}\end{tabular}         \\ \hline
		TCP SYN FIN flood  & 
	\begin{tabular}[c]{@{}l@{}}\texttt{hping3 -SF -{}-flood {[}-p puerto destino{]} \textless{}IP} \\ \texttt{host víctima\textgreater {[}-d tamaño del paquete{]}}\end{tabular}         \\ \hline
		TCP PUSH ACK flood & 
	\begin{tabular}[c]{@{}l@{}}\texttt{hping3 -PA -{}-flood {[}-p puerto destino{]} \textless{}IP} \\ \texttt{host víctima\textgreater  {[}-d tamaño del paquete{]}}\end{tabular}         \\ \hline
	\begin{tabular}[c]{@{}l@{}} ICMP flood con\\ dirección de origen\\ falsificada y dirección\\ de destino distinta a la\\ de difusión \end{tabular} & 
\begin{tabular}[c]{@{}l@{}}
\texttt{hping3 -{}-icmp -{}-flood \textless{}IP host víctima\textgreater {[}-a IP} \\ \texttt{de origen falsificada{]}}\end{tabular}         \\ \hline
	
	\end{tabular}
	\caption{Comandos Hping3 utilizados.}
	\label{tabla:comandos_hping3}
\end{table}
% Appendix Template

\chapter{Algunas de las características del kernel de Linux} % Main appendix title

\label{AppendixB} % Change X to a consecutive letter; for referencing this appendix elsewhere, use \ref{AppendixX}


\paragraph{Linux namespaces.} 
De acuerdo a \parencite{linux_namespaces}, esta característica del \textit{kernel} de Linux permite  virtualizar y aislar recursos del sistema entre procesos independientes. Esto es muy útil cuando se trata de \textit{containers}, debido a que si no existieran los \textit{namespaces}, por ejemplo, un proceso corriendo en un \textit{container} A podría desmontar el sistema de archivos de un \textit{container} B, ya que dichos recursos no estarían aislados. Existen diferentes tipos de \textit{namespaces}, dependiendo de lo que buscan virtualizar, como son por ejemplo los \textit{Linux network namespaces}, que  aíslan y virtualizan \textit{stacks} y recursos de red, como por ejemplo, las interfaces de red.


\paragraph{Virtual Ethernet devices (veth).}
Teniendo en cuenta a \parencite{veth}, los dispositivos Ethernet virtuales pueden actuar como túneles entre \textit{network namespaces}, permitiendo la  comunicación entre estos, a partir de los dos extremos virtuales del dispositivo Ethernet. Uno de dichos extremos se conecta a una interfaz de red de un \textit{namespace} y  el otro extremo a alguna del otro \textit{namespace} correspondiente, formándose así un enlace de comunicación virtual.


\paragraph{Linux control groups (cgroups).}
De acuerdo a \parencite{cgroups}, son otra característica del \textit{kernel} de Linux, utilizada para la gestión de los recursos. Permite organizar, en una estructura jerárquica de grupos, a procesos cuyo uso de ciertos recursos como, por ejemplo, memoria, CPU, etc., necesita ser limitado, restringido, contabilizado y monitoreado.

\paragraph{Linux bridges.}
Teniendo en cuenta a \parencite{linux_bridge_1} y \parencite{linux_bridge_2}, se trata de un componente del \textit{kernel} de Linux que se comporta como un conmutador virtual de red (dispositivo de capa 2). Permite unir segmentos de una red y se le pueden conectar tantos dispositivos virtuales como reales a sus puertos. No presenta capacidad OpenFlow. 
% Appendix Template

\chapter{Reglas de Snort} % Main appendix title

\label{AppendixC} % Change X to a consecutive letter; for referencing this appendix elsewhere, use \ref{AppendixX}

Si se observa la Figura \ref{fig:arch_snort}, una parte muy importante de ese flujo de datos interno de Snort son las reglas que utiliza el motor de detección para crear las firmas de los ataques. Este IDS permite crearlas al escribirlas en un archivo o también posibilita la elección de aquellas preconfiguradas por parte de la comunidad. La estructura de estas reglas está constituida por una cabecera \parencite{snort_manual}, tal como se observa en la Tabla \ref{tab:cabecera_snort}, y un campo de opciones.

\begin{table}[htpb]
	\centering
    \begin{tabular}{|c|c|c|c|c|c|c|}
	\hline
	\cellcolor[HTML]{EFEFEF}\textbf{Acción} & \cellcolor[HTML]{EFEFEF}\textbf{Protocolo} & \cellcolor[HTML]{EFEFEF}\textbf{\begin{tabular}[c]{@{}c@{}}Red \\ origen\end{tabular}} & \cellcolor[HTML]{EFEFEF}\textbf{\begin{tabular}[c]{@{}c@{}}Puerto \\ origen\end{tabular}} & \cellcolor[HTML]{EFEFEF}\textbf{Dirección} & \cellcolor[HTML]{EFEFEF}\textbf{\begin{tabular}[c]{@{}c@{}}Red \\ destino\end{tabular}} & \cellcolor[HTML]{EFEFEF}\textbf{\begin{tabular}[c]{@{}c@{}}Puerto \\ Destino\end{tabular}} \\ \hline
	alert            & tcp                & any                                    & any & -\textgreater{} & \$HOME\_NET & 80 \\ \hline
    \end{tabular}
    \caption{Cabecera de regla de Snort.}
    \label{tab:cabecera_snort}
\end{table}

Teniendo en cuenta los campos de la cabecera, se describen a continuación algunas de las acciones (valores del campo \textbf{\textit{Acción}}) que pueden realizarse sobre el paquete que coincida con la regla en cuestión:

\begin{itemize}
\item{\textbf{alert:}} generar alerta y registrar el paquete. (Utilizada en este proyecto).
\item{\textbf{pass:}} ignorar el paquete.
\item{\textbf{log:}} registrar el paquete.
\end{itemize}

Por otra parte, el campo \textbf{\textit{Protocolo}} indica el protocolo de comunicación del paquete con el que se quiere coincidir y puede ser \textit{tcp}, \textit{icmp}, \textit{udp} e \textit{ip}. Este último engloba a los 3 anteriores.

Luego, los campos \textbf{\textit{Red origen}} y \textbf{\textit{Red destino}} indican las direcciones IP de origen y destino, respectivamente, de las redes que constituyen los extremos de la comunicación que dio origen al paquete que se quiere hacer coincidir con la regla. En dichos campos, se puede escribir una dirección IP, un conjunto de direcciones IP o una variable que se debe definir en un archivo de configuración de Snort. Ya existen variables por defecto y son:

\begin{itemize}
\item {\textbf{ANY:}} cualquier red.
\item {\textbf{\$EXTERNAL\_NET:}} red externa.
\item {\textbf{\$HOME\_NET:}} red interna.
\end{itemize}

Siguiendo con los campos de la cabecera, el campo \textbf{\textit{Dirección}} puede ser ->, <- ó <>. El mismo indica el sentido de la comunicación.

Por último, los campos \textbf{\textit{Puerto origen}} y \textbf{\textit{Puerto destino}} permiten especificar los puertos de la comunicación. En estos campos se puede escribir el número de puerto, un rango de puertos, \textit{any} (cualquier puerto) o el mencionado número de puerto con un signo de exclamación antepuesto, por ejemplo, !50, lo cual indica cualquiera menos el número 50.	

Además de la cabecera, se encuentra un campo de opciones, las cuales van encerradas entre paréntesis y separadas por punto y coma. Aquellas utilizadas en este proyecto fueron las siguientes \parencite{snort_manual}:

\begin{itemize}
\item{\textbf{msg:}} mensaje que debe mostrar la alerta cuando se active la regla.
\item{\textbf{flow:}} indica a qué flujo de tráfico debe aplicarse la regla. Por ejemplo, \textit{“flow: established”} expresa que se va a aplicar a tráfico propio de conexiones TCP establecidas. Otro ejemplo es \textit{“flow: stateless”}, donde se indica que dicha regla va a ser aplicada a cualquier tipo de tráfico.
\item{\textbf{classtype:}} permite indicar la categoría de la regla, las cuales se definen en un fichero de configuración de Snort. Algunas por defecto son \textit{attempted-dos}, \textit{denial-of-service},  etc.
\item{\textbf{sid:}} identificador de la regla de Snort.
\item{\textbf{rev:}} número de revisión de la regla de Snort.
\item{\textbf{priority:}} indica la prioridad de la regla de Snort.
\item{\textbf{content:}} cadena que Snort debe buscar dentro de la parte útil de un paquete.
\item{\textbf{fast\_pattern:}} en el presente proyecto se lo utilizó acompañado de la palabra \textit{only}. Esto genera un incremento en el rendimiento del motor de detección debido a que el contenido de la opción \textit{content} es enviado directamente a un detector de patrones rápido sin efectuar controles ni evaluaciones innecesarias acerca de algunas otras opciones. Esto es útil cuando se necesita detectar determinado contenido conocido en cualquier parte del \textit{payload} del paquete.
\item{\textbf{threshold:}} esta opción se utiliza cuando se necesita que la regla no se active por cada evento, sino por una cantidad determinada de éstos que se producen en un cierto tiempo. Posee subcampos, los cuales son:
\begin{itemize}
\item{\textbf{type:}} puede ser \textit{limit}, \textit{threshold} o \textit{both}. La más utilizada es \textit{both} debido a que alerta una vez por intervalo de tiempo cuando se observa una cantidad de coincidencias con la regla igual al valor que se encuentra en el subcampo \textit{count}. 
\item{\textbf{track:}} puede ser \textit{by\_dst} o \textit{by\_src}. Respectivamente, define si se realizará una cuenta de coincidencias por cada una de las direcciones IP de destino a las cuales se dirigen los paquetes o por cada una de las direcciones IP de origen de las cuales provengan los mismos.
\item{\textbf{count:}} debe ser un valor distinto de cero y expresa una cantidad de coincidencias con la regla en el intervalo de tiempo definido por el subcampo \textit{seconds}.
\item{\textbf{seconds:}} debe ser un valor distinto de cero y define el intervalo de tiempo en segundos durante los cuales un contador interno del IDS se va a poder incrementar frente a cada coincidencia con la regla. Finalizado este intervalo, dicho contador se resetea.
\end{itemize}
\item{\textbf{flags:}} esta opción es utilizada cuando se necesita en el paquete chequear \textit{flags} TCP en alto. Por ejemplo, \textit{“flags: F”} indica que Snort chequea si el bit TCP FIN tiene un valor igual a uno.
\item{\textbf{itype:}} utilizado para chequear si un paquete ICMP tiene un valor de \textit{ICMP type} determinado.
\end{itemize}

Habiendo definido la sintaxis de las reglas de Snort, se mostrará en \ref{lst:rule_snort} un ejemplo de una de ellas, en la cual se expresa que se generará una alerta por cada paquete ICMP que se envíe desde cualquier red y desde cualquier puerto hacia la red destino definida como \texttt{\$HOME\_NET} y hacia cualquier puerto de destino. En dichas alertas se imprimirá el mensaje \textit{“ICMP test”}. Por otra parte, la regla tendrá un identificador de Snort igual a 1000001, un número de revisión de 1 y pertenecerá a la categoría \textit{icmp-event}.\\


\begin{lstlisting} [label=lst:rule_snort, caption= Regla de Snort., captionpos=b]

alert icmp any any -> $HOME_NET any (msg:"ICMP test"; sid:1000001; rev:1; classtype:icmp-event;)

\end{lstlisting}  


% Appendix Template

\chapter{Tutorial} % Main appendix title

\label{AppendixD} % Change X to a consecutive letter; for referencing this appendix elsewhere, use \ref{AppendixX}

En este apéndice se detallarán los pasos a seguir a modo de tutorial para poder utilizar las soluciones desarrolladas. Cuenta de 4 partes y se lo probó en una máquina con un sistema operativo Ubuntu 18.04. La primera se basa en la generación de las imágenes de Docker y en ejecutar la correspondiente de ONOS. La segunda en la instalación de ContainerNet para poder así levantar la topología. La tercera tiene como finalidad poner a correr las aplicaciones en el controlador y levantar la topología de prueba y la interfaz gráfica de usuario. Por último, la cuarta implica el conexionado de los distintos IDSs con el controlador, la configuración de las etiquetas en los distintos dispositivos de red y la generación del tráfico legítimo y de los ataques. Para lograr todo esto es indispensable clonar el repositorio correspondiente con el siguiente comando:

\begin{lstlisting}[language=,]
git clone https://gitlab.com/sulca/ddosdn.git
\end{lstlisting}


En caso de no tener instalado Docker ni Docker-Compose, seguir los pasos que figuran en \parencite{pasos_instalar_docker} y \parencite{pasos_instalar_docker_compose}, respectivamente.

Otro requisito previo es disponer del software Open vSwitch. Para ello, se debe seguir alguno de los procedimientos de instalación que figuran en \parencite{pasos_instalar_OVS}.



\section {Controlador ONOS y otras imágenes de Docker}

Para empezar, se tienen que ejecutar los comandos que figuran en el Código FUente \ref{lst:comando_repo_onos} para emplear una versión estable del controlador ONOS:\\

\begin{lstlisting} [label=lst:comando_repo_onos, language=, caption= Comandos para poseer una versión estable del controlador., captionpos=b]
    >> cd ~
    >> git clone https://github.com/opennetworkinglab/onos.git
    >> git checkout onos-1.13
\end{lstlisting}

A su vez, se deben insertar y configurar ciertas variables de entorno para el controlador, agregando al archivo \textasciitilde{}/.bashrc las líneas del Bloque \ref{lst:lineasbashrc}.\\

\begin{lstlisting} [label=lst:lineasbashrc, language=,caption= Configuración de variables de entorno., captionpos=b]
    export ONOS_ROOT=~/onos
    source $ONOS_ROOT/tools/dev/bash_profile
\end{lstlisting}

A continuación, hay que situarse sobre el directorio raíz del repositorio con nombre \textit{ddosdn} e ingresar los comandos del Código FUente \ref{lst:comandos_onos} con el fin de ejecutar y levantar dicho controlador. \\

\begin{lstlisting} [label=lst:comandos_onos, language=, caption= Comandos para poner en funcionamiento a ONOS., captionpos=b]
    >> cd ./src/Docker/
    >> sudo docker-compose up onos
\end{lstlisting}



Para ingresar a la CLI del controlador se utiliza el comando
de conexión \texttt{ssh karaf@192.168.60.2 -p 8101} (contraseña: \textit{karaf}), mientras que para observar la interfaz web de ONOS, en un navegador se debe colocar la URL:
\url{http://192.168.60.2:8181/onos/ui/index.html}. Si en cambio se utiliza la dirección \url{http://192.168.60.2:8181/onos/v1/docs/} (nombre de usuario: \textit{karaf}, contraseña: \textit{karaf}), se tiene acceso a la documentación de las APIs que posee dicho controlador.


Luego de esto, se deben generar las imágenes de Docker correspondientes para poder instanciar la topología de prueba. Para ello, manteniendo el mismo directorio se deben ejecutar los comandos del Código FUente \ref{lst:comandos_imagenes_docker}.\\

\begin{lstlisting} [label=lst:comandos_imagenes_docker, caption= Comandos para generar las imágenes de Docker., captionpos=b]
    >> sudo docker-compose build snort
    >> sudo docker-compose build apache
    >> sudo docker-compose build bind
    >> sudo docker-compose build siege
    >> sudo docker-compose build bot
    >> sudo docker-compose build gui
\end{lstlisting}


\section {Instalación de ContainerNet}

Para la instalación de ContainerNet se deben ejecutar los comandos que figuran en el Bloque \ref{lst:containerNet}.\\

\begin{lstlisting} [label=lst:containerNet, language=, caption= Comandos de instalación de ContainerNet., captionpos=b]
    >> git clone https://github.com/containernet/containernet.git
    >> sudo apt update
    >> sudo apt install python-pip 
    >> sudo apt install python-dev 
    >> sudo apt install python-setuptools 
    >> sudo apt install iptables 
    >> sudo apt install build-essential 
    >> pip install --user pytest 
    >> pip install --user docker 
    >> pip install --user python-iptables 
    >> cd containernet/util/
    >> sudo bash ./install.sh
    >> cd .. 
    >> sudo make develop 
    >> sudo mn -c
\end{lstlisting}


\section {Aplicaciones, topología e interfaz gráfica de usuario}

Para poder instalar las aplicaciones en el controlador, primero hay que compilarlas. Para ello se requiere solucionar ciertas dependencias de Java e instalar Maven. En el Código FUente \ref{lst:maven} aparecen los comandos para corregir el primer problema.\\


\begin{lstlisting} [label=lst:maven, language=, caption= Comandos para poder compilar las aplicaciones de Java para Onos., captionpos=b]
    >> sudo apt-get install software-properties-common -y 
    >> sudo add-apt-repository ppa:webupd8team/java -y 
    >> sudo apt-get update 
    >> echo "oracle-java8-installer shared/accepted-oracle-license-v1-1 select true" | sudo debconf-set-selections
    >> sudo apt-get install oracle-java8-installer oracle-java8-set-default -y
\end{lstlisting}




Paso siguiente se debe instalar Maven (versión 3 o superior) con el comando \texttt{sudo apt install maven}.


Habiendo efectuado los pasos anteriores, ahora se tienen que compilar e instalar las aplicaciones, además del instanciado de la topología de prueba. Para ello, en el directorio \textit{./ddosdn/src/ScriptsSH/Topologia/} se debe ejecutar el comando \texttt{bash ./topo3.sh 0}. Luego de que el controlador reconozca todos los \textit{hosts} de dicha topología, ejecutar el mismo comando pero con el parámetro igual a 3 en vez de 0. Con ello, se configura el tiempo en que ONOS determina como caído a un enlace cualquiera de la red, con el fin de evitar derrumbes rápidos de éstos frente a determinados ataques.

Luego, se debe levantar la interfaz gráfica de usuario. Para ello, se tiene que introducir el comando \texttt{sudo docker-compose up gui} en el directorio relativo a la raíz del repositorio
 \textit{./src/Docker/}. Por último, en el navegador, para acceder a la GUI, se debe ingresar la URL \url{http://localhost:5000/}.

\section {Etiquetas, IDSs, tráfico legítimo y ataques}

Una vez en funcionamiento la interfaz gráfica de usuario explicada en el Capítulo \ref{Chapter6}, en la vista de configuración se debe hacer clic sobre el botón \textit{Setear modo por defecto} para la inserción de las etiquetas correspondientes.

Posteriormente, se tienen que activar los procesos de Snort en los distintos
IDSs y efectuar el conexionado vía \textit{sockets} de dichos detectores al controlador.
Esto se efectúa a través de los comandos que figuran en el Código FUente \ref{lst:conexion_socket}. \\


\begin{lstlisting} [label=lst:conexion_socket, language=,caption= Conexión de los IDSs al controlador., captionpos=b]
    >> cd ./src/ScriptsSH/Topologia/
    >> sudo bash ./scriptNodosReTrasnsmisores.sh 4
\end{lstlisting}

Luego, para generar tráfico legítimo por parte de los diferentes \textit{hosts}, ejecutar el comando \texttt{sudo python master\_etapas\_simulacion.py -d 192.168.10.5} en el directorio \textit{./src/Python/Simulacion/}.

Por último, para producir los ataques DDoS por inundación TCP o UDP, se debe utilizar el \textit{script} de Python \textit{master\_hping3.py} que se encuentra en \textit{./src/Python/GenAttack/}. Dicho archivo presenta una ayuda que se puede acceder a través del agregado del parámetro \textit{-h} al comando correspondiente. Allí figuran todas las posibles opciones a utilizar. Un ejemplo podría ser un ataque TCP DDoS SYN flood que se logra al ejecutar \texttt{sudo python master\_hping3.py -t 1 -c 4}.

En cambio, para los ataques DDoS de tipo Smurf, en el mismo directorio usar el \textit{script} master\_smurf.py. 

%----------------------------------------------------------------------------------------
%	BIBLIOGRAPHY
%----------------------------------------------------------------------------------------

\printbibliography[heading=bibintoc]
% 
%----------------------------------------------------------------------------------------

\end{document}  
